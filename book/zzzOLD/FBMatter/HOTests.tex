\chapter{Hypothesis Test Dichotomous Key}
\vspace{-40pt}
This appendix contains a dichotomous key to identify which of the five hypothesis tests examined in this book should be used in a given situation.  This key can be used by answering each question about the hypothesis test situation beginning with the first couplet below and continuing until a couplet says to use a particular type of test.  The two most basic questions of this dichotomous key are to identify the type of variable for the response variable (see \sectref{sect:VarTypes}) and to determine how many populations were sampled.

\begin{enumerate}
  \item $\left\{
    \begin{array}{lll}
      a. & \hbox{If variable is quantitative then $\ldots$} & \hbox{goto 2.} \\
      b. & \hbox{If variable is categorical then $\ldots$} & \hbox{goto 5.}
    \end{array}
    \right.$
  \item $\left\{
    \begin{array}{lll}
      a. & \hbox{If one population was sampled then $\ldots$} & \hbox{goto 3.} \\
      b. & \hbox{If more than one population was sampled then $\ldots$} & \hbox{goto 4.}
    \end{array}
    \right.$
  \item $\left\{
    \begin{array}{lll}
      a. & \hbox{If $\sigma$ is known then $\ldots$} & \hbox{use a \textbf{1-sample Z-test}.} \\
      b. & \hbox{If $\sigma$ is \textbf{UN}known then $\ldots$} & \hbox{use a \textbf{1-sample t-test}.}
    \end{array}
    \right.$
  \item $\left\{
    \begin{array}{lll}
      a. & \hbox{If samples are dependent then $\ldots$} & \hbox{use a \textbf{matched-pairs t-test} (\textit{not discussed in this book}).} \\
      b. & \hbox{If samples are \textbf{IN}dependent then $\ldots$} & \hbox{use a \textbf{2-sample t-test}.}
    \end{array}
    \right.$
  \item $\left\{
    \begin{array}{lll}
      a. & \hbox{If one population was sampled then $\ldots$} & \hbox{use a \textbf{goodness-of-fit test}.} \\
      b. & \hbox{If more than one population was sampled then $\ldots$} & \hbox{use a \textbf{chi-square test}.}
    \end{array}
    \right.$

\end{enumerate}
