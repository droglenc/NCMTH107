\documentclass[10pt,openany]{book}\usepackage[]{graphicx}\usepackage[]{color}
%% maxwidth is the original width if it is less than linewidth
%% otherwise use linewidth (to make sure the graphics do not exceed the margin)
\makeatletter
\def\maxwidth{ %
  \ifdim\Gin@nat@width>\linewidth
    \linewidth
  \else
    \Gin@nat@width
  \fi
}
\makeatother

\definecolor{fgcolor}{rgb}{0.345, 0.345, 0.345}
\newcommand{\hlnum}[1]{\textcolor[rgb]{0.686,0.059,0.569}{#1}}%
\newcommand{\hlstr}[1]{\textcolor[rgb]{0.192,0.494,0.8}{#1}}%
\newcommand{\hlcom}[1]{\textcolor[rgb]{0.678,0.584,0.686}{\textit{#1}}}%
\newcommand{\hlopt}[1]{\textcolor[rgb]{0,0,0}{#1}}%
\newcommand{\hlstd}[1]{\textcolor[rgb]{0.345,0.345,0.345}{#1}}%
\newcommand{\hlkwa}[1]{\textcolor[rgb]{0.161,0.373,0.58}{\textbf{#1}}}%
\newcommand{\hlkwb}[1]{\textcolor[rgb]{0.69,0.353,0.396}{#1}}%
\newcommand{\hlkwc}[1]{\textcolor[rgb]{0.333,0.667,0.333}{#1}}%
\newcommand{\hlkwd}[1]{\textcolor[rgb]{0.737,0.353,0.396}{\textbf{#1}}}%
\let\hlipl\hlkwb

\usepackage{framed}
\makeatletter
\newenvironment{kframe}{%
 \def\at@end@of@kframe{}%
 \ifinner\ifhmode%
  \def\at@end@of@kframe{\end{minipage}}%
  \begin{minipage}{\columnwidth}%
 \fi\fi%
 \def\FrameCommand##1{\hskip\@totalleftmargin \hskip-\fboxsep
 \colorbox{shadecolor}{##1}\hskip-\fboxsep
     % There is no \\@totalrightmargin, so:
     \hskip-\linewidth \hskip-\@totalleftmargin \hskip\columnwidth}%
 \MakeFramed {\advance\hsize-\width
   \@totalleftmargin\z@ \linewidth\hsize
   \@setminipage}}%
 {\par\unskip\endMakeFramed%
 \at@end@of@kframe}
\makeatother

\definecolor{shadecolor}{rgb}{.97, .97, .97}
\definecolor{messagecolor}{rgb}{0, 0, 0}
\definecolor{warningcolor}{rgb}{1, 0, 1}
\definecolor{errorcolor}{rgb}{1, 0, 0}
\newenvironment{knitrout}{}{} % an empty environment to be redefined in TeX

\usepackage{alltt}

%\input{c:/aaaWork/zGnrlLatex/BookPreamble_HC}   % use for the hard-copy version
\input{c:/aaaWork/zGnrlLatex/BookPreamble}
\hypersetup{pdftitle = MTH107 Notes,bookmarksdepth=0}
\input{c:/aaaWork/zGnrlLatex/JustRPreamble}
\usepackage{animate}
\usepackage{titlesec}
\titlespacing\section{0pt}{12pt plus 4pt minus 2pt}{0pt plus 2pt minus 2pt}
\titlespacing\subsection{-3pt}{12pt plus 4pt minus 2pt}{0pt plus 2pt minus 2pt}
\titlespacing\subsubsection{-3pt}{12pt plus 4pt minus 2pt}{0pt plus 2pt minus 2pt}
\renewcommand{\chaptername}{Module}
\IfFileExists{upquote.sty}{\usepackage{upquote}}{}
\begin{document}




  \frontmatter
    %MAKE MINI TABLE OF CONTENTS for each chapter ----------------------------------
\dominitoc
\setcounter{minitocdepth}{1} %sets the depth to show in chapter TOC -- 0 is chapters, 1 would be sections, etc.

\VerbatimFootnotes  % allows verbatim in footnotes

% This is need to make module TOCs but not an overall TOC
\faketableofcontents


  \mainmatter



\chapter{Why Statistics is Important} \label{chap:WhyStatsImportant}

\minitoc

\section{Realities}\label{sect:Realities}
\lettrine{T}{he city of Ashland} performed an investigation in the area of Kreher Park \figrefp{fig:KreherParkMap} when considering the possible expansion of an existing wastewater treatment facility in 1989. The discovery of contamination from creosote waste in the subsoils and ground water at Kreher Park prompted the city to abandon the project. A subsequent assessment by the Wisconsin Department of Natural Resources (WDNR) indicated elevated levels of hazardous substances in soil borings, ground water samples, and in the sediments of Chequamegon Bay directly offshore of Kreher Park. In 1995 and 1999, the Northern States Power Company conducted investigations that further defined the area of contamination and confirmed the presence of specific contaminants associated with coal tar wastes. This site is now listed as a superfund site and is being given considerably more attention.\footnote{More information at the \href{https://cumulis.epa.gov/supercpad/cursites/csitinfo.cfm?id=0507952}{EPA} and the \href{http://dnr.wi.gov/topic/brownfields/ashland.html}{WDNR} websites.}

\begin{figure}[htbp]
  \centering
    \includegraphics[width=6in]{Figs/Kreher_Park_Map.png}
  \caption{Location of the Ashland superfund site (left) with the location of 119 historical sediment sampling sites (right).}
  \label{fig:KreherParkMap}
\end{figure}

The WDNR wants to study elements in the sediment (among other things) in the entire 3000 m$^2$ area shaded in \figref{fig:KreherParkMap}. Is it physically possible to examine every square meter of that area?  Is it prudent, ecologically and economically, to examine every square meter of this area?  The answer, of course, is ``no.''  How then will the WDNR be able to make conclusions about this entire area if they cannot reasonably examine the whole area?  The most reasonable solution is to sample a subset of the area and use the results from this sample to make inferences about the entire area.

Methods for properly selecting a sample that fairly represents a larger collection of individuals are an important area of study in statistics. For example, the WDNR would not want to sample areas that are only conveniently near shore because this will likely not be an accurate representation of the entire area. In this example, it appears that the WDNR used a grid to assure a relatively even dispersal of samples throughout the study area \figrefp{fig:KreherParkMap}. Methods for choosing the number of individuals to select and how to select those individuals are discussed in \modref{chap:DataProd}.


Suppose that the WDNR measured the concentration of lead at each of the 119 locations shown in \figref{fig:KreherParkMap}. Further suppose that they presented their results at a public meeting by simply showing the list of lead concentration measurements \tabrefp{tab:KreherParkPbconc}.\footnote{These are hypothetical data for this site.}  Is it easy to make conclusions about what these data mean from this type of presentation?

\begin{table}[htbp]   \label{tab:KreherParkPbconc}
  \caption{Lead concentration ($\mu g \cdot m^{-3}$) from 119 sites in Kreher Park superfund site.}
  \begin{center}
% latex table generated in R 3.4.0 by xtable 1.8-2 package
% Wed Jun 14 09:31:35 2017
\begin{tabular}{rrrrrrrrrrrrrrr}
  \hline
  \hline
0.91 & 1.09 & 1.00 & 1.09 & 1.06 & 0.98 & 0.98 & 0.94 & 0.89 & 1.09 & 0.91 & 1.06 & 0.81 & 0.90 & 1.21 \\ 
  1.03 & 0.95 & 1.14 & 0.99 & 0.99 & 0.96 & 1.13 & 0.84 & 1.03 & 0.86 & 0.98 & 1.04 & 0.91 & 1.27 & 0.90 \\ 
  0.87 & 1.23 & 1.12 & 0.98 & 0.79 & 1.10 & 1.06 & 1.09 & 0.73 & 0.81 & 1.18 & 0.92 & 0.82 & 1.11 & 0.97 \\ 
  1.24 & 1.06 & 1.09 & 0.78 & 0.94 & 1.08 & 0.91 & 0.98 & 1.22 & 1.04 & 0.77 & 1.18 & 0.93 & 1.14 & 0.94 \\ 
  1.05 & 0.91 & 1.14 & 0.93 & 0.94 & 0.90 & 1.05 & 1.36 & 1.02 & 0.93 & 1.09 & 1.17 & 0.91 & 1.06 & 0.95 \\ 
  0.88 & 0.67 & 1.12 & 1.06 & 0.99 & 0.89 & 0.83 & 0.99 & 1.33 & 1.00 & 1.05 & 1.11 & 1.01 & 1.25 & 0.96 \\ 
  1.07 & 1.17 & 1.01 & 1.20 & 1.17 & 1.05 & 1.21 & 1.10 & 1.07 & 1.01 & 1.16 & 1.24 & 0.86 & 0.90 & 1.07 \\ 
  1.11 & 0.99 & 0.70 & 0.98 & 1.11 & 1.12 & 1.30 & 1.00 & 0.89 & 0.91 & 0.95 & 1.08 & 1.02 & 0.93 &  \\ 
   \hline
\end{tabular}

  \end{center}
\end{table}

\vspace{-12pt}
Instead, suppose that the scientists brought a simple plot of the frequency of observed lead concentrations and brief numerical summaries \figrefp{fig:KreherParkPbhist} to the meeting. With these one can easily see that the measurements were fairly symmetric with no obviously ``weird'' values. The lead concentrations ranged from as low as 0.67 $\mu g \cdot m^{-3}$ to as high as 1.36 $\mu g \cdot m^{-3}$ with the measurements centered on approximately 1.0 $\mu g \cdot m^{-3}$. These summaries are discussed in detail in \modref{chap:UnivEDAQuant}. However, at this point, note that summarizing large quantities of data with few graphical or numerical summaries makes it is easier to identify meaning from data.

\begin{knitrout}
\definecolor{shadecolor}{rgb}{0.922, 0.922, 0.922}\color{fgcolor}\begin{figure}[hbtp]

{\centering \includegraphics[width=.4\linewidth]{Figs/KreherParkPbhist-1} 
\includegraphics[width=.4\linewidth]{Figs/KreherParkPbhist-2} 

}

\caption[Histogram and summary statistics of lead concentration measurements ($\mu g \cdot m^{-3}$) at each of 119 sites in Kreher Park superfund site]{Histogram and summary statistics of lead concentration measurements ($\mu g \cdot m^{-3}$) at each of 119 sites in Kreher Park superfund site.}\label{fig:KreherParkPbhist}
\end{figure}


\end{knitrout}

A critical question at this point is whether or not the results from the one sample of 119 sites perfectly represents the results for the entire area. One way to consider this question is to examine the results obtained from another sample of 119 sites. The results from this second sample \figrefp{fig:KreherParkPbhist1} are clearly, though not radically, different from the results of the first sample. Thus, it is seen that any one sample from a larger whole will not perfectly represent the large whole. This will lead to some uncertainty in our summaries of the larger whole.

\begin{knitrout}
\definecolor{shadecolor}{rgb}{0.922, 0.922, 0.922}\color{fgcolor}\begin{figure}[hbtp]

{\centering \includegraphics[width=.4\linewidth]{Figs/KreherParkPbhist1-1} 
\includegraphics[width=.4\linewidth]{Figs/KreherParkPbhist1-2} 

}

\caption[Kreher Park 1]{Histogram and summary statistics of lead concentration measurements ($\mu g \cdot m^{-3}$) at each of 119 sites (different from the sites shown in \figref{fig:KreherParkPbhist}) in Kreher Park superfund site.}\label{fig:KreherParkPbhist1}
\end{figure}


\end{knitrout}

The results from two different samples do not perfectly agree because each sample contains different individuals (sites in this example), and no two individuals are exactly alike. The fact that no two individuals are exactly alike is \textbf{natural variability}\index{Natural Variability!Definition}, because of the ``natural'' differences that occur among individuals. The fact that the results from different samples are different is called \textbf{sampling variability}\index{Sampling Variability!Definition}. If there was no natural variability, then there would be no sampling variability. If there was no sampling variability, then the field of statistics would not be needed because a sample (even of one individual) would perfectly represent the larger group of individuals. Thus, understanding variability is at the core of statistical practice. Natural and sampling variability will be revisited continuously throughout this course.

This may be unsettling! First, it was shown that an entire area or all of the individuals of interest cannot be examined. It was then shown that a sample of individuals from the larger whole did not perfectly represent the larger whole. Furthermore, each sample is unique and will likely lead to a (slightly) different conclusion. These are all real and difficult issues faced by the practicing scientist and considered by the informed consumer. However, the field of statistics is designed to ``deal with'' these issues such that the results from a relatively small subset of measurements can be used to make conclusions about the entire collection of measurements.

\warn{Statistics provides methods for overcoming the difficulties caused by the requirement of sampling and the presence of sampling variability.}


\section{Major Goals of Statistics}
As seen in the Kreher Park example, the field of statistics has two primary purposes. First, statistics provides methods to summarize large quantities of data into concise and informative numerical or graphical summaries. For example, it was easier to discern the general underlying structure of the lead measurements from the statistics and histograms presented in Figures \ref{fig:KreherParkPbhist} and \ref{fig:KreherParkPbhist1}, than it was from the full list of lead measurements in \tabref{tab:KreherParkPbconc}. Second, statistical methods allow inferences to be made about all individuals (i.e., a population) from a few individuals (i.e., a sample).\footnote{Population and sample are defined more completely in \sectref{sect:IVPPSS}.}


\section{Definition of Statistics}
Statistics is the science of collecting, organizing, and interpreting numerical information or data \citep{MooreMcCabe1998}\index{Statistics, Field of!Definition}. People study statistics for a variety of reasons, including \citep{Bluman2000}:
\begin{Enumerate}
  \item To understand the statistical studies performed in their field (i.e., be knowledgeable about the vocabulary, symbols, concepts, and statistical procedures used in those studies).
  \item To conduct research in their field (i.e., be able to design experiments and samples; collect, organize, analyze, and summarize data; make reliable predictions or forecasts for future use; and communicate statistical results).
  \item To be better consumers of statistical information.
\end{Enumerate}

Statistics permeates a wide variety of disciplines. \cite{MooreMcCabe1998} state:
\begin{quote}
The study and collection of data are important in the work of many professions, so that training in the science of statistics is valuable preparation for a variety of careers. Each month, for example, government statistical offices release the latest numerical information on unemployment and inflation. Economists and financial advisers, as well as policy makers in government and business study these data in order to make informed decisions. Doctors must understand the origin and trustworthiness of the data that appear in medical journals if they are to offer their patients the most effective treatments. Politicians rely on data from polls of public opinion. Business decisions are based on market research data that reveal customer tastes. Farmers study data from field trials of new crop varieties. Engineers gather data on the quality and reliability of manufactured products. Most areas of academic study make use of numbers, and therefore also make use of the methods of statistics.
\end{quote}



\chapter{Foundational Definitions} \label{chap:FoundationalDefinitions}

\minitoc
\vspace{18pt}

\lettrine{S}{tatistical inference is the process} of forming conclusions about a parameter of a population from statistics computed from individuals in a sample.\index{Inference!Definition}\footnote{Formal methods of inference are discussed beginning with \modref{chap:ProbIntro}.} Thus, understanding statistical inference requires understanding the difference between a population and a sample and a parameter and a statistic. And, to properly describe those items, the individual and variable(s) of interest must be identified. Understanding and identifying these six items is the focus of this module.

The following hypothetical example is used throughout this module. Assume that we are interested in the average length of 1015 fish in Square Lake. To illustrate important concepts in this module, assume that all information for all 1015 fish in this lake is known \figrefp{fig:SquareLakePopn}. In ``real life'' this complete information would not be known.

\begin{knitrout}
\definecolor{shadecolor}{rgb}{0.922, 0.922, 0.922}\color{fgcolor}\begin{figure}[hbtp]

{\centering \includegraphics[width=.4\linewidth]{Figs/SquareLakePopn-1} 
\includegraphics[width=.4\linewidth]{Figs/SquareLakePopn-2} 

}

\caption[Schematic representation of individual fish (i.e., dots]{Schematic representation of individual fish (i.e., dots; \textbf{Left}) and histogram (\textbf{Right}) of the total length of the 1015 fish in Square Lake.}\label{fig:SquareLakePopn}
\end{figure}


\end{knitrout}

\section{Definitions} \label{sect:IVPPSS}
The \textbf{individual} in a statistical analysis is one of the ``items'' examined by the researcher.\index{Individual, Definition}  Sometimes the individual is a person, but it may be an animal, a piece of wood, a location, a particular time, or an event. It is extremely important that you don't always visualize a person when considering an individual in a statistical sense. Synonyms for individual are unit, experimental unit (usually used in experiments), sampling unit (usually used in observational studies), case, and subject (usually used in studies involving humans). An individual in the Square Lake example is a fish, because the researcher will collect a set of fish and examine each individual fish.

The \textbf{variable} is the characteristic recorded about each individual.\index{Variable!Definition} The variable in the Square Lake example is the length of each fish. In most studies, the researcher will record more than one variable. For example, the researcher may also record the fish's weight, sex, age, time of capture, and location of capture. In this module, only one variable is considered. In other modules, two variables will be considered.

A \textbf{population} is ALL individuals of interest.\index{Population} In the Square Lake example, the population is all 1015 fish in the lake. The population should be defined as thoroughly as possible including qualifiers, especially those related to time and space, as necessary. This example is simple because Square Lake is so well defined; however, as you will see in the review exercises, the population is often only well-defined by your choice of descriptors.

A \textbf{parameter} is a summary computed from ALL individuals in a population.\index{Parameter!Definition}  The term for the particular summary is usually preceded by the word ``population.'' For example, the population average length of all 1015 fish in Square Lake is 98.06 mm and the population standard deviation is 31.49 mm \tabrefp{tab:SquareLakePopn}.\footnote{We will discuss how to compute and interpret each of these values in later modules.} Parameters are ultimately what is of interest, because interest is in all individuals in the population. However, in practice, parameters cannot be computed because the entire population cannot usually be ``seen.''
% latex table generated in R 3.4.0 by xtable 1.8-2 package
% Wed Jun 14 09:31:36 2017
\begin{table}[ht]
\centering
\caption{Parameters for the total length of ALL 1015 fish in the Square Lake population.} 
\label{tab:SquareLakePopn}
\begin{tabular}{cccccccc}
 n & mean & sd & min & Q1 & median & Q3 & max \\ 
  \hline
1015 & 98.06 & 31.49 & 39 & 72 & 93 & 117 & 203 \\ 
   \hline
\end{tabular}
\end{table}


The entire population cannot be ``seen'' in real life. Thus, to learn something about the population, a subset of the population is usually examined. This subset is called a \textbf{sample}.\index{Sample!Definition} The red dots in \figref{fig:SquareLakeSample1} represent a random sample of n=50 fish from Square Lake (note that the sample size is usually denoted by n).

\begin{knitrout}
\definecolor{shadecolor}{rgb}{0.922, 0.922, 0.922}\color{fgcolor}\begin{figure}[hbtp]

{\centering \includegraphics[width=.4\linewidth]{Figs/SquareLakeSample1-1} 
\includegraphics[width=.4\linewidth]{Figs/SquareLakeSample1-2} 

}

\caption{Schematic representation (\textbf{Left}) of a sample of 50 fish (i.e., red dots) from Square Lake and histogram (\textbf{Right}) of the total length of the 50 fish in this sample.}\label{fig:SquareLakeSample1}
\end{figure}


\end{knitrout}

Summaries computed from individuals in a sample are called \textbf{statistics}.\index{Statistic!Definition}  Specific names of statistics are preceded by ``sample.''  The statistic of interest is always the same as the parameter of interest; i.e., the statistic describes the sample in the same way that the parameter describes the population. For example, if interest is in the population mean, then the sample mean would be computed.

Some statistics computed from the sample from Square Lake are shown in \tabref{tab:SquareLakeSample1} and \figref{fig:SquareLakeSample1}. The sample mean of 100.04 mm is the best ``guess'' at the population mean. Not surprisingly from the discussion in \modref{chap:WhyStatsImportant}, the sample mean does not perfectly equal the population mean.

% latex table generated in R 3.4.0 by xtable 1.8-2 package
% Wed Jun 14 09:31:36 2017
\begin{table}[ht]
\centering
\caption{Summary statistics for the total length of a sample of 50 fish from the Square Lake population.} 
\label{tab:SquareLakeSample1}
\begin{tabular}{cccccccc}
 n & mean & sd & min & Q1 & median & Q3 & max \\ 
  \hline
50 & 100.04 & 31.94 & 49 & 81 & 91 & 118 & 203 \\ 
   \hline
\end{tabular}
\end{table}


\warn{An individual is not necessarily a person.}

\vspace{-12pt}
\warn{Populations and parameters can generally not be ``seen.''}


\section{Performing an IVPPSS}
In each statistical analysis it is important that you determine the Individual, Variable, Population, Parameter, Sample, and Statistic (\textbf{IVPPSS}). First, determine what items you are actually going to look at; those are the individuals. Second, determine what is recorded about each individual; that is the variable. Third, ALL individuals is the population. Fourth, the summary (e.g., mean or proportion) of the variable recorded from ALL individuals in the population is the parameter.\footnote{Again, parameters generally cannot be computed because all of the individuals in the population can not be seen. Thus, the parameter is largely conceptual.} Fifth, the population usually cannot be seen, so only a few individuals are examined; those few individuals are the sample. Finally, the summary of the individuals in the sample is the statistic.

When performing an IVPPSS, keep in mind that parameters describe populations (note that they both start with ``p'') and statistics describe samples (note that they both start with ``s''). This can also be looked at from another perspective. A sample is an estimate of the population and a statistic is an estimate of a parameter. Thus, the statistic has to be the same summary (mean or proportion) of the sample as the parameter is of the population.

The IVPPSS process is illustrated for the following situation:
\vspace{-6pt}
\begin{quote}
\textit{A University of New Hampshire graduate student (and Northland College alum) investigated habitat utilization by New England (Sylvilagus transitionalis) and Eastern (Sylvilagus floridanus) cottontail rabbits in eastern Maine in 2007. In a preliminary portion of his research he determined the proportion of ``rabbit patches'' that were inhabited by New England cottontails. He examined 70 ``patches'' and found that 53 showed evidence of inhabitance by New England cottontails.}
\end{quote}
\vspace{-6pt}

\begin{Itemize}
  \item An individual is a rabbit patch in eastern Maine in 2007 (i.e., a rabbit patch is the ``item'' being sampled and examined).
  \item The variable is ``evidence for New England cottontails or not (yes or no)'' (i.e., the characteristic of each rabbit patch that was recorded).
  \item The population is ALL rabbit patches in eastern Maine in 2007.
  \item The parameter is the proportion of ALL rabbit patches in eastern Maine in 2007 that showed evidence for New England cottontails.\footnote{Note that this population and parameter cannot actually be calculated but it is what the researcher wants to know.}
  \item The sample is the 70 rabbit patches from eastern Maine in 2007 that were actually examined by the researcher.
  \item The statistic is the proportion of the 70 rabbit patches from eastern Maine in 2007 actually examined that showed evidence for New England cottontails. [In this case, the statistic would be 53/70 or 0.757.]
\end{Itemize}

In the descriptions above, take note that the individual is very carefully defined (including stating a specific time (2007) and place (eastern Maine)), the population and parameter both use the word ``ALL'', the sample and statistic both use the specific sample size (70 rabbits), and that the parameter and statistics both use the same summary (i.e., proportion of patches that showed evidence of New England cottontails).

In some situations it may be easier to identify the sample first. From this, and realizing that a sample is always ``of the individuals,'' it may be easier to identify the individual. This process is illustrated in the following example, with the items listed in the order identified rather than in the traditional IVPPSS order.

\vspace{-6pt}
\begin{quote}
\textit{The Duluth, MN Touristry Board is interested in the average number of raptors seen per year at Hawk Ridge.\footnote{Information about Hawk Ridge is found \href{http://www.hawkridge.org/}{here}.}  To determine this value, they collected the total number of raptors seen in a sample of years from 1971-2003.}
\end{quote}
\vspace{-6pt}

\begin{Itemize}
  \item The sample is the 32 years between 1971 and 2003 at Hawk Ridge.
  \item An individual is a year (because a ``sample of \emph{years}'' was taken) at Hawk Ridge.
  \item The variable recorded was the number of raptors seen in one year at Hawk Ridge.
  \item The population is ALL years at Hawk Ridge (this is a bit ambiguous but may be thought of as all years that Hawk Ridge has existed).
  \item The parameter is the average number of raptors seen per year in ALL years at Hawk Ridge.
  \item The statistic is the average number of raptors seen in the 1971-2003 sample of years at Hawk Ridge.
\end{Itemize}

Again, note that the individual is very carefully defined (including stating a specific time and place), the population and parameter both use the word ``ALL'', the sample and statistic both use the specific sample size (32 years), and that the parameter and statistics both use the same summary (i.e., average number of raptors).

\warn{An individual is usually defined by a specific time and place.}

\vspace{-12pt}
\warn{Descriptions for population and parameter will always include the word ``All.''}

\vspace{-12pt}
\warn{Descriptions for sample and statistic will contain the specific sample size.}

\vspace{-12pt}
\warn{Descriptions for parameter and statistic will contain the same summary (usually average/mean or proportion/percentage). Howeve the summary is for a different set of individuals -- the population for the parameter and the sample for the statistic.}


\subsection{Sampling Variability (Revisited)}
It is instructive to once again (see \modref{chap:WhyStatsImportant}) consider how statistics differ among samples. \tabref{tab:SquareLakeSample234} and \figref{fig:SquareLakeSample234} show results from three more samples of n=50 fish from the Square Lake population. The means from all four samples (including the sample in \tabref{tab:SquareLakeSample1} and \figref{fig:SquareLakeSample1}) were quite different from the known population mean of 98.06 mm. Similarly, all four histograms were similar in appearance but slightly different in actual values. These results illustrate that a statistic (or sample) will only approximate the parameter (or population) and that statistics vary among samples. This \textbf{sampling variability} is one of the most important concepts in statistics and is discussed in great detail beginning in \modref{chap:SamplingDist}.\index{Sampling Variability!Definition}

\begin{knitrout}
\definecolor{shadecolor}{rgb}{0.922, 0.922, 0.922}\color{fgcolor}\begin{figure}[hbtp]

{\centering \includegraphics[width=.7\linewidth]{Figs/SquareLakeSample234-1} 

}

\caption{Schematic representation (\textbf{Left}) of three samples of 50 fish (i.e., red dots) from Square Lake and histograms (\textbf{Right}) of the total length of the 50 fish in each sample.}\label{fig:SquareLakeSample234}
\end{figure}


\end{knitrout}

% latex table generated in R 3.4.0 by xtable 1.8-2 package
% Wed Jun 14 09:31:36 2017
\begin{table}[htbp]
\centering
\caption{Summary statistics for the total length in three samples of 50 fish from the Square Lake population.} 
\label{tab:SquareLakeSample234}
\begin{tabular}{cccccccc}
 n & mean & sd & min & Q1 & median & Q3 & max \\ 
  \hline
50 & 99.56 & 32.47 & 57 & 69 & 91 & 123 & 167 \\ 
  50 & 88.64 & 24.52 & 53 & 68 & 86 & 106 & 166 \\ 
  50 & 112.74 & 35.86 & 61 & 84 & 108 & 147 & 174 \\ 
   \hline
\end{tabular}
\end{table}


%\defn{Sampling Variability}{The realization that no two samples are exactly alike. Thus, statistics computed from different samples will likely vary.}

This example also illustrates that parameters are fixed values because populations don't change. If a population does change, then it is considered a different population. In the Square Lake example, if a fish is removed from the lake, then the fish in the lake would be considered a different population. Statistics, on the other hand, vary depending on the sample because each sample consists of different individuals that vary (i.e., sampling variability exists because natural variability exists).

\warn{Parameters are fixed in value, while statistics vary in value.}


\clearpage
\section{Variable Types}\label{sect:VarTypes}
The type of statistic that can be calculated is dictated by the type of variable recorded. For example, an average can only be calculated for quantitative variables (defined below). Thus, the type of variable should be identified immediately after performing an IVPPSS.

\subsection{Variable Definitions}
There are two main groups of variable types -- quantitative and categorical \figrefp{fig:VarTypes}. \textbf{Quantitative} variables are variables with numerical values for which it makes sense to do arithmetic operations (like adding or averaging).\index{Quantitative Variable}  Synonyms for quantitative are measurement or numerical. \textbf{Categorical} variables are variables that record which group or category an individual belongs.\index{Categorical Variable}  Synonyms for categorical are qualitative or attribute. Within each main type of variable are two subgroups \figrefp{fig:VarTypes}.

\begin{knitrout}
\definecolor{shadecolor}{rgb}{0.922, 0.922, 0.922}\color{fgcolor}\begin{figure}[hbtp]

{\centering \includegraphics[width=.5\linewidth]{Figs/VarTypes-1} 

}

\caption[Schematic representation of the four types of variables]{Schematic representation of the four types of variables.}\label{fig:VarTypes}
\end{figure}


\end{knitrout}
\vspace{9pt} % added because of paragraph compressions following R code

The two types of quantitative variables are continuous and discrete variables. \textbf{Continuous} variables are quantitative variables that have an uncountable number of values.\index{Continuous Variable}  In other words, a potential value \textsc{does} exist between every pair of values of a continuous variable. \textbf{Discrete} variables are quantitative variables that have a countable number of values.\index{Discrete Variable}  Stated differently, a potential value \textsc{does not} exist between every pair of values for a discrete variable. Typically, but not always, discrete variables are counts of items.

Continuous and discrete variables are easily distinguished by determining if it is possible for a value to exist between every two values of the variable. For example, can there be between 2 and 3 ducks on a pond?  No!  Thus, the number of ducks is a discrete variable. Alternatively, can a duck weigh between 2 and 3 kg?  Yes!  Can it weigh between 2 and 2.1 kg?  Yes!  Can it weigh between 2 and 2.01 kg?  Yes!  You can see that this line of questions could continue forever; thus, duck weight is a continuous variable.

\warn{A quantitative variable is continuous if a possible value exists between every two values of the variable; otherwise, it is discrete.}

The two types of categorical variables are ordinal and nominal. \textbf{Ordinal} variables are categorical variables where a natural order or ranking exists among the categories.\index{Ordinal Variable}  \textbf{Nominal} variables are categorical variables where no order or ranking exists among the categories.\index{Nominal Variable}

Ordinal and nominal variables are easily distinguished by determining if the order of the categories matters. For example, suppose that a researcher recorded a subjective measure of condition (i.e., poor, average, excellent) and the species of each duck. Order matters with the condition variable -- i.e., condition improves from the first (poor) to the last category (excellent) -- and some reorderings of the categories would not make sense -- i.e., average, poor, excellent does not make sense. Thus, condition is an ordinal variable. In contrast, species (e.g., mallard, redhead, canvasback, and wood duck) is a nominal variable because there is no inherent order among the categories (i.e., any reordering of the categories also ``makes sense'').

\warn{\textbf{Ord}inal means that an \textbf{ord}er among the categories exists (note ``ord'' in both ordinal and order).}

The following are some issues to consider when identifying the type of a variable:
\begin{Enumerate}
  \item The categories of a categorical variable are sometimes labeled with numbers. For example, 1=``Poor'', 3=``Fair'', and 5=``Good''. Don't let this fool you into calling the variable quantitative.
  \item Rankings, ratings, and preferences are ordinal (categorical) variables.
  \item Counts of numbers are discrete (quantitative) variables.
  \item Measurements are typically continuous (quantitative) variables.
  \item It does not matter how precisely quantiative variables are recorded when deciding if the variable is continuous or discrete. For example, the weight of the duck might have been recorded to the nearest kg. However, this was just a choice that was made, the actual values can be continuously finer than kg and, thus, weight is a continuous variable.
  \item Categorical variables that consist of only two levels or categories will be labeled as a nominal variable (because any order of the groups makes sense). This type of variable is also often called ``binomial.''
  \item Do not confuse ``what type of variable'' (answer is one of ``continuous'', ``discrete'', ``nominal'', or ``ordinal'')  with ``what type of variability'' (answer is ``natural'' or ``sampling'') questions.
\end{Enumerate}

\warn{``What type of variable is ...?'' is a different question than ``what type of variability is ...?''  Be careful to note the word difference (i.e., ``variable'' versus ``variability'') when answering these questions.}

\vspace{-12pt}
\warn{The precision to which a quantitative variable was recorded does not determine whether it is continuous or discrete. How precisely the variable COULD have been recorded is the important consideration.}



\chapter{Data Production} \label{chap:DataProd}

\minitoc
\vspace{24pt}

\lettrine{S}{tatistical inference is the process} of making conclusions about a population from the results of a single sample.\index{Inference!Definition} To make conclusions about the larger population, the sample must fairly represent the larger population. Thus, the proper collection (or production) of data is critical to statistics (and science in general). In this module, two ways of producing data -- (1) Experiments and (2) Observational Studies -- are described.

\warn{Inferences cannot be made if data are not properly collected.}


\section{Experiments}
An experiment deliberately imposes a \textit{condition} on individuals to observe the effect on the \textbf{response variable}.\index{Experiment!Definition}\index{Response Variable!Experiment} In a properly designed experiment, all variables that are not of interest are held constant, whereas the variable(s) that is (are) of interest are changed among treatments. As long as the experiment is designed properly (see below), differences among treatments are either due to the variable(s) that were deliberately changed or randomness (chance). Methods to determine if differences were likely due to randomness are developed in later modules. Because we can determine if differences most likely occurred o randomness or changes in the variales, strong \textit{cause-and-effect conclusions} can be made from data collected from carefully designed experiments.

\subsection{Single-factor Experiments}\index{Experiment!Single-Factor}
A \textbf{factor} is a variable that is deliberately manipulated to determine its effect on the response variable.\index{Factor!Experiment} A factor is sometimes called an \textbf{explanatory variable} because we are attempting to determine how it affects (or ``explains'') the response variable. The simplest experiment is a single-factor experiment where the individuals are split into groups defined by the categories of a single factor.

For example, suppose that a researcher wants to examine the effect of temperature on the total number of bacterial cells after two weeks. They have inoculated 120 agars\footnote{An agar, in this case, is a petri dish with a growth medium for the bacteria.} with the bacteria and placed them in a chamber where all environmental conditions (e.g., temperature, humidity, light) are controlled exactly. The researchers will use only two temperatures in this simple experiment -- $10^{o}$C and $15^{o}$C. All other variables are maintained at constant levels. Thus, temperature is the only factor in this simple experiement because it is the only variable manipulated to different values to determine its impact on the number of bacterial cells.

\warn{In a single-factor experiment only one explanatory variable (i.e., factor) is allowed to vary; all other explanatory variables are held constant.}

\textbf{Levels} are the number of categories of the factor variable.\index{Level!Experimental} In this example, there are two levels -- $10^{o}$C and $15^{o}$C. \textbf{Treatments} are the number of unique conditions that individuals in the experiment are exposed to. In a single-factor experiment, the number of treatments is the same as the number of levels of the single factor.\index{Treatment, Experimental} Thus, in this simple experiment, there are two treatments -- $10^{o}$C and $15^{o}$C. Treatments are discussed more thoroughly in the next section.

The \textbf{number of replicates} in an experiment is the number of individuals that will receive each treatment.\index{Replicates} In this example, a replicate is an inoculated agar. The number of replicates is the number of inoculated agars that will receive each of the two temperature treatments. The number of replicates is determined by dividing the total number of available individuals (120) by the number of treatments (2). Thus, in this example, the number of replicates is 60 inoculated agars.

The agars used in this experiment will be randomly allocated to the two temperature treatments. All other variables -- humidity, light, etc. -- are kept the same for each treatment. At the end of two weeks, the total number of bacterial cells on each agar (i.e., the response variable) will be recorded and compared between the agars kept at both temperatures.\footnote{Methods for making this comparison are in \modref{chap:tTest2}.} Any difference in mean number of bacterial cells will be due to either different temperature treatments or randomness, because all other variables were the same between the two treatments.

\warn{Differences among treatments are either caused by randomness (chance) or the factor.}

The single factor is not restricted to just two levels. For example, more than two temperatures, say $10^{o}$C, $12.5^{o}$C, $15^{o}$C, and $17.5^{o}$C, could have been tested. With this modification, there is still only one factor -- temperature -- but there are now four levels (and only four treatments).

\subsection{Multi-factor Experiments -- Design and Definitions}
More than one factor can be tested in an experiment.\index{Experiment!Multi-Factor} In fact, it is more efficient to have a properly designed experiment where more than one factor is varied at a time than it is to use separate experiments in which only one factor is varied in each. However, before showing this benefit, let's examine the definitions from the previous section in a multi-factor experiment.

Suppose that the previous experiment was modified to also examine the effect of relative humidity on the number of bacteria cells. This modified experiment has two factors -- temperature (with two levels of $10^{o}$C or $15^{o}$C) and relative humidity (with four levels of 20\%, 40\%, 60\%, and 80\%). The number of treatments, or combinations of all factors, in this experiment is found by multiplying the levels of all factors (i.e., 2$\times$4=8 in this case).\index{Treatment, Experimental} The number of replicates in this experiment is now 15 (i.e., total number of available agars divided by the number of treatments; 120/8).

\warn{The number of treatments is determined for the overall experiment, whereas the number of levels is determined for each factor.}

A drawing of the experimental design can be instructive (below). The drawing is a grid where the levels of one factor are the rows and the levels of the other factor are the columns. The number of rows and columns correspond to the levels of the two factors, respectively, whereas the number of cells in the grid is the number of treatments (numbered in this table to show eight treatments).

\begin{center}
\begin{tabular}{cc|c|c|c}
 & \multicolumn{4}{c}{Relative Humidity} \\
\cline{2-5}
 & 20\% & 40\% & 60\% & 80\% \\
\cline{2-5}
\multicolumn{1}{c|}{$10^{o}$C} & 1 & 2 & 3 & \multicolumn{1}{c|}{4} \\
\hline
\multicolumn{1}{c|}{$15^{o}$C} & 5 & 6 & 7 & \multicolumn{1}{c|}{8} \\
\cline{2-5}
\end{tabular}
\end{center}


\subsection{Multi-factor Experiments -- Benefits}
The analysis of a multi-factor experimental design is more involved than what will be shown in this course. However, multi-factor experiments have many benefits, which can be illustrated by comparing a multi-factor experiment to separate single-factor experiments. For example, in addition to the two factor experiment in the previous section, consider separate single-factor experiments to determine the effect of each factor separately (further assume that individuals (i.e., agars) can be used in only one of these separate experiments).

To conduct the two separate experiments, randomly split the 120 available agars into two equally-sized groups of 60. The first 60 will be split into two groups of 30 for the first experiment with two temperatures. The second 60 will be split into four groups of 15 for the second experiment with four relative humidities. These separate single-factor experiments are summarized in the following tables (where the numbers in the cells represent the number of replicates in each treatment).

\begin{center}
\begin{tabular}{|c|c|c|c|c|c|c|}
\multicolumn{2}{c}{Temperature} & \multicolumn{1}{c}{} & \multicolumn{4}{c}{Relative Humidity} \\
\cline{1-2}\cline{4-7}
$10^{o}$C & $15^{o}$C & & 20\% & 40\% & 60\% & 80\% \\
\cline{1-2}\cline{4-7}
30 & 30 & & 15 & 15 & 15 & 15 \\
\cline{1-2}\cline{4-7}
\end{tabular}
\end{center}

The tabel below was modified from the previous section to show the number of replicates in each treatment of the experiment where both factors were simultaneously manipulated.

\begin{center}
\begin{tabular}{cc|c|c|c}
 & \multicolumn{4}{c}{Relative Humidity} \\
\cline{2-5}
 & 20\% & 40\% & 60\% & 80\% \\
\cline{2-5}
\multicolumn{1}{c|}{$10^{o}$C} & 15 & 15 & 15 & \multicolumn{1}{c|}{15} \\
\hline
\multicolumn{1}{c|}{$15^{o}$C} & 15 & 15 & 15 & \multicolumn{1}{c|}{15} \\
\cline{2-5}
\end{tabular}
\end{center}

The key to examining the benefits of the multi-factor experiment is to determine the number of individuals that give ``information'' about (i.e., are exposed to) each factor. From the last table it is seen that all 120 individuals were exposed to one of the temperature levels with 60 individuals exposed to each level. In contrast, only 30 individuals were exposed to these levels in the single-factor experiment. In addition, all 120 individuals were exposed to one of the relative humidity levels with 30 individuals exposed to each level. Again, this is in contrast to the single-factor experiment where only 15 individuals were exposed to these levels. Thus, the first advantage of multi-factor experiments is that the available individuals are used more efficiently. In other words, more ``information'' (i.e., the responses of more individuals) is obtained from a multi-factor experiment than from combinations of single-factor experiments.\footnote{The real importance of this advantage will become apparent when statistical power is introduced in \modref{chap:HypothesisTests}.}

A properly designed multi-factor experiment also allows researchers to determine if multiple factors interact to impact an individual's response.\index{Interaction effect} For example, consider the hypothetical results from this experiment in \figref{fig:ExpDInt}.\footnote{The means of each treatment are plotted and connected with lines in this plot.} The effect of relative humidity is to increase the growth rate for those individuals at $10^{o}$C (black line) but to decrease the growth rate for those individuals at $15^{o}$C (blue line). That is, the effect of relative humidity differs depending on the level of temperature. When the effect of one factor differs depending on the level of the other factor, then the two factors are said to \textit{interact}. Interactions cannot be determined from the two single-factor experiments because the same individuals are not exposed to levels of the two factors at the same time.

\begin{knitrout}
\definecolor{shadecolor}{rgb}{0.922, 0.922, 0.922}\color{fgcolor}\begin{figure}[hbtp]

{\centering \includegraphics[width=.4\linewidth]{Figs/ExpDInt-1} 

}

\caption[Mean growth rates in a two-factor experiment that depict an interaction effect]{Mean growth rates in a two-factor experiment that depict an interaction effect.}\label{fig:ExpDInt}
\end{figure}


\end{knitrout}
\vspace{9pt} % added because of paragraph compressions following R code

Multi-factor experiments are used to detect the presence or absence of interaction, not just the presence of it. The hypothetical results in \figref{fig:ExpDNoInt} show that the growth rate increases with increasing relative humidity at about the same rate for both temperatures. Thus, because the effect of relative humidity is the same for each temperature (and vice versa), there does not appear to be an interaction between the two factors. Again, this could not be determined from the separate single-factor experiments.

\begin{knitrout}
\definecolor{shadecolor}{rgb}{0.922, 0.922, 0.922}\color{fgcolor}\begin{figure}[hbtp]

{\centering \includegraphics[width=.4\linewidth]{Figs/ExpDNoInt-1} 

}

\caption[Mean growth rates in a two-factor experiment that depict no interaction effect]{Mean growth rates in a two-factor experiment that depict no interaction effect.}\label{fig:ExpDNoInt}
\end{figure}


\end{knitrout}


\subsection{Allocating Individuals}
Individuals\footnote{When discussing experiments, an ``individual'' is often referred to as a ``replicate'' or an ``experimental unit.''} should be randomly allocated (i.e., placed into) to treatments.\index{Replicates} Randomization will tend to even out differences among groups for variables not considered in the experiment. In other words, randomization should help assure that all groups are similar before the treatments are imposed. Thus, randomly allocating individuals to treatments removes any bias (foreseen or unforeseen) from entering the experiment.

In the single-factor experiment above -- two treatments of temperature -- there were 120 agars. To randomly allocate these individuals to the treatments, 60 pieces of paper marked with ``10'' and 60 marked with ``15'' could be placed into a hat. One piece of paper would be drawn for each agar and the agar would receive the temperature found on the piece of paper. Alternatively, each agar could be assigned a unique number between 1 and 120 and pieces of paper with these numbers could be placed into the hat. Agars corresponding to the first 60 numbers drawn from the hat could then be placed into the first treatment. Agars for the next (or remaining) 60 numbers would be placed in the second treatment. This process is essentially the same as randomly ordering 120 numbers.

A random order of numbers is obtained with R by including the count of numbers as the only argument to \R{sample()}. For example, randomly ordering 1 through 120 is accomplished with

\begin{knitrout}
\definecolor{shadecolor}{rgb}{0.922, 0.922, 0.922}\color{fgcolor}\begin{kframe}
\begin{verbatim}
> sample(120)
\end{verbatim}
\end{kframe}
\end{knitrout}
\vspace{-12pt}
\begin{knitrout}
\definecolor{shadecolor}{rgb}{0.922, 0.922, 0.922}\color{fgcolor}\begin{kframe}
\begin{verbatim}
  [1]  80  30 100  90  21  68 104  79  64 106  98  16  73  91 107   1  60  54  26  99
 [21] 108 111  31  47  57  92   5  58  37  50  34  88  41  66  65  29 110 113   4  75
 [41]  93  23  49  97  35  84  74   7  15  39  70  94 114  14  71  20  33  67  86   8
 [61]   6  28  52  48  13  18  63  72  69 120  55  83  42   3  77  82  38  22  96  43
 [81]  56  89  78  17 112  44 103  46  59  85 109 115 118  87  32  62  51  95  24  40
[101] 119 102  19  27 116  36   2  12  45  53  11  76 117  61 105   9 101  25  81  10
\end{verbatim}
\end{kframe}
\end{knitrout}

Thus, the first five (of 60) agars in the 10$^{o}$C treatment are 80, 30, 100, 90, and 21. The first five (of 60) agars in the 15$^{o}$C treatment are 6, 28, 52, 48, and 13.

In the modified experiment with two factors -- temperature and relative humidity -- with eight treatments containing 15 agars each, it is more efficient to save the random numbers into an object and then select the numbers in the first 15 positions, then the second 15 positions, etc. Positions are selected from an object by putting the position numbers in square brackets following the object name. Additionally, a colon is used to make a sequence of integers from the number before to the number after the colon.\footnote{For example, \R{1:4} will make an object with the numbers 1, 2, 3, and 4 in it.}
\begin{knitrout}
\definecolor{shadecolor}{rgb}{0.922, 0.922, 0.922}\color{fgcolor}\begin{kframe}
\begin{verbatim}
> ragars2 <- sample(120)
> ragars2[1:15]     # "grab" the first 15 numbers
 [1]  61  82 103  31  66  81 105  40 104 106   5   9  71  36   8
> ragars2[16:30]    # "grab" the second 15 numbers, and so on
 [1] 120   6  26  41  62 111  83  20  57   1  63  86  70  85  73
\end{verbatim}
\end{kframe}
\end{knitrout}

This design might be shown with the following table, where the numbers in each cell represent the first two agars selected to receive that treatment.\footnote{Only the first two numbers are shown because of space constraints.}

\begin{center}
\begin{tabular}{cc|c|c|c}
 & \multicolumn{4}{c}{Relative Humidity} \\
\cline{2-5}
 & 20\% & 40\% & 60\% & 80\% \\
\cline{2-5}
\multicolumn{1}{c|}{$10^{o}$C} & 61,82,$\cdots$ & 120,6,$\cdots$ & 60,72,$\cdots$ & \multicolumn{1}{c|}{89,49,$\cdots$} \\
\hline
\multicolumn{1}{c|}{$15^{o}$C} & 78,10,$\cdots$ & 109,101,$\cdots$ & 22,2,$\cdots$ & \multicolumn{1}{c|}{114,77,$\cdots$} \\
\cline{2-5}
\end{tabular}
\end{center}

\warn{Individuals should be randomly allocated to treatments to remove bias.}


\subsection{Design Principles}
There are many other methods of designing experiments and allocating individuals that are beyond the scope of this book.\footnote{Other common designs include blocked, Latin square, and nested designs.} However, all experimental designs contain the following three basic principles.\index{Experiment!Principles}
\begin{Enumerate}
  \item \textbf{Control} the effect of variables on the response variable by deliberately manipulating factors to certain levels and maintaining constancy among other variables.
  \item \textbf{Randomize} the allocation of individuals to treatments to eliminate bias.
  \item \textbf{Replicate individuals} (use many individuals) in the experiment to reduce chance variation in the results.
\end{Enumerate}

Proper control in an experiment allows for strong cause-and-effect conclusions to be made (i.e., to state that an observed difference in the response variable was due to the levels of the factor or chance variation rather than some other foreseen or unforeseen variable). Randomly allocating individuals to treatments removes any bias that may be included in the experiment. For example, if we do not randomly allocate the agars to the treatments, then it is possible that a set of all ``poor'' agars may end up in one treatment. In this case, any observed differences in the response may not be due to the levels of the factor but to the prior quality of the agars. Replication means that there should be more than one or a few individuals in each treatment. This reduces the effect of each individual on the overall results. For example, if there was one agar in each treatment, then, even with random allocation, the effect of that treatment may be due to some inherent properties of that agar rather than the levels of the factors. Replication, along with randomization, helps assure that the groups of individuals in each treatment are as alike as possible at the start of the experiment.


\section{Observational Studies -- Sampling}
In observational studies the researcher has no control over any of the variables observed for an individual.\index{Observational Study} The researcher simply observes individuals, disturbing them as little as possible, trying to get a ``picture'' of the population. Observational studies cannot be used to make cause-and-effect statements because all variables that may impact the outcome may not have been measured or specifically controlled. Thus, any observed difference among groups may be caused by the variables measured, some other unmeasured variables, or chance (randomness).

Consider the following as an example of the problems that can occur when all variables are not measured. For many years scientists thought that the brains of females weighed less than the brains of males. They used this finding to support all kinds of ideas about sex-based differences in learning ability. However, these earlier researchers failed to measure body weight, which is strongly related to brain weight in both males and females. After controlling for the effect of differences in body weights, there was no difference in brain weights between the sexes. Thus, many sexist ideas persisted for years because cause-and-effect statements were inferred from data where all variables were not considered.

\warn{Strong cause-and-effect statements CANNOT be made from observational studies.}

In observational studies, it is important to understand to which population inferences will refer.\footnote{Thus, it is very important to first perform an IVPPS as discussed in \modref{chap:FoundationalDefinitions}.} To make useful inferences from a sample, the sample must be an unbiased representation of the population. In other words, it must not systematically favor certain individuals or outcomes.

For example, consider that you want to determine the mean length of all fish in a particular lake (e.g., Square Lake from \modref{chap:FoundationalDefinitions}). Using a net with large mesh, such that only large fish are caught, would produce a biased sample because interest is in all fish not just the large fish. Setting the nets near spawning beds (i.e., only adult fish) would also produce a biased sample. In both instances, a sample would be collected from a population other than the population of interest. Thus it is important to select a sample from the specified population.

\warn{It is important to understand the population before considering how to take a sample.}

\subsection{Types of Sampling Designs}
Three common types of sampling designs -- voluntary response, convenience, and probability-based samples -- are considered in this section. Voluntary response and convenience samples tend to produce biased samples, whereas proper probability-based samples will produce an unbiased sample.

A \textbf{voluntary response} sample consists of individuals that have chosen themselves for the sample by responding to a general appeal.\index{Voluntary Response Sample} An example of a voluntary response sample would be the group of people that respond to a general appeal placed in the school newspaper. If the population of interest in this sample was all students at the school, then this type of general appeal would likely produce a biased sample of students that (i) read the school newspaper, (ii) feel strongly about the topic, or (iii) both.

A \textbf{convenience} sample consists of individuals who are easiest to reach for the researcher.\index{Convenience Sample} An example of a convenience sample is when a researcher queries only those students in a particular class. This sample is ``convenient'' because the individuals are easy to gather. However, if the population of interest was all students at the school, then this type of sample would likely produce a biased sample of students that is likely of (i) one major or another, (ii) one or a few ``years-in-school'' (e.g., Freshman or Sophomores), or (iii) both.

In probability-based sampling, each individual of the population has a known chance of being selected for the sample. The simplest probability-based sample is the \textbf{Simple Random Sample} (SRS) where each individual has the same chance of being selected.\index{Simple Random Sample} Proper selection of an SRS requires each individual to be assigned a unique number. The SRS is then formed by choosing random numbers and collecting the individuals that correspond to those numbers.

For example, an auditor may need to select a sample of 30 financial transactions from all transactions of a particular bank during the previous month. Because each transaction is numbered, the auditor may know that there were 1112 transactions during the previous month (i.e., the population). The auditor would then number each transaction from 1 to 1112, randomly select 30 numbers (with no repeats) from between 1 and 1112, and then physically locate the 30 transactions that correspond to the 30 selected numbers. Those 30 transactions are the SRS.

Random numbers are selected in R by including the population size as the first and sample size as the second argument to \R{sample()}. For example, 30 numbers from between 1 and 1112 is selected with

\begin{knitrout}
\definecolor{shadecolor}{rgb}{0.922, 0.922, 0.922}\color{fgcolor}\begin{kframe}
\begin{verbatim}
> sample(1112,30)
\end{verbatim}
\end{kframe}
\end{knitrout}
\vspace{-12pt}
\begin{knitrout}
\definecolor{shadecolor}{rgb}{0.922, 0.922, 0.922}\color{fgcolor}\begin{kframe}
\begin{verbatim}
 [1]   75  320  874  104  128  870  607 1091 1030 1053 1031  518  433  893  816  903
[17]  342 1016  136  580  670  376  576 1076 1034  365  492  189  409   66
\end{verbatim}
\end{kframe}
\end{knitrout}

Thus, accounts 75, 320, 874, 104, and 128 would be the first five (of 30) selected.

There are other more complex types of probability-based samples that are beyond the scope of this course.\footnote{For example, stratified samples, nested, and multistage samples.} However, the goal of these more complex types of samples is generally to impart more control into the sampling design.

\warn{A proper SRS requires each individual i the population to be assigned a unique number.}

If the population is such that a number cannot be assigned to each individual, then the researcher must try to use a method for which they feel each individual has an equal chance of being selected. Usually this means randomizing the technique rather than the individuals. In the fish example discussed on the previous page, the researcher may consider choosing random mesh sizes, random locations for placing the net, or random times for placing the net. Thus, in many real-life instances, the researcher simply tries to use a method that is likely to produce an SRS or something very close to it.

\warn{If a number cannot be assigned to each individual in the population, then the researcher should randomize the ``technique'' to assure as close to a random sample as possible.}

Polls, campaign or otherwise, are examples of observational studies that you are probably familiar with. The following are links where various aspects of polling are discussed.
\begin{Itemize}
  \item \href{http://media.gallup.com/PDF/FAQ/HowArePolls.pdf}{How Polls are Conducted by Frank Newport, Lydia Saad, and David Moore, The Gallup Organization}.
  \item \href{http://www2.psych.purdue.edu/~codelab/Invalid.Polls.html}{Why Do Campaign Polls Zigzag So Much? by G.S. Wasserman, Purdue U}.
\end{Itemize}


\subsection{Of What Value are Observational Studies?}
Properly designed experiments can lead to ``cause-and-effect'' statements, whereas observational studies (even properly designed) are unlikely to lead to such statements. Furthermore, in the last section, it was suggested that it is very difficult to take a proper probability-based sample because it is hard to assign a number to each individual in the population (precisely because entire populations are very difficult to ``see''). So, do observational studies have any value?  There are at least three reasons why observational studies are useful.

The scientific method begins with making an observation about a natural phenomenon. Observational studies may serve to provide such an observation. Alternatively, observational studies may be deployed after an observation has been made to see if that observation is ``prevalent'' and worthy of further investigation. Thus, observational studies may lead directly to hypotheses that form the basis of experiments.

Experiments are often conducted under very confined and controlled conditions so that the effect of one or more factors on the response variable can be identified. However, at the conclusion of an experiment it is often questioned whether a similar response would be observed ``in nature'' under much less controlled conditions. For example, one might determine that a certain fertilizer increases growth of a certain plant in the greenhouse, with consistent soil characteristics, temperatures, lighting, etc. However, it is a much different, and, perhaps, more interesting, question to determine if that fertilizer elicits the same response when applied to an actual field.

Finally, there are situations where conducting an experiment simply cannot be done, either for ethical, financial, size, or other constraints. For example, it is generally accepted that smoking causes cancer in humans even though an experiment where one group of people was forced to smoke while another was not allowed to smoke has not been conducted. Similarly, it is also very difficult to perform valid experiments on ``ecosystems.''  In these situations, an observational study is simply the best study allowable. Cause-and-effect statements are arrived at in these situations because observational studies can be conducted with some, though not absolute, control and control can be imparted mathematically into some analyses.\footnote{These analyses are beyond the scope of this book, though.} In addition, a ``preponderance of evidence'' may be arrived at if enough observational studies point to the same conclusion.



\chapter{Getting Started with R} \label{chap:FoundationsR}

\minitoc

\section{Setting Up R and Helpers} \label{sect:RSetup}
Detailed methods for downloading, installing, and configuring R, RStudio, and \R{NCStats} on your personal computer are given on the \href{http://derekogle.com/NCMTH107/resources/}{Resources page of the course website}.

\section{Working With R Basics} \label{sect:RBasics}
\subsection{Saving Results} \label{sect:RSaving}
Results are not saved in R or RStudio.  Rather, ``scripts'' of successful R commands are saved and, then, if the analysis needs to be re-done, the entire set of commands is opened in RStudio and run again.  When writing a report, all tabular and graphical output should be copied from RStudio and pasted into your report document.  This document will serve as your analysis report and can be modified to include answers to questions, references to the tables and graphs, etc.\footnote{Specifics for how to format homework assignments is on the course syllabus}  All data that is not a simple vector (see \sectref{sect:RVectors}) should be entered into R through text files (see \sectref{sect:REnterData}).

R does allow one to save a ``workspace'', though I urge you not to do that.  Rather, save your ``good'' commands in a script and save your ``good'' results in a report document; do not save the workspace.

\warn{Do NOT save the workspace in R.}


\subsection{Expressions and Assignments} \label{sect:RExprAssn}
Expressions in R are mathematical ``equations'' that are evaluated by R with a result seen immediately.  An example of an expression in R is
\begin{knitrout}
\definecolor{shadecolor}{rgb}{0.922, 0.922, 0.922}\color{fgcolor}\begin{kframe}
\begin{verbatim}
> 5+log(7)-pi
[1] 3.804317
\end{verbatim}
\end{kframe}
\end{knitrout}

where \R{log()} and \R{pi} are built-in functions used to compute the natural log and find the value of $\pi$, respectively.  Expressions in R are like using a calculator where the result is shown, but not saved for subsequent analyses.  In addition, expressions in R follow the same order of operations and use of parentheses as expressions entered into your calculator.

\warn{The results of expressions in R are temporary unless the result is assigned to an object.}

Results from an expression are typically saved for further computations by assigning the results of the expression to an object with the assignment operator (i.e., \R{<-}).  The general form for saving the result of an expression into an object is \R{object <- expression}.  The result of the expression will not be seen unless the object name is subsequently typed into R (but see below).  For example, the result of the previous expression is saved into an object called \R{x} and then viewed with
\begin{knitrout}
\definecolor{shadecolor}{rgb}{0.922, 0.922, 0.922}\color{fgcolor}\begin{kframe}
\begin{verbatim}
> x <- 5+log(7)-pi
> x
[1] 3.804317
\end{verbatim}
\end{kframe}
\end{knitrout}
The result of an expression can be both assigned and printed by surrounding the command in parentheses.  For example, the following assigns the result of the expression to \R{y} and prints the result.\footnote{The spaces between the expression and the parentheses are only needed to increase legibility.}
\begin{knitrout}
\definecolor{shadecolor}{rgb}{0.922, 0.922, 0.922}\color{fgcolor}\begin{kframe}
\begin{verbatim}
> ( y <- 15*exp(2) )
[1] 110.8358
\end{verbatim}
\end{kframe}
\end{knitrout}

\warn{The convention of surrounding commands in parentheses to both assign and print the results will be used extensively in this book to save space.}

An object can be named whatever you want, with the exception that it cannot start with a number, contain a space, or be the name of a reserved word or function in R (e.g., \R{pi} or \R{log}).  Furthermore, object names should be short and simple enough that you can remember what is contained in the object.  It is also good practice to view the object immediately after making the assignment to make sure that it contains results that seem appropriate.

\warn{In general, computational results should be assigned to an object.}

\vspace{-12pt}
\warn{Type the name of the object after making the assignment to confirm the results.}


\begin{exsection}
  \item \label{revex:BasicsExpr1} \rhw{} Compute the value of $\frac{3}{7}+\frac{1}{2}$. \ansref{ans:BasicsExpr1}
  \item \label{revex:BasicsExpr2} \rhw{} Compute the value of $\pi*3.7^{2}$. \ansref{ans:BasicsExpr2}
  \item \label{revex:BasicsExpr3} \rhw{} Assign the value of 3.7 to \var{r}. \ansref{ans:BasicsExpr3}
  \item \label{revex:BasicsExpr4} \rhw{} Compute the value of $\pi r^{2}$ using the value of \var{r} assigned in the previous problem. \ansref{ans:BasicsExpr4}
  \item \label{revex:BasicsExpr5} \rhw{} \hspace{18pt} Assign the value 1.2 to \var{r} and then re-evaluate $\pi r^{2}$. \ansref{ans:BasicsExpr5}
\end{exsection}


\subsection{Functions and Arguments}  \label{sect:RFunctions}
R contains many ``programs,'' or functions, to perform particular tasks.  A function is ``called'' by typing the function name followed by open and closed parentheses.  Arguments, which the function will use to perform its task, are contained within the parentheses.  The \R{log()} function, used in the previous section, is an example of a function.  The name of the function is \R{log} and the argument, the number for which to compute the natural log, is contained within the parentheses following the function name.  Many other functions will be described below and in subsequent modules.

\defn{Function}{An R program that performs a particular task.}

\vspace{-12pt}
\defn{Argument}{A ``directive'' that is provided to a function.  Arguments are contained within parentheses that follow the function name.}

\vspace{-12pt}
\warn{Regular curved parentheses have two primary uses in R: (1) to control order of operations in expressions (as with a calculator) and (2) to contain the arguments sent to a function.}


\section{Working With Data}
\subsection{Data Types}  \label{sect:RDataTypes}
Data in R will be designated as an integer (whole numbers), numeric (non-integer numerica values), character (strings), factor (group membershop), or logical (TRUE/FALSE).  The type of data largely dictates the type of analysis that can be performed.  Data types will be discussed in more detail as needed.  Note, however, that the \textbf{factor} data type is a special case of the character data type, where the specific items describe the group to which an individual belongs.  This description allows for certain analyses in later modules.

\defn{Factor}{A special type of variable that identifies the group to which an individual belongs.}


\subsection{Entering Data}  \label{sect:REnterData}
For real data (i.e., several variables from many individuals) it is most efficient to enter data into a comma-separated values (CSV) file and then import that file into R.  Creating a CSV file with Microsoft Excel is described below, though there are other ways to create CSV files (see \href{http://derekogle.com/NCMTH107/resources/FAQ/}{FAQs on class webpage}).  This explanation assumes that you have a basic understanding of Excel (or other spreadsheet softwares).

\warn{Realistic datasets are most efficiently entered into a comma-separated values (CSV) file in preparation for importing into R.}

The spreadsheet should be organized with variables in columns and individuals in rows, with the exception that the first row should contain variable names.  The example spreadsheet below shows the length (cm), weight (kg), and capture location data for a small sample of Black Bears.

\begin{center}
  \includegraphics[width=1.5in]{Figs/Data_File_1.jpg}
\end{center}


\defn{data.frame}{A two-dimensional organization of variables (as columns, possibly of different data types) recorded on multiple individuals (as rows).}

\vspace{-12pt}
\warn{The columns of a data.frame correspond to variables and the rows of a data.frame correspond to individuals.}

Variable names must NOT contain spaces.  For example, don't use \var{total length} or \var{length (cm)}.  If you feel the need to have longer variable names, then separate the parts with a period (e.g., \var{length.cm}) or an underscore (e.g., \var{length\_cm}).  Furthermore, numerical measurements should NOT include units (e.g., don't use \verb"7 cm").  Finally, for categorical data, make sure that all categories are consistent (e.g., do not have a column that contains both \verb"bayfield" and \verb"Bayfield").

\warn{Variable names and data should not contain spaces.  An "\R{Error in scan}" message usually indicates spaces in the variable names or data.}

The spreadsheet is saved as a CSV file by selecting the \verb"File..Save As" menu item, which will produce the dialog box below. In this dialog box, change \verb"Save as type" to \verb"CSV (Comma delimited) (*.csv)" (you may have to scroll down), provide a file name (don't have any periods in the name besides for ``.csv'', which you should not have to type), select a location to save the file (don't forget this location!!), and press \verb"Save".  Two ``warning'' dialog boxes may then appear -- select \verb"OK" for the first and \verb"YES" for the second.  You can now close the spreadsheet file (you may be asked to save changes -- you should say \verb"No").
\begin{center}
  \includegraphics[width=3.5in]{Figs/Data_File_2.jpg}
\end{center}

The following steps are used to load the data in the CSV file into RStudio.

\begin{Itemize}
  \item Open RStudio.
  \item Open a new script by selecting the \verb"File", \verb"New File", \verb"R Script" menu items.
  \item Type \R{library(NCStats)} in the new script.
  \item Save this script by selecting the \verb"File", \verb"Save" menu items.  In the ensuing dialog box, navigate to the \textbf{exact same directory} where you saved the data, type a name for the file in the \verb"File name:" box (\textbf{do not use a period in this name!!}), and press \verb"Save".

\begin{center}
  \includegraphics[width=3.5in]{Figs/Data_File_3.jpg}
\end{center}

  \item Set the working directory (tell R where the file is) with the \verb"Session", \verb"Set Working Directory ...", \verb"To Source File Location" menu items in RStudio.  RStudio will print an appropriate \R{setwd()} command to the console.  Copy this command from the console to the second line in your script.\footnote{Doing this will eliminate the need to manually select the menu options every time you want to run this script.}  For example, I stored the file created above in the \verb"C:/data" directory, so that RStudio will create this \R{setwd("C:/data")}.
  \item The CSV file is read into R by including the name of the file (in quotes) in \R{read.csv()}.  For example, \R{"Bears.csv"} is read into R and stored into an object called \R{bears} with \R{bears <- read.csv("Bears.csv")}.


  \item One should check the data in this object as descried in \sectref{sect:RViewdf} below
\end{Itemize}

\warn{Data stored in an external CSV file is read into R with \R{read.csv()}.}

It is important that each row of the data.frame correspond to one individual.  This is critically important when data are recorded for two different groups (e.g., for a two-sample t-test; see \modref{chap:tTest2}).  For example, the following data are methyl mercury levels recorded in mussels from two locations labeled as ``impacted'' and ``reference.''
\begin{Verbatim}
  impacted   0.011  0.054  0.056  0.095  0.051  0.077
  reference  0.031  0.040  0.029  0.066  0.018  0.042  0.044
\end{Verbatim}
To follow the ``one individual per row'' rule, these data are entered in stacked format where the ``reference'' data are stacked underneath the ``impacted'' data and a column is used to indicate to which group the individuals belong.  For example, the Excel file for data entry would look like the following

\begin{center}
  \includegraphics[width=1in]{Figs/StackedData.jpg}
\end{center}

\warn{Data files are constructed with data from only one individual in each row.}

\subsubsection*{Alternative Forms of Getting Data} \label{sect:RAltData}
Some of the data files that you will use are provided on the \href{http://derekogle.com/NCMTH107/resources/data_107}{Data for MTH107} resource page of the class webpage.  In these cases, the data should be downloaded from the webpage and saved in the same directory or folder as your analysis script.  The downloaded file is then read into R in the same manner as described previously (i.e., set the working directory with \R{setwd()} and use \R{read.csv()}).

A few data files used in this book are supplied with R or the NCStats package.  These files are loaded with \R{data()}.  For example, the \dfile{iris} data file is loaded into R with
\begin{knitrout}
\definecolor{shadecolor}{rgb}{0.922, 0.922, 0.922}\color{fgcolor}\begin{kframe}
\begin{verbatim}
> data(iris)
\end{verbatim}
\end{kframe}
\end{knitrout}


\subsection{Working With Data Frames}  \label{sect:RWorkdf}
\subsubsection{Viewing a Data Frame}  \label{sect:RViewdf}
Many users are disoriented in RStudio because they cannot ``see'' their data in the same way that they see it in a spreadsheet program.  There are, however, several options for viewing your data.  First, you can type the name of the data.frame object to see its entire contents.
\begin{knitrout}
\definecolor{shadecolor}{rgb}{0.922, 0.922, 0.922}\color{fgcolor}\begin{kframe}
\begin{verbatim}
> bears
   length.cm weight.kg      loc
1      139.0       110 Bayfield
2      138.0        60 Bayfield
3      139.0        90 Bayfield
4      120.5        60 Bayfield
5      149.0        85 Bayfield
6      141.0       100  Ashland
7      141.0        95  Ashland
8      150.0        85  Douglas
9      166.0       155  Douglas
10     151.5       140  Douglas
11     129.5       105  Douglas
12     150.0       110  Douglas
\end{verbatim}
\end{kframe}
\end{knitrout}

Typing the name is adequate for small data.frames, but not useful for large data.frames.  The entire data.frame is opened in a separate window by double-clicking on the name of the data.frame in the \R{Environment} tab of RStudio.  Alternatively, the first and last three rows are viewed by including the data.frame object in \R{headtail()}.
\begin{knitrout}
\definecolor{shadecolor}{rgb}{0.922, 0.922, 0.922}\color{fgcolor}\begin{kframe}
\begin{verbatim}
> headtail(bears)
   length.cm weight.kg      loc
1      139.0       110 Bayfield
2      138.0        60 Bayfield
3      139.0        90 Bayfield
10     151.5       140  Douglas
11     129.5       105  Douglas
12     150.0       110  Douglas
\end{verbatim}
\end{kframe}
\end{knitrout}

In addition to viewing the contents, it is useful to examine the structure of the data.frame as returned from \R{str()}.  In this example, it is seen that three variables were recorded on 12 individuals.  The first variables -- \var{length.cm} and \var{weight.kg} -- are numerical measurements made on the bears.  The last variable -- \var{loc} -- is a factor variable that records the capture location for each bear.
\begin{knitrout}
\definecolor{shadecolor}{rgb}{0.922, 0.922, 0.922}\color{fgcolor}\begin{kframe}
\begin{verbatim}
> str(bears)
'data.frame':	12 obs. of  3 variables:
 $ length.cm: num  139 138 139 120 149 ...
 $ weight.kg: int  110 60 90 60 85 100 95 85 155 140 ...
 $ loc      : Factor w/ 3 levels "Ashland","Bayfield",..: 2 2 2 2 2 1 1 3 3 3 ...
\end{verbatim}
\end{kframe}
\end{knitrout}
The levels of the \var{loc} variable may be seen by including this variable (with the data.frame name) as the argument to \R{levels()}.
\begin{knitrout}
\definecolor{shadecolor}{rgb}{0.922, 0.922, 0.922}\color{fgcolor}\begin{kframe}
\begin{verbatim}
> levels(bears$loc)
[1] "Ashland"  "Bayfield" "Douglas" 
\end{verbatim}
\end{kframe}
\end{knitrout}

In the previous example, the \R{\$} notation was used to identify a particular variable (i.e., \R{loc}) within a data.frame (\R{bears}).  Think of variables as being nested inside data.frames and, thus, to access the variable you must first identify the data.frame in which it exists and then the name of the variable.  The \R{\$} simply separates the data.frame from the variable.
\begin{knitrout}
\definecolor{shadecolor}{rgb}{0.922, 0.922, 0.922}\color{fgcolor}\begin{kframe}
\begin{verbatim}
> bears$length.cm
 [1] 139.0 138.0 139.0 120.5 149.0 141.0 141.0 150.0 166.0 151.5 129.5 150.0
> bears$loc
 [1] Bayfield Bayfield Bayfield Bayfield Bayfield Ashland  Ashland  Douglas  Douglas 
[10] Douglas  Douglas  Douglas 
Levels: Ashland Bayfield Douglas
\end{verbatim}
\end{kframe}
\end{knitrout}

\warn{A dollar sign is ONLY used in R to separate the name of a data.frame from the name of a variable within that data.frame.}

\subsubsection{Selecting Individuals}  \label{sect:RSelectIndivs}
In some instances, it may be important to select or exclude an individual from a data.frame.  Data.frames are two-dimensional objects that are indexed by a row and a column, in that order.  Positions within a data.frame are selected within paired square brackets.  For example, the item in the third row and second column of \R{bears} is selected below.
\begin{knitrout}
\definecolor{shadecolor}{rgb}{0.922, 0.922, 0.922}\color{fgcolor}\begin{kframe}
\begin{verbatim}
> bears[3,2]
[1] 90
\end{verbatim}
\end{kframe}
\end{knitrout}

\warn{Identifying the position of an item in an object is the ONLY time that square brackets are used in R.}

An entire row or column may be selected by omitting the other dimension.  For example, one could select the entire second column with \R{bears[,2]}, but this is also the \R{weight.kg} variable and is better selected, as shown above, with \R{bears\$weight.kg}.  As a better example, the entire third row is selected below (note that the column designation was omitted).
\begin{knitrout}
\definecolor{shadecolor}{rgb}{0.922, 0.922, 0.922}\color{fgcolor}\begin{kframe}
\begin{verbatim}
> bears[3,]
  length.cm weight.kg      loc
3       139        90 Bayfield
\end{verbatim}
\end{kframe}
\end{knitrout}

Multiple rows are selected by combining row indices together with \R{c()} (more about \R{c()} in \sectref{sect:RVectors}).  For example, the third, fifth, and eighth rows are selected below (again, the column index is omitted).
\begin{knitrout}
\definecolor{shadecolor}{rgb}{0.922, 0.922, 0.922}\color{fgcolor}\begin{kframe}
\begin{verbatim}
> bears[c(3,5,8),]
  length.cm weight.kg      loc
3       139        90 Bayfield
5       149        85 Bayfield
8       150        85  Douglas
\end{verbatim}
\end{kframe}
\end{knitrout}

Finally, rows can be excluded by preceding the row indices with a negative sign.
\begin{knitrout}
\definecolor{shadecolor}{rgb}{0.922, 0.922, 0.922}\color{fgcolor}\begin{kframe}
\begin{verbatim}
> bears[-c(3,5,8,10,12),]
   length.cm weight.kg      loc
1      139.0       110 Bayfield
2      138.0        60 Bayfield
4      120.5        60 Bayfield
6      141.0       100  Ashland
7      141.0        95  Ashland
9      166.0       155  Douglas
11     129.5       105  Douglas
\end{verbatim}
\end{kframe}
\end{knitrout}


\subsubsection{Filtering a data.frame}  \label{sect:RSubsetdf}
It is common to create a new data.frame that contains only some of the individuals from an existing data.frame.  For example, a researcher may want a data.frame of only bears captured in Bayfield County or bears that weighed more than 100 kg.  The process of creating the newer, smaller data.frame is called filtering (or subsetting) and is accomplished with \R{filterD()}.  The \R{filterD()} function requires the original data.frame as the first argument and a condition statement as the second argument.  The condition statement is used to either include or exclude individuals from the original data.frame.  Condition statements consist of the name of a variable in the original data.frame, a comparison operator, and a comparison value \tabrefp{tab:RSubsetConditions}.  The result from \R{filterD()} should be assigned to an object, which is then the name of the new data.frame.

\warn{\R{filterD()} is used to create a new data.frame that consists of individuals selected by some criterion from an existing data.frame.}

\begin{table}[htbp]
  \caption{Condition operators used in \R{filterD()} and their results.  Note that \emph{variable} generically represents a variable in the original data.frame and \emph{value} is a generic value or level.  Both \emph{variable} and \emph{value} would be replaced with specific items.}  \label{tab:RSubsetConditions}
  \centering
\begin{tabular}{cc}
\hline\hline
Condition Operator &  Individuals Returned from Original Data Frame \\
\hline
\widen{-1}{6}{\emph{variable}} $==$ \emph{value} & all individual that are \textbf{equal} to the given value \\
\widen{-1}{5}{\emph{variable}} $!=$ \emph{value} & all individuals that are \textbf{NOT equal} to the given value \\
\widen{-1}{5}{\emph{variable}} $>$ \emph{value} & all individuals that are \textbf{greater than} the given value \\
\widen{-1}{5}{\emph{variable}} $>=$ \emph{value} & all individuals that are \textbf{greater than or equal} to the given value \\
\widen{-1}{5}{\emph{variable}} $<$ \emph{value} & all individuals that are \textbf{less than} the given value \\
\widen{-1}{5}{\emph{variable}} $<=$ \emph{value} & all individuals that are \textbf{less than or equal} to the given value \\
\widen{-1}{5}{\emph{condition}}, \emph{condition} & all individuals that \textbf{meet both conditions} \\
\widen{-2}{6}{\emph{condition}} $|$ \emph{condition} & all individuals that \textbf{meet one or both conditions}\footnote{Note that this ``or'' operator is a ``vertical line'' which is typed with the shift-backslash key.} \\
\hline\hline
\end{tabular}
\end{table}

\vspace{18pt}
The following are examples of new data.frames created from \var{bears}.  The name of the new data.frame (i.e., object to the left of the assignment operator) can be any valid object name.  As demonstrated below, the new data.frame (or its structure) should be examined after each filtering to ensure that the data.frame actually contains the items that you desire.

\begin{itemize}
  \item Only individuals from \emph{Bayfield} county.
\begin{knitrout}
\definecolor{shadecolor}{rgb}{0.922, 0.922, 0.922}\color{fgcolor}\begin{kframe}
\begin{verbatim}
> bf <- filterD(bears,loc=="Bayfield")
> bf
  length.cm weight.kg      loc
1     139.0       110 Bayfield
2     138.0        60 Bayfield
3     139.0        90 Bayfield
4     120.5        60 Bayfield
5     149.0        85 Bayfield
\end{verbatim}
\end{kframe}
\end{knitrout}

  \item Individuals from both \emph{Bayfield} and \emph{Ashland} counties.
\begin{knitrout}
\definecolor{shadecolor}{rgb}{0.922, 0.922, 0.922}\color{fgcolor}\begin{kframe}
\begin{verbatim}
> bfash <- filterD(bears,loc %in% c("Bayfield","Ashland"))
> bfash
  length.cm weight.kg      loc
1     139.0       110 Bayfield
2     138.0        60 Bayfield
3     139.0        90 Bayfield
4     120.5        60 Bayfield
5     149.0        85 Bayfield
6     141.0       100  Ashland
7     141.0        95  Ashland
\end{verbatim}
\end{kframe}
\end{knitrout}

  \item Individuals with a weight greater than 100 kg.
\begin{knitrout}
\definecolor{shadecolor}{rgb}{0.922, 0.922, 0.922}\color{fgcolor}\begin{kframe}
\begin{verbatim}
> gt100 <- filterD(bears,weight.kg>100)
> gt100
  length.cm weight.kg      loc
1     139.0       110 Bayfield
2     166.0       155  Douglas
3     151.5       140  Douglas
4     129.5       105  Douglas
5     150.0       110  Douglas
\end{verbatim}
\end{kframe}
\end{knitrout}

  \item Individuals from \emph{Douglas} County that weigh at least 150 kg.
\begin{knitrout}
\definecolor{shadecolor}{rgb}{0.922, 0.922, 0.922}\color{fgcolor}\begin{kframe}
\begin{verbatim}
> do150 <- filterD(bears,loc=="Douglas",weight.kg>=150)
> do150
  length.cm weight.kg     loc
1       166       155 Douglas
\end{verbatim}
\end{kframe}
\end{knitrout}
\end{itemize}

\warn{View or ``structure'' the data.frame from \R{filterD()} to be sure that it contains data.}

\begin{exsection}
  \item \label{revex:BasicsDataFrame1} \rhw{} Two students at Seattle Community College made biometric measurements on 25 Douglas fir (\emph{Pseudotsuga menziesii}) trees in the lowlands of western Washington.  The variables recorded in the \href{https://raw.githubusercontent.com/droglenc/NCData/master/DougFirBiometrics.csv}{DougFirBiometrics.csv} file are a unique tree identifier (\var{tree}), the observer's name (\var{observer}; either ``Ingrid'' or ``Dylan''), the circumference at breast height (meters; \var{circ}), the height to the eye of the observer (meters; \var{eyeht}), the horizontal distance between observer and tree (meters; \var{horizdist}), the angle between observer and top of tree (degrees; \var{angle}), and the estimated height of tree (meters; \var{height}) using right-angle trigonometry. \ansref{ans:BasicsDataFrame1}
  \begin{Enumerate}
    \item Read this data file into an object called \R{DF}.
    \item Examine the structure of this data.frame.
    \item Show all measurements made on the third tree. [Do not do this manually; use R code.]
    \item Show all estimated tree heights.
    \item Show the estimated tree height for the fifth tree.
    \item Show all measurements for all trees measured by ``Ingrid''.  [HINT: use filtering.]
    \item Show all estimated tree heights for all trees measured by ``Dylan''.  [HINT: use filtering.]
    \item Show all measurements for tree heights less than 10 m.  [HINT: use filtering.]
    \item Show all measurements for tree heights greater than 10 m and circumference less than 1 m.  [HINT: use filtering.]
  \end{Enumerate}
\end{exsection}


\subsection{Vectors}  \label{sect:RVectors}
Data.frames are the primary structure in which to store real data.  However, much simpler situations that don't require a data.frame may arise.  In R, items of the same data type \sectrefp{sect:RDataTypes} are stored in a one-dimensional ``list'' called a \emph{vector}.  Vectors are usually displayed in one row (with many columns), but they may also be thought of as a single column (with many rows).  Items are entered into a vector with \R{c()}, where the individual arguments are specific numbers, characters, or logical values.\footnote{Note that \R{c} comes from the word ``concatenate.''}  Items for a vector of characters must be contained within paired quotes.
\begin{knitrout}
\definecolor{shadecolor}{rgb}{0.922, 0.922, 0.922}\color{fgcolor}\begin{kframe}
\begin{verbatim}
> ( v <- c(1,2,5) )
[1] 1 2 5
> ( y <- c("Iowa","Minnesota","Wisconsin") )
[1] "Iowa"      "Minnesota" "Wisconsin"
\end{verbatim}
\end{kframe}
\end{knitrout}

\defn{Vector}{A one-dimensional list of items of the same data type.  The primary information storage unit in R.}

Single variables from a data.frame are vectors.  Vectors that are not extracted from a data.frame will only be used in this course for very simple lists of items, usually as arguments in a function.
\begin{knitrout}
\definecolor{shadecolor}{rgb}{0.922, 0.922, 0.922}\color{fgcolor}\begin{kframe}
\begin{verbatim}
> bears$length.cm
 [1] 139.0 138.0 139.0 120.5 149.0 141.0 141.0 150.0 166.0 151.5 129.5 150.0
\end{verbatim}
\end{kframe}
\end{knitrout}

\warn{The columns of a data.frame are accessed with the name of the data.frame, a dollar sign, and then the name of the variable -- i.e., generically, \R{dataframe\$varname}.}


\begin{exsection}
  \item \label{revex:BasicsData1}\rhw{} Create a vector called \var{h} that contains nine heights of people. \ansref{ans:BasicsData1}
  \item \label{revex:BasicsData2}\rhw{} Create a vector called \var{w} that contains nine weights of people. \ansref{ans:BasicsData2}
  \item \label{revex:BasicsData3}\rhw{} Create a vector called \var{hc} that contains nine hair colors of people. \ansref{ans:BasicsData3}
  \item \label{revex:BasicsData4}\rhw{} Create a vector called \var{m} that contains nine logical values (=\R{TRUE} if male). \ansref{ans:BasicsData4}
  \item \label{revex:BasicsData5}\rhw{} Using the vectors from the previous questions,  \ansref{ans:BasicsData5}
  \begin{Enumerate}
    \item ... create the largest possible data.frame (use \R{data.frame()}).
    \item ... identify the height of the third individual of this data.frame.
    \item ... identify the hair color for the sixth individual of this data.frame.
  \end{Enumerate}
\end{exsection}



\chapter[Summary One Quant Var]{Summaries for One Quantitative Variable)} \label{chap:UnivEDAQuant1}

\vspace{-48pt}
\minitoc
\vspace{12pt}

\lettrine{S}{ummarizing large quantities of data with} few graphical or numerical summaries makes it is easier to identify meaning from data (discussed in \modref{chap:WhyStatsImportant}). Numeric and graphical summaries specific to a single quantitative variable are described in this module. Interpretations from these numeric and graphical summaries are described in the next module.

Two data sets will be considered in this module when making calculations ``by hand'' (i.e., without using R). The first data set consists of the number of open pit mines in countries that have open pit mines \tabrefp{tab:MineData}.\footnote{These data were collected from \href{https://en.wikipedia.org/wiki/List_of_open-pit_mines}{this page}. See \sectref{sect:REnterData} for how to enter these data into R.} The second data set is Richter scale recordings for 15 major earthquakes \tabrefp{tab:EQData}. A third data set -- number of days of ice cover at ice gauge station 9004 in Lake Superior -- will be used to demonstrate calculations with R. These data are in \href{https://raw.githubusercontent.com/droglenc/NCData/master/LakeSuperiorIce.csv}{LakeSuperiorIce.csv} and are loaded into \R{LSI} below.\footnote{See \sectref{sect:RAltData} for how to access these data. These data are originally from the \href{http://www.nsidc.org/}{National Snow and Ice Data Center}.}
\begin{knitrout}
\definecolor{shadecolor}{rgb}{0.922, 0.922, 0.922}\color{fgcolor}\begin{kframe}
\begin{verbatim}
> LSI <- read.csv("data/LakeSuperiorIce.csv")
\end{verbatim}
\end{kframe}
\end{knitrout}


% latex table generated in R 3.4.0 by xtable 1.8-2 package
% Wed Jun 14 09:31:37 2017
\begin{table}[ht]
\centering
\caption{Number of open pit mines in countries that have open pit mines.} 
\label{tab:MineData}
\begin{tabular}{rrrrrrrrrrrrr}
   \hline
2.0 & 11.0 & 4.0 & 1.0 & 15.0 & 12.0 & 1.0 & 1.0 & 3.0 & 2.0 & 2.0 & 1.0 & 1.0 \\ 
  1.0 & 1.0 & 2.0 & 4.0 & 1.0 & 4.0 & 2.0 & 4.0 & 2.0 & 1.0 & 4.0 & 11.0 & 1.0 \\ 
   \hline
\end{tabular}
\end{table}


% latex table generated in R 3.4.0 by xtable 1.8-2 package
% Wed Jun 14 09:31:37 2017
\begin{table}[ht]
\centering
\caption{Richter scale recordings for 15 major earthquakes.} 
\label{tab:EQData}
\begin{tabular}{rrrrrrrrrrrrrrr}
   \hline
5.5 & 6.3 & 6.5 & 6.5 & 6.8 & 6.8 & 6.9 & 7.1 & 7.3 & 7.3 & 7.7 & 7.7 & 7.7 & 7.8 & 8.1 \\ 
   \hline
\end{tabular}
\end{table}



\section{Numerical Summaries} \label{sec:quEDACenter}
A ``typical'' value and the ``variability'' of a quantitative variable are often described from numerical summaries. Calculation of these summaries is described in this module, whereas their interpretation is described in \modref{chap:UnivEDAQuant1}. As you will see in \modref{chap:UnivEDAQuant1}, ``typical'' values are measures of \textbf{center} and ``variability'' is often described as \textbf{dispersion} (or spread). Three measures of center are the median, mean, and mode. Three measures of dispersion are the inter-quartile range, standard deviation, and range.

All measures computed in this module are summary statistics -- i.e., they are computed from individuals in a sample. Thus, the name of each measure should be preceded by ``sample'' -- e.g., sample median, sample mean, and sample standard deviation. These measures could be computed from every individual, if the population was known. These values would then be parameters and would be preceded by ``population'' -- e.g., population median, population mean, and population standard deviation.\footnote{See \modref{sect:IVPPSS} for clarification on the differences between populations and samples and parameters and statistics.}


\subsection{Median} \label{sec:Median}
The median is the value of the individual in the position that splits the \textbf{ordered} list of individuals into two equal-\textbf{sized} halves. In other words, if the data are ordered, half the values will be smaller than the median and half will be larger.

The process for finding the median consists of three steps,\footnote{Most computer programs use a more sophisticated algorithm for computing the median and, thus, will produce different results than what will result from applying these steps.}
\vspace{-8pt}
\begin{Enumerate}
  \item Order the data from smallest to largest.
  \item Find the ``middle \textbf{position}'' ($mp$) with $mp=\frac{n+1}{2}$.
  \item If $mp$ is an integer (i.e., no decimal), then the median is the value of the individual in that position. If $mp$ is not an integer, then the median is the average of the value immediately below and the value immediately above the $mp$.
\end{Enumerate}

As an example, the open pit data from \tabref{tab:MineData} are,

% latex table generated in R 3.4.0 by xtable 1.8-2 package
% Wed Jun 14 09:31:37 2017
\begin{tabular}{rrrrrrrrrrrrr}
  1 & 1 & 1 & 1 & 1 & 1 & 1 & 1 & 1 & 1 & 2 & 2 & 2 \\ 
  2 & 2 & 2 & 3 & 4 & 4 & 4 & 4 & 4 & 11 & 11 & 12 & 15 \\ 
  \end{tabular}


Because $n=26$, the $mp=\frac{26+1}{2}=13.5$. The $mp$ is not an integer so the median is the average of the values in the 13th and 14th ordered positions (i.e., the two positions closest to $mp$). Thus, the median number of open pit mines in this sample of countries is $\frac{2+2}{2}=2$.

Consider finding the median of the Richter Scale magnitude recorded for fifteen major earthquakes as another example (ordered data are in \tabref{tab:EQData}). Because $n=15$, the $mp=\frac{15+1}{2}=8$. The $mp$ is an integer so the median is the value of the individual in the 8th ordered position, which is 7.1.

\warn{Don't forget to order the data when computing the median.}


\subsection{Inter-Quartile Range}
Quartiles are the values for the three individuals that divide ordered data into four (approximately) equal parts. Finding the three quartiles consists of finding the median, splitting the data into two equal parts at the median, and then finding the medians of the two halves.\footnote{You should review how a median is computed before proceeding with this section.}  A concern in this process is that the median is NOT part of either half if there is an odd number of individuals. These steps are summarized as,
\begin{Enumerate}
  \item Order the data from smallest to largest.
  \item Find the median -- this is the second quartile (Q2).
  \item Split the data into two halves at the median. If $n$ is odd (so that the median is one of the observed values), then the median is not part of either half.\footnote{Some authors put the median into both halves when $n$ is odd. The difference between the two methods is minimal for large $n$.}
  \item Find the median of the lower half of data -- this is the 1st quartile (Q1).
  \item Find the median of the upper half of data -- this is the third quartile (Q3).
\end{Enumerate}

These calculations are illustrated with the open pit mine data (the median was computed in \sectref{sec:Median}). Because $n=26$ is even, the halves of the data split naturally into two halves each with 13 individuals. Therefore, the $mp=\frac{13+1}{2}=7$ and the median of each half is the value of the individual in the seventh position. Thus, $Q1=1$ and $Q3=4$.

% latex table generated in R 3.4.0 by xtable 1.8-2 package
% Wed Jun 14 09:31:37 2017
\begin{tabular}{rrrrrrr}
  1 & 1 & 1 & 1 & 1 & 1 & 1 \\ 
  1 & 1 & 1 & 2 & 2 & 2 &  \\ 
  2 & 2 & 2 & 3 & 4 & 4 & 4 \\ 
  4 & 4 & 11 & 11 & 12 & 15 &  \\ 
  \end{tabular}


In summary, the first, second, and third quartiles for the open pit mine data are 1, 2, and 4, respectively. These three values separate the ordered individuals into approximately four equally-sized groups -- those with values less than (or equal to) 1, with values between (inclusive) 1 and 2, with values between (inclusive) 2 and 4, and with values greater (or equal to) than 4.

As another example, consider finding the quartiles for the earthquake data \tabrefp{tab:EQData}. Recall from above \sectrefp{sec:Median} that the median (=7.1) is in the eighth position of the ordered data. The value in the eighth position will not be included in either half. Thus, the two halves of the data are 5.5, 6.3, 6.5, 6.5, 6.8, 6.8, 6.9 and 7.3, 7.3, 7.7, 7.7, 7.7, 7.8, 8.1. The middle position for each half is then $mp=\frac{7+1}{2}=4$. Thus, the median for each half is the individual in the fourth position. Therefore, the median of the first half is $Q1=6.5$ and the median of the second half is $Q3=7.7$.

The interquartile range (IQR) is the difference between $Q3$ and $Q1$, namely $Q3-Q1$. However, the IQR (as strictly defined) suffers from a lack of information. For example, what does an IQR of 9 mean?  It can have a completely different interpretation if the IQR is from values of 1 to 10 or if it is from values of 1000 to 1009. Thus, the IQR is more useful if presented as both $Q3$ and $Q1$, rather than as the difference. Thus, for example, the IQR for the open pit mine data is from a $Q3$ of 4 to a $Q1$ of 1 and the IQR for the earthquake data is from a $Q3$ of 7 to a $Q1$ of 6.5.

\warn{The IQR can be thought of as the ``range of the middle half of the data.''}

\vspace{-12pt}
\warn{When reporting the IQR, explicitly state both $Q3$ and $Q1$ (i.e., do not subtract them).}


\subsection{Mean}
The mean is the arithmetic average of the data. The sample mean is denoted by $\bar{x}$ and the population mean by $\mu$. The mean is simply computed by adding up all of the values and dividing by the number of individuals. If the measurement of the generic variable $x$ on the $i$th individual is denoted as $x_{i}$, then the sample mean is computed with these two steps,
\begin{Enumerate}
  \item Sum (i.e., add together) all of the values -- $\Sum_{i=1}^{n}x_{i}$.
  \item Divide by the number of individuals in the sample -- $n$.
\end{Enumerate}
or more succinctly summarized with this equation,

\begin{equation} \label{eqn:SampleMean}
     \bar{x} = \frac{\Sum_{i=1}^{n}x_{i}}{n}
\end{equation}

For example, the sample mean of the open pit mine data is computed as follows:

\[ \bar{x} = \frac{2+11+4+1+15+ ... +2+1+4+11+1}{26} = \frac{94}{26} = 3.6  \]

Note in this example with a discrete variable that it is possible (and reasonable) to present the mean with a decimal. For example, it is not possible for a country to have 3.6 open pit mines, but it IS possible for the mean of a sample of countries to be 3.6 open pit mines.

\warn{As a general rule-of-thumb, present the mean with one more decimal than the number of decimals it was recorded in.}



\subsection{Standard Deviation}\label{sect:StdDev}
The sample standard deviation, denoted by $s$, is computed with these six steps:
\begin{Enumerate}
  \item Compute the sample mean (i.e., $\bar{x}$).
  \item For each value ($x_{i}$), find the difference between the value and the mean (i.e., $x_{i}-\bar{x}$).
  \item Square each difference (i.e., $(x_{i}-\bar{x})^{2}$).
  \item Add together all the squared differences.
  \item Divide this sum by $n-1$. [\textit{Stopping here gives the sample variance, $s^{2}$.}]
  \item Square root the result from the previous step to get $s$.
\end{Enumerate}

These steps are neatly summarized with
\vspace{-6pt}
\begin{equation}
  \label{eqn:SampleSD}
     s = \sqrt{\frac{\Sum_{i=1}^{n}(x_{i}-\bar{x})^{2}}{n-1}}
\end{equation}

The calculation of the standard deviation of the earthquake data \tabrefp{tab:EQData} is facilitated with the calculations shown in \tabref{tab:SDCalc}. In \tabref{tab:SDCalc}, note that
\vspace{-6pt}
\begin{Itemize}
  \item $\bar{x}$ is the sum of the ``Value'' column divided by $n=15$ (i.e., $\bar{x}=7.07$).
  \item The ``Diff'' column is each observed value minus $\bar{x}$ (i.e., Step 2).
  \item The ``Diff$^2$'' column is the square of the differences (i.e., Step 3).
  \item The sum of the ``Diff$^2$'' column is Step 4.
  \item The sample variance (i.e., Step 5) is equal to this sum divided by $n-1=14$ or $\frac{6.773}{14}=0.484$.
  \item The sample standard deviation is the square root of the sample variance or $s=\sqrt{0.484}=0.696$.
\end{Itemize}

\begin{table}[htbp]
  \caption{Table showing an efficient calculation of the standard deviation of the earthquake data.}
  \label{tab:SDCalc}
    \centering
    \begin{tabular}{cccc}
\hline\hline
Indiv & Value & Diff & Diff$^2$ \\
i & $x_{i}$ & $x_{i}-\bar{x}$ & $(x_{i}-\bar{x})^{2}$ \\
\hline
1 & 5.5 & -1.57 & 2.454 \\
2 & 6.3 & -0.77 & 0.588 \\
3 & 6.5 & -0.57 & 0.321 \\
4 & 6.5 & -0.57 & 0.321 \\
5 & 6.8 & -0.27 & 0.071 \\
6 & 6.8 & -0.27 & 0.071 \\
7 & 6.9 & -0.17 & 0.028 \\
8 & 7.1 & 0.03 & 0.001 \\
9 & 7.3 & 0.23 & 0.054 \\
10 & 7.3 & 0.23 & 0.054 \\
11 & 7.7 & 0.63 & 0.401 \\
12 & 7.7 & 0.63 & 0.401 \\
13 & 7.7 & 0.63 & 0.401 \\
14 & 7,8 & 0.73 & 0.538 \\
15 & 8.1 & 1.03 & 1.068 \\
\hline
Sum & 106 & 0 & 6.773 \\
\hline\hline
    \end{tabular}
\end{table}

From this, on average, each earthquake is approximately 0.7 Richter Scale units different than the average earthquake in these data.

\warn{In the standard deviation calculations don't forget to take the square root of the variance.}

\vspace{-12pt}
\warn{The standard deviation is greater than or equal to zero.}

The standard deviation can be thought of as ``the average difference between the values and the mean.'' This is, however, not a strict definition because the formula for the standard deviation does not simply add the differences and divide by $n$ as this definition would imply. Notice in \tabref{tab:SDCalc} that the sum of the differences from the mean is 0. This will be the case for all standard deviation calculations using the correct mean, because the mean balances the distance to individuals below the mean with the distance of individuals above the mean (see \sectref{sect:MeanMedian} in the next module). Thus, the mean difference will always be zero. This ``problem'' is corrected by squaring the differences before summing them. To get back to the original units, the squaring is later ``reversed'' by the square root. So, more accurately, the standard deviation is the square root of the average squared differences between the values and the mean. Therefore, ``the average difference between the values and the mean'' works as a practical definition of the meaning of the standard deviation, but it is not strictly correct.

\warn{Use the fact that the sum of all differences from the mean equals zero as a check of your standard deviation calculation.}

Further note that the mean is the value that minimizes the value of the standard deviation calculation -- i.e., putting any other value besides the mean into the standard deviation equation will result in a larger value.

Finally, you may be wondering why the sum of the squared differences in the standard deviation calculation is divided by $n-1$, rather than $n$. Recall (from \sectref{sect:IVPPSS}) that statistics are meant to estimate parameters. The sample standard deviation is supposed to estimate the population standard deviation ($\sigma$). Theorists have shown that if we divide by $n$, $s$ will consistently underestimate $\sigma$. Thus, $s$ calculated in this way would be a biased estimator of $\sigma$. Theorists have found, though, that dividing by $n-1$ will cause $s$ to be an unbiased estimator of $\sigma$. Being unbiased is generally good -- it means that on average our statistic estimates our parameter (this concept is discussed in more detail in \modref{chap:SamplingDist}).


\subsection{Mode}
The mode is the value that occurs most often in a data set. For example, one open pit mine is the mode in the open pit mine data \tabrefp{tab:MCmode}.

% latex table generated in R 3.4.0 by xtable 1.8-2 package
% Wed Jun 14 09:31:37 2017
\begin{table}[ht]
\centering
\caption{Frequency of countries by each number of open pit mines.} 
\label{tab:MCmode}
\begin{tabular}{rrrrrrrr}
   \hline
Number of Mines & 1 & 2 & 3 & 4 & 11 & 12 & 15 \\ 
  Freq of Countries & 10 & 6 & 1 & 5 & 2 & 1 & 1 \\ 
   \hline
\end{tabular}
\end{table}


The mode for a continuous variable is the class or bin with the highest frequency of individuals. For example, if 0.5-unit class widths are used in the Richter scale data, then the modal class is 6.5-6.9 \tabrefp{tab:EQmode}.

% latex table generated in R 3.4.0 by xtable 1.8-2 package
% Wed Jun 14 09:31:37 2017
\begin{table}[ht]
\centering
\caption{Frequency of earthquakes by Richter Scale class.} 
\label{tab:EQmode}
\begin{tabular}{rllllll}
   \hline
Richter Scale Class & 5.5-5.9 & 6-6.4 & 6.5-6.9 & 7-7.4 & 7.5-7.9 & 8-8.4 \\ 
  Freq of Earthquakes & 1 & 1 & 5 & 3 & 4 & 1 \\ 
   \hline
\end{tabular}
\end{table}


Some data sets may have two values or classes with the maximum frequency. In these situations the variable is said to be \textbf{bimodal}.

\subsection{Range}
The range is the difference between the maximum and minimum values in the data and measures the ultimate dispersion or spread of the data. The range in the open pit mine data is 15-1 = 14.

The range should \textbf{never be used by itself} as a measure of dispersion. The range is extremely sensitive to outliers and is best used only to show all possible values present in the data. The range (as strictly defined) also suffers from a lack of information. For example, what does a range of 9 mean?  It can have a completely different interpretation if it came from values of 1 to 10 or if it came from values of 1000 to 1009. Thus, the range is more instructive if presented as both the maximum and minimum value rather than the difference.


\subsection{Computation of Summaries in R} \label{sect:DescStats}
All summary statistics described above, with the exception of the mode, is  calculated in R with \R{Summarize()}. To summarize a single variable a one-sided formula of the form \R{\TILDE quant} is used, where \R{quant} generically represents the quantitative variable, along with the \R{data=} argument. The number of digits after the decimal place is controlled with \R{digits=}.
\begin{knitrout}
\definecolor{shadecolor}{rgb}{0.922, 0.922, 0.922}\color{fgcolor}\begin{kframe}
\begin{verbatim}
> Summarize(~days,data=LSI,digits=2)
     n nvalid   mean     sd    min     Q1 median     Q3    max 
 42.00  39.00 107.85  21.59  48.00  97.00 114.00 118.00 146.00 
\end{verbatim}
\end{kframe}
\end{knitrout}

From this it is seen that the sample median is 114 days, sample mean is 107.8 days, sample IQR is from 97 to 118 days, the sample standard deviation is 21.59 days, and the range is from 48 to 146.



\newpage
\section{Graphical Summaries}
\subsection{Histogram}
A histogram plots the frequency of individuals (y-axis) in classes of values of the quantitative variable (x-axis). Construction of a histogram begins by creating classes of values for the variable of interest. The easiest way to create a list of classes is to divide the range (i.e., maximum minus minimum value) by a ``nice'' number near eight to ten, and then round up to make classes that are easy to work with. The ``nice'' number between eight and ten is chosen to make the division easy and will be the number of classes. For example, the range of values in the open pit mine example is 15-1 = 14. A ``nice'' value near eight and ten to divide this range by is seven. Thus, the classes should be two units wide (=14/7) and, for ease, will begin at 0 \tabrefp{tab:MineFreqTable}.

\vspace{-6pt}
\begin{table}[htbp]
  \caption{Frequency table of number of countries in two-mine-wide classes.}
  \label{tab:MineFreqTable}
    \begin{Verbatim}[xleftmargin=25mm]
Class      0-1   2-3   4-5   6-7   8-9 10-11 12-13 14-15
Frequency   10     7     5     0     0     2     1     1
    \end{Verbatim}
\end{table}
\vspace{-12pt}

The frequency of individuals in each class is then counted (shown in the second row of \tabref{tab:MineFreqTable}). The plot is prepared with values of the classes forming the x-axis and frequencies forming the y-axis (\figref{fig:MineHist1}A). The first bar added to this skeleton plot has the bottom-left corner at 0 and the bottom-right corner at 2 on the x-axis, and a height equal to the frequency of individuals in the 0 and 1 class (\figref{fig:MineHist1}B). A second bar is then added with the bottom-left corner at 2 and the bottom-right corner at 4 on the x-axis, and a height equal to the frequency of individuals in the 2 and 3 class (\figref{fig:MineHist1}C). This process is continued with the remaining classes until the full histogram is constructed (\figref{fig:MineHist1}D).

\begin{knitrout}
\definecolor{shadecolor}{rgb}{0.922, 0.922, 0.922}\color{fgcolor}\begin{figure}[hbtp]

{\centering \includegraphics[width=.6\linewidth]{Figs/MineHist1-1} 

}

\caption[Steps (described in text) illustrating the construction of a histogram]{Steps (described in text) illustrating the construction of a histogram.}\label{fig:MineHist1}
\end{figure}


\end{knitrout}

Ideally eight to ten classes are used in a histogram. Too many or too few bars make it difficult to identify the shape and may lead to different interpretations. A dramatic example of the effect of changing the number of classes is seen in histograms of the length of eruptions for the Old Faithful geyser \figrefp{fig:histOF}.

\begin{knitrout}
\definecolor{shadecolor}{rgb}{0.922, 0.922, 0.922}\color{fgcolor}


















































\begin{figure}[hbtp]

{\centering \animategraphics[width=.35\linewidth,controls,palindrome,autoplay]{1}{Figs/histOF-}{1}{52}

}

\caption[Histogram of length of eruptions for Old Faitfhul geyser with varying number of classes]{Histogram of length of eruptions for Old Faitfhul geyser with varying number of classes.}\label{fig:histOF}
\end{figure}


\end{knitrout}


\subsection{Boxplot}
The \textbf{five-number summary} consists of the minimum, Q1, median, Q3, and maximum values (effectively contains the range, IQR, and median). For example, the five-number summary for the open pit mine data is 1, 1, 2, 4, and 15 (all values computed in the previous section). The five-number summary may be displayed as a \textbf{boxplot}. A traditional boxplot (\figref{fig:MineBoxplot}-Left) consists of a horizontal line at the median, horizontal lines at Q1 and Q3 that are connected with vertical lines to form a box, and vertical lines from Q1 to the minimum and from Q3 to the maximum. In modern boxplots (\figref{fig:MineBoxplot}-Right) the upper line extends from Q3 to the last observed value that is within 1.5 IQRs of Q3 and the lower line extends from Q1 to the last observed value that is within 1.5 IQRs of Q1. Observed values outside of the whiskers are termed ``outliers'' by this algorithm and are typically plotted with circles or asterisks. If no individuals are deemed ``outliers'' by this algorithm, then the two traditional and modern boxplots will be the same.

\begin{knitrout}
\definecolor{shadecolor}{rgb}{0.922, 0.922, 0.922}\color{fgcolor}\begin{figure}[hbtp]

{\centering \includegraphics[width=.65\linewidth]{Figs/MineBoxplot-1} 

}

\caption[Traditional (Left) and modern (Right) boxplots of the open pit mine data]{Traditional (Left) and modern (Right) boxplots of the open pit mine data.}\label{fig:MineBoxplot}
\end{figure}


\end{knitrout}


\subsection{Construction of Graphs in R}
A simple (by default) histogram is constructed in R with \R{hist()} using a one-sided formula of the form \R{\TILDE quant}, where \R{quant} generically represents the quantitative variable, and the corresponding data frame in \R{data=}.\footnote{Note that this is the same formula used in \R{Summarize()}.} The x-axis label may be improved from the default value by including a label in \R{xlab=}. The width of the classes may be controlled with a positive integer in \R{w=}.\footnote{The endpoints for the classes may also be set by giving a vector of endpoints to \R{breaks=}.}

\begin{knitrout}
\definecolor{shadecolor}{rgb}{0.922, 0.922, 0.922}\color{fgcolor}\begin{kframe}
\begin{verbatim}
> hist(~days,data=LSI,xlab="Days of Ice Cover")      # Fig 5.4-Left
> hist(~days,data=LSI,xlab="Days of Ice Cover",w=20) # Fig 5.4-Right
\end{verbatim}
\end{kframe}\begin{figure}[hbtp]

{\centering \includegraphics[width=.34\linewidth]{Figs/Histogram1-1} 
\includegraphics[width=.34\linewidth]{Figs/Histogram1-2} 

}

\caption[Histograms of the duration of ice cover at ice gauge 9004 in Lake Superior using the default class widths (Left) and widths of 20 days (Right)]{Histograms of the duration of ice cover at ice gauge 9004 in Lake Superior using the default class widths (Left) and widths of 20 days (Right).}\label{fig:Histogram1}
\end{figure}


\end{knitrout}

A modern boxplot of a single variable is constructed in R with \R{boxplot()}, where the first argument is usually a specific variable in a data.frame. Additionally, the y-axis may be properly labeled with \R{ylab=}.

\begin{knitrout}
\definecolor{shadecolor}{rgb}{0.922, 0.922, 0.922}\color{fgcolor}\begin{kframe}
\begin{verbatim}
> boxplot(LSI$days,ylab="Days of Ice Cover")
\end{verbatim}
\end{kframe}\begin{figure}[hbtp]

{\centering \includegraphics[width=.3\linewidth]{Figs/BoxplotLSI-1} 

}

\caption[Boxplot of the duration of ice cover at ice gauge 9004 in Lake Superior]{Boxplot of the duration of ice cover at ice gauge 9004 in Lake Superior.}\label{fig:BoxplotLSI}
\end{figure}


\end{knitrout}

\warn{The default histogram and boxplot should be modified by properly labeling the axes.}


\section{Multiple Groups} \label{sect:MultGroups}
It is common to need to compute numerical or construct graphical summaries of a quantitative variable separately for groups of individuals. In these cases it is beneficial to have a function that will efficiently construct a histogram and compute summary statistics for the quantitative variable separated by the levels of a factor variable. Separate histograms are constructed with \R{hist()}, if the first argument is a ``formula'' of the type \R{quant\TILDE group} where \R{quant} represents the quantitative response variable of interest and \R{group} represents the factor variable that indicates to which group the individual belongs. The data.frame that contains \R{quant} and \R{group} is given to \R{data=}. Summary statistics are separated by group by supplying the same formula and \R{data=} arguments to \R{Summarize()}.

As an example, the LSI data.frame contains a \R{period} variable that indicates whether the ice season was ``pre-1975'' or ``post-1975'' (which included 1975). Thus, one may be interested in examining the distribution of annual days of ice for each of these periods period. Histograms \figrefp{fig:mhist1} and summary statistics separated by period are constructed below.
\begin{knitrout}
\definecolor{shadecolor}{rgb}{0.922, 0.922, 0.922}\color{fgcolor}\begin{kframe}
\begin{verbatim}
> hist(days~period,data=LSI,ylab="Days of Ice Cover",w=20)
> Summarize(days~period,data=LSI,digits=2)
     period  n nvalid   mean    sd min Q1 median  Q3 max
1 post-1975 22     21 106.76 26.01  48 99  116.0 123 146
2  pre-1975 20     18 109.11 15.59  82 97  110.5 118 137
\end{verbatim}
\end{kframe}\begin{figure}[hbtp]

{\centering \includegraphics[width=.8\linewidth]{Figs/mhist1-1} 

}

\caption[Histograms of the duration of ice cover at ice gauge 9004 in Lake Superior by period]{Histograms of the duration of ice cover at ice gauge 9004 in Lake Superior by period.}\label{fig:mhist1}
\end{figure}


\end{knitrout}

Side-by-side boxplots \figrefp{fig:Boxplot1} are an alternative to separated histograms and are constructed by including the same formula and \R{data=} arguments to \R{boxplot()}.
\begin{knitrout}
\definecolor{shadecolor}{rgb}{0.922, 0.922, 0.922}\color{fgcolor}\begin{kframe}
\begin{verbatim}
> boxplot(days~period,data=LSI,ylab="Days of Ice Cover",xlab="Period")
\end{verbatim}
\end{kframe}\begin{figure}[hbtp]

{\centering \includegraphics[width=.4\linewidth]{Figs/Boxplot1-1} 

}

\caption[Boxplot of the duration of ice cover at ice gauge 9004 in Lake Superior by period]{Boxplot of the duration of ice cover at ice gauge 9004 in Lake Superior by period.}\label{fig:Boxplot1}
\end{figure}


\end{knitrout}

Note that the formulae above required the grouping variable to be a factor. In some instances, a grouping variable may appear as an integer variable to R. For example, one may want to explore days of ice by decade, but the decade variable is not a factor variable.
\begin{knitrout}
\definecolor{shadecolor}{rgb}{0.922, 0.922, 0.922}\color{fgcolor}\begin{kframe}
\begin{verbatim}
> str(LSI)
'data.frame':	42 obs. of  5 variables:
 $ season: int  1955 1956 1957 1958 1959 1960 1961 1962 1963 1964 ...
 $ decade: int  1950 1950 1950 1950 1950 1960 1960 1960 1960 1960 ...
 $ period: Factor w/ 2 levels "post-1975","pre-1975": 2 2 2 2 2 2 2 2 2 2 ...
 $ temp  : num  22.9 23 25.7 20 24.8 ...
 $ days  : int  87 137 106 97 105 118 118 136 91 NA ...
\end{verbatim}
\end{kframe}
\end{knitrout}

In these cases, the variale needs to be explicitly converted to a factor variable using \R{factor()}, as shown below. The use of \R{factor()} is not needed if R already recognizes the variable as a factor variable.
\begin{knitrout}
\definecolor{shadecolor}{rgb}{0.922, 0.922, 0.922}\color{fgcolor}\begin{kframe}
\begin{verbatim}
> LSI$decade <- factor(LSI$decade)
> str(LSI)
'data.frame':	42 obs. of  5 variables:
 $ season: int  1955 1956 1957 1958 1959 1960 1961 1962 1963 1964 ...
 $ decade: Factor w/ 5 levels "1950","1960",..: 1 1 1 1 1 2 2 2 2 2 ...
 $ period: Factor w/ 2 levels "post-1975","pre-1975": 2 2 2 2 2 2 2 2 2 2 ...
 $ temp  : num  22.9 23 25.7 20 24.8 ...
 $ days  : int  87 137 106 97 105 118 118 136 91 NA ...
\end{verbatim}
\end{kframe}
\end{knitrout}



\chapter[Univ EDA Quantitative]{Univariate EDA - Quantitative} \label{chap:UnivEDAQuant2}

\minitoc
\vspace{40pt}

\lettrine{A}{ univariate EDA for a quantitative variable} is concerned with describing the distribution of values for that variable; i.e., describing what values occurred and how often those values occurred. Specifically, the distribution is described by four specific attributes:

\vspace{-12pt}
\begin{Enumerate}
  \item \textbf{shape} of the distribution,
  \item presence of \textbf{outliers},
  \item \textbf{center} of the distribution, and
  \item \textbf{dispersion} or spread of the distribution.
\end{Enumerate}
\vspace{-8pt}

Graphs are used to identify shape and the presence of outliers and to get a general feel for center and dispersion. Numerical summaries, however, are used to specifically describe center and dispersion of the variable. Computing and constructing the required numerical and graphical summaries was described in \modref{chap:UnivEDAQuant1}. Those summaries are interpreted here to provide an overall description of the distribution of the quantitative variable.

The same three data sets used in \modref{chap:UnivEDAQuant1} are used here.

\vspace{-12pt}
\begin{Itemize}
  \item Number of open pit mines in countries with open pit mines \tabrefp{tab:MineData}.
  \item Richter scale recordings for 15 major earthquakes \tabrefp{tab:EQData}.
  \item The number of days of ice cover at ice gauge station 9004 in Lake Superior.
\end{Itemize}

\section{Interpreting Shape}
A distribution has two tails -- a left-tail of smaller or more negative values and a right-tail of larger or more positive values \figrefp{fig:ShapeExamples1}. The relative appearance of these two tails is used to identify three different shapes of distributions -- symmetric, left-skewed, and right-skewed. If the left- and right-tail of a distribution are approximately equal in shape (length and height), then the distribution is said to be \textbf{symmetric} (or more specifically \textbf{approximately symmetric}). If the left-tail is stretched out or is longer and flatter than the right-tail, then the distribution is negatively- or \textbf{left-skewed}. If the right-tail is stretched out or is longer and flatter than the left-tail, then the distribution is positively- or \textbf{right-skewed}. The type of skew is defined by the longer tail; a longer right-tail means the distribution is right-skewed and a longer left-tail means it is left-skewed.

\begin{knitrout}
\definecolor{shadecolor}{rgb}{0.922, 0.922, 0.922}\color{fgcolor}\begin{figure}[hbtp]

{\centering \includegraphics[width=.9\linewidth]{Figs/ShapeExamples1-1} 

}

\caption[Examples of left-skewed (center), symmetric (left), and right-skewed (right) distributions]{Examples of left-skewed (center), symmetric (left), and right-skewed (right) distributions.}\label{fig:ShapeExamples1}
\end{figure}


\end{knitrout}

\warn{The longer tail defines the type of skew.}

In practice, these labels form a continuum. For example, it may be difficult to discern whether the shape approximately symmetric or one of the skewed distributions. To partially address this issue, ``slightly'' or ``strongly'' may be used with ``skewed'' to distinguish whether the distribution is obviously skewed (i.e., ``strongly skewed'') or nearly symmetric (i.e., ``slightly skewed'').

\warn{Symmetric, left-skewed, and right-skewed descriptors are guides; many ``real'' distributions will not fall neatly into these categories.}

The shape of a distribution is most easily identified from a histogram. Histograms that are examples of each shape are in \figref{fig:ShapeExamples2}. For the sets of skewed distributions, the distributions are less strongly skewed from left-to-right.

\begin{knitrout}
\definecolor{shadecolor}{rgb}{0.922, 0.922, 0.922}\color{fgcolor}\begin{figure}[hbtp]

{\centering \includegraphics[width=.7\linewidth]{Figs/ShapeExamples2-1} 

}

\caption[Examples of approximately symmetric (top, red), left-skewed (middle, blue), and right-skewed (bottom, green) histograms]{Examples of approximately symmetric (top, red), left-skewed (middle, blue), and right-skewed (bottom, green) histograms. Note that the axes labels were removed to focus on the shape of the histograms.}\label{fig:ShapeExamples2}
\end{figure}


\end{knitrout}

The shape of a distribution can also be determined from a boxplot. The relative length from the median to Q1 and the median to Q3 (i.e., the relative position of the median line in the box) indicates the shape of the distribution. If the distribution is left-skewed (i.e., lesser-valued individuals are ``spread out''; \figref{fig:BoxplotShape}-Right), then median-Q1 will be greater than Q3-median. In contrast, if the distribution is right-skewed (i.e., larger-valued individuals are spread out; \figref{fig:BoxplotShape}-Middle), then Q3-median will be greater than median-Q1. Thus, the median is nearer the top of the box for a left-skewed distribution, nearer the bottom of the box for a right-skewed distribution, and nearer the center of the box for a symmetric distribution \figrefp{fig:BoxplotShape}.

\begin{knitrout}
\definecolor{shadecolor}{rgb}{0.922, 0.922, 0.922}\color{fgcolor}\begin{figure}[hbtp]

{\centering \includegraphics[width=.25\linewidth]{Figs/BoxplotShape-1} 
\includegraphics[width=.25\linewidth]{Figs/BoxplotShape-2} 
\includegraphics[width=.25\linewidth]{Figs/BoxplotShape-3} 
\includegraphics[width=.25\linewidth]{Figs/BoxplotShape-4} 
\includegraphics[width=.25\linewidth]{Figs/BoxplotShape-5} 
\includegraphics[width=.25\linewidth]{Figs/BoxplotShape-6} 

}

\caption[Histograms and boxplots for several different shapes of distributions]{Histograms and boxplots for several different shapes of distributions.}\label{fig:BoxplotShape}
\end{figure}


\end{knitrout}

\warn{Even though shape can be described from a boxplot, it is always easier to describe shape from a histogram.}

\section{Interpreting Outliers}
An outlier is an individual whose value is widely separated from the main cluster of values in the sample. On histograms, outliers appear as bars that are separated from the main cluster of bars by ``white space'' or areas with no bars \figrefp{fig:OutlierExHist}. In general, outliers must be \textbf{on the margins of the histogram, should be separated by one or two missing bars, and should only be one or two individuals.}

\begin{knitrout}
\definecolor{shadecolor}{rgb}{0.922, 0.922, 0.922}\color{fgcolor}\begin{figure}[hbtp]

{\centering \includegraphics[width=.4\linewidth]{Figs/OutlierExHist-1} 

}

\caption[Example histogram with an outlier to the right]{Example histogram with an outlier to the right.}\label{fig:OutlierExHist}
\end{figure}


\end{knitrout}

An outlier may be a result of human error in the sampling process. If this is the case, then the value should be corrected or removed. Other times an outlier may be an individual that was not part of the population of interest -- e.g., an adult animal that was sampled when only immature animals were being considered. In this case, the individual should be removed from the sample. Still other times, an outlier is part of the population and should generally not be removed from the sample. In fact you may wish to highlight an outlier as an interesting observation! Regardless, it is important that you construct a histogram to determine if outliers are present or not.

Don't let outliers completely influence how you define the shape of a distribution. For example, if the main cluster of values is approximately symmetric and there is one outlier to the right of the main cluster (as illustrated in \figref{fig:OutlierExHist}), \textbf{DON'T} call the distribution right-skewed. You should describe this distribution as approximately symmetric with an outlier to the right.

\warn{Not all outliers warrant removal from your sample.}

\vspace{-12pt}
\warn{Don't let outliers completely influence how you define the shape of a distribution.}



\section{Comparing the Median and Mean} \label{sect:MeanMedian}
As mentioned previously, numerical measures will be used to describe the center and dispersion of a distribution. However, which values should be used? Should one use the mean or the median as a measure of center? Should one use the IQR or the standard deviation as a measure of dispersion? Which measures are used depends on how the measures respond to skew and the presence of outliers. Thus, before stating a rule for which measures should be used, a fundamental difference among the measures discussed in \modref{chap:UnivEDAQuant1} is explored here.

The following discussion is focused on comparing the mean and the median. However, note that the IQR is fundamentally linked to the median (i.e., to find the IQR, the median must first be found) and the standard deviation is fundamentally linked to the mean (i.e., to find the standard deviation, the mean must first be found). Thus, \textbf{the median and IQR will always be used together to measure center and dispersion, as will the mean and standard deviation.}

The mean and median measure center in different ways. The median balances the number of individuals smaller and larger than it. The mean, on the other hand, balances the sum of the distances from it to all points smaller than it and the sum of the distances from it to all points greater than it. Thus, the median is primarily concerned with the \textbf{position} of the value rather than the value itself, whereas the mean is very much concerned about the \textbf{values} for each individual (i.e., the values are used to find the ``distance'' from the mean).

\warn{The actual values of the data (beyond ordering the data) are not considered when calculating the median; whereas the actual values are very much considered when calculating the mean.}

A plot of the Richter scale data against the corresponding ordered individual number is shown in \figref{fig:MeanMedianComp1}-Left.\footnote{This is a rather non-standard graph but it is useful for comparing how the mean and median measure the center of the data.}  The median (blue line) is found by locating the middle position on the individual number axis and then finding the corresponding Richter scale value (move right until the point is intercepted and then move down to the x-axis). The vertical blue line represents the median; i.e., it has the same \textbf{number} of individuals (i.e., points) above and below it. In contrast, the mean finds the Richter scale value that has the same total distance to values below it as total distance to values above it. In other words, the mean is the vertical red line placed such that the total \textbf{length} of the horizontal dashed red lines is the same to the left as it is to the right. Thus, the median balances the number of individuals above and below the median, whereas the mean balances the total difference in values above and below the mean.

\begin{knitrout}
\definecolor{shadecolor}{rgb}{0.922, 0.922, 0.922}\color{fgcolor}\begin{figure}[hbtp]

{\centering \includegraphics[width=.45\linewidth]{Figs/MeanMedianComp1-1} 
\includegraphics[width=.45\linewidth]{Figs/MeanMedianComp1-2} 

}

\caption[Plot of the individual number versus Richter scale values for the original earthquake data (\textbf{Left}) and the earthquake data with an extreme outlier (\textbf{Right})]{Plot of the individual number versus Richter scale values for the original earthquake data (\textbf{Left}) and the earthquake data with an extreme outlier (\textbf{Right}). The median value is shown as a blue vertical line and the mean value is shown as a red vertical line. Differences between each individual value and the mean value are shown with horizontal red lines.}\label{fig:MeanMedianComp1}
\end{figure}


\end{knitrout}

\warn{The mean balances the distance to individuals above and below the mean. The median balances the number of individuals above and below the median.}

\vspace{-12pt}
\warn{The sum of all differences between individual values and the mean (as properly calculated) equals zero.}

The mean and median differ in their sensitivity to outliers (\figref{fig:MeanMedianComp1}-Right). For example, suppose that an incredible earthquake with a Richter Scale value of 19.0 was added to the earthquake data set. With this additional individual, the median increases from 7.1 to 7.2, but the mean increases from 7.1 to 7.8. The outlier impacts the value of the mean more than the value of the median because of the way that each statistic measures center. The mean will be pulled towards an outlier because it must ``put'' many values on the ``side'' of the mean away from the outlier so that the sum of the differences to the larger values and the sum of the differences to the smaller values will be equal. In this example, the outlier creates a large difference to the right of the mean such that the mean has to ``move'' to the right to make this difference smaller, move more individuals to the left side of the mean, and increase the differences of individuals to the left of the mean to balance this one large individual. The median on the other hand will simply ``put'' one more individual on the side opposite of the outlier because it balances the number of individuals on each side of it. Thus, the median has to move very little to the right to accomplish this balance.

\warn{The mean is more sensitive (i.e., changes more) to outliers than the median; it will be ``pulled'' towards the outlier more than the median.}

The shape of the distribution, even if outliers are not present, also has an impact on the mean and median \figrefp{fig:MeanMedianShape}. If a distribution is approximately symmetric, then the median and mean (along with the mode) will be nearly identical. If the distribution is left-skewed, then the mean will be less than the median. Finally, if the distribution is right-skewed, then the mean will be greater than the median.

\begin{knitrout}
\definecolor{shadecolor}{rgb}{0.922, 0.922, 0.922}\color{fgcolor}\begin{figure}[hbtp]

{\centering \includegraphics[width=.3\linewidth]{Figs/MeanMedianShape-1} 
\includegraphics[width=.3\linewidth]{Figs/MeanMedianShape-2} 
\includegraphics[width=.3\linewidth]{Figs/MeanMedianShape-3} 

}

\caption[Three differently shaped histograms with vertical lines superimposed at the median (M]{Three differently shaped histograms with vertical lines superimposed at the median (M; blue lines) and the mean ($\bar{x}$; red lines).}\label{fig:MeanMedianShape}
\end{figure}


\end{knitrout}

\warn{The mean is pulled towards the long tail of a skewed distribution. Thus, the mean is greater than the median for right-skewed distributions and the mean is less than the median for left-skewed distributions.}

As shown above, the mean and median measure center in different ways. The question now becomes ``which measure of center is better?''  The median is a ``better'' measure of center when outliers are present. In addition, the median gives a better measure of a typical individual when the data are skewed. Thus, in this course, the median is used when outliers are present or the distribution of the data is skewed. If the distribution is symmetric, then the purpose of the analysis will dictate which measure of center is ``better.''  However, in this course, use the mean when the data are symmetric or, at least, not strongly skewed.

As note above, the IQR and standard deviation behave similarly to the median and mean, respectively, in the face of outliers and skews. Specifically, the IQR is less sensitive to outliers than the standard deviation.


\section{Synthetic Interpretations}
The graphical and numerical summaries from \modref{chap:UnivEDAQuant1} and the rationale described above can be used to construct a synthetic description of the shape, outliers, center, and dispersion of the distribution of a quantitative variable. In the examples below specifically note the 1) reference to figures and tables, 2) labeling of the figures and tables, 3) that only the mean and standard deviation or the median and IQR are discussed, 4) the range was not used alone as a measure of dispersion, 5) the explanation for why either the median and IQR or the mean and standard deviation were used, and 6) an appendix of R code used was provided.

\subsubsection{Number of Open Pit Mines}
\begin{quote}
\textit{Construct a proper EDA for the following situation and data -- ``The number of open pit mines in countries that have open pit mines \tabrefp{tab:MineData}.''}
\end{quote}
\vspace{-12pt}

The number of open pit mines in countries with open pit mines is strongly right-skewed with no outliers present \figrefp{fig:MineHist2}. [\textit{I did not call the group of four countries with 10 or more open pit mines outliers because there were more than one or two countries there.}] The center of the distribution is best measured by the median, which is 2 \tabrefp{tab:MineStats}. The range of open pit mines in the sample is from 1 to 15 while the dispersion as measured by the inter-quartile range (IQR) is from a Q1 of 1.0 to a Q3 of  4.0 \tabrefp{tab:MineStats}. I chose to use the median and IQR because the distribution was strongly skewed.

% latex table generated in R 3.4.0 by xtable 1.8-2 package
% Wed Jun 14 09:31:38 2017
\begin{table}[ht]
\centering
\caption{Descriptive statistics of number of open pit mines in countries with open pit mines.} 
\label{tab:MineStats}
\begin{tabular}{rrrrrrrr}
  \hline
n & mean & sd & min & Q1 & median & Q3 & max \\ 
  \hline
26.0 & 3.6 & 4.0 & 1.0 & 1.0 & 2.0 & 4.0 & 15.0 \\ 
   \hline
\end{tabular}
\end{table}


\begin{knitrout}
\definecolor{shadecolor}{rgb}{0.922, 0.922, 0.922}\color{fgcolor}\begin{figure}[hbtp]

{\centering \includegraphics[width=.4\linewidth]{Figs/MineHist2-1} 

}

\caption[Histogram of number of open pit mines in countries with open pit mines]{Histogram of number of open pit mines in countries with open pit mines.}\label{fig:MineHist2}
\end{figure}


\end{knitrout}

\begin{minipage}{\textwidth}
R Code Appendix:
\begin{knitrout}
\definecolor{shadecolor}{rgb}{0.922, 0.922, 0.922}\color{fgcolor}\begin{kframe}
\begin{verbatim}
setwd("c:/data/")
mc <- read.csv("MineData.csv")
str(mc)
Summarize(~mines,data=mc,digits=1)
hist(~mines,data=mc,w=2,xlab="Number of open pit mines")
\end{verbatim}
\end{kframe}
\end{knitrout}
\end{minipage}


\subsubsection{Lake Superior Ice Cover}
\begin{quote}
\textit{Thoroughly describe the distribution of number of days of ice cover at ice gauge station 9004 in Lake Superior (data are in \href{https://raw.githubusercontent.com/droglenc/NCData/master/LakeSuperiorIce.csv}{LakeSuperiorIce.csv}).}
\end{quote}



The shape of number of days of ice cover at gauge 9004 in Lake Superior is approximately symmetric with no obvious outliers \figrefp{fig:LSIHist}. The center is at a mean of 107.8 days and the dispersion is a standard deviation of 21.6 days \tabrefp{tab:LSIStats}. The mean and standard deviation were used because the distribution was not strongly skewed and no outlier was present.

\begin{knitrout}
\definecolor{shadecolor}{rgb}{0.922, 0.922, 0.922}\color{fgcolor}\begin{figure}[hbtp]

{\centering \includegraphics[width=.4\linewidth]{Figs/LSIHist-1} 

}

\caption[Histogram of number of days of ice cover at ice gauge 9004 in Lake Superior]{Histogram of number of days of ice cover at ice gauge 9004 in Lake Superior.}\label{fig:LSIHist}
\end{figure}


\end{knitrout}

% latex table generated in R 3.4.0 by xtable 1.8-2 package
% Wed Jun 14 09:31:38 2017
\begin{table}[ht]
\centering
\caption{Descriptive statistics of number of days of ice cover at ice gauge 9004 in Lake Superior..} 
\label{tab:LSIStats}
\begin{tabular}{rrrrrrrrr}
  \hline
n & nvalid & mean & sd & min & Q1 & median & Q3 & max \\ 
  \hline
42.0 & 39.0 & 107.8 & 21.6 & 48.0 & 97.0 & 114.0 & 118.0 & 146.0 \\ 
   \hline
\end{tabular}
\end{table}


\begin{minipage}{\textwidth}
R Appendix:
\begin{knitrout}
\definecolor{shadecolor}{rgb}{0.922, 0.922, 0.922}\color{fgcolor}\begin{kframe}
\begin{verbatim}
setwd("c:/data/")
LSI <- read.csv("LakeSuperiorIce.csv")
str(LSI)
hist(~days,data=LSI,xlab="Day of Ice Cover",ylab="Frequency of Years",w=20)
Summarize(~days,data=LSI,digits=1)
\end{verbatim}
\end{kframe}
\end{knitrout}
\end{minipage}

\subsubsection{Crayfish Temperature Selection}
\begin{quote}
\textit{Peck (1985) examined the temperature selection of dominant and subdominant crayfish (\textit{Orconectes virilis}) together in an artificial stream. The temperature ($^{o}$C) selection by the dominant crayfish in the presence of subdominant crayfish in these experiments was recorded below. Thoroughly describe all aspects of the distribution of selected temperatures.}
\end{quote}

% latex table generated in R 3.4.0 by xtable 1.8-2 package
% Wed Jun 14 09:31:38 2017
\begin{tabular}{rrrrrrrrrrrrrrrr}
  30 & 26 & 26 & 26 & 25 & 25 & 25 & 25 & 25 & 24 & 24 & 24 & 24 & 24 & 24 & 23 \\ 
  23 & 23 & 23 & 22 & 22 & 22 & 22 & 21 & 21 & 21 & 20 & 20 & 19 & 19 & 18 & 16 \\ 
  \end{tabular}


The shape of temperatures selected by the dominant crayfish is slightly left-skewed \figrefp{fig:CrayfishTempHist} with a possible weak outlier at the maximum value of 30$^{o}$C \tabrefp{tab:CrayfishTempStats}. The center is best measured by the median, which is 23$^{o}$C \tabrefp{tab:CrayfishTempStats} and the dispersion is best measured by the IQR, which is from 21 to 25$^{o}$C \tabrefp{tab:CrayfishTempStats}. I used the median and IQR because of the (combined) skewed shape and outlier present.

\begin{knitrout}
\definecolor{shadecolor}{rgb}{0.922, 0.922, 0.922}\color{fgcolor}\begin{figure}[hbtp]

{\centering \includegraphics[width=.4\linewidth]{Figs/CrayfishTempHist-1} 

}

\caption[Histogram of crayfish temperature preferences]{Histogram of crayfish temperature preferences.}\label{fig:CrayfishTempHist}
\end{figure}


\end{knitrout}

% latex table generated in R 3.4.0 by xtable 1.8-2 package
% Wed Jun 14 09:31:38 2017
\begin{table}[ht]
\centering
\caption{Descriptive statistics of crayfish temperature preferences.} 
\label{tab:CrayfishTempStats}
\begin{tabular}{rrrrrrrr}
  \hline
n & mean & sd & min & Q1 & median & Q3 & max \\ 
  \hline
32.00 & 22.88 & 2.79 & 16.00 & 21.00 & 23.00 & 25.00 & 30.00 \\ 
   \hline
\end{tabular}
\end{table}


\begin{minipage}{\textwidth}
R Appendix:
\begin{knitrout}
\definecolor{shadecolor}{rgb}{0.922, 0.922, 0.922}\color{fgcolor}\begin{kframe}
\begin{verbatim}
setwd("c:/data/")
cray <- read.csv("Crayfish.csv")
str(cray)
hist(~temp,data=cray,xlab="Preferred Temperature",ylab="Frequency of Crayfish",w=2)
Summarize(~temp,data=cray,digits=2)
\end{verbatim}
\end{kframe}
\end{knitrout}
\end{minipage}



\chapter{Univariate EDA - Categorical} \label{chap:UnivEDACat}

\vspace{-24pt}
\minitoc
\vspace{12pt}

\lettrine{I}{nterpreting summaries of a} single categorical variable is more intuitive and less defined than that for quantitative data. Specifically, one DOES NOT describe shape, center, dispersion, and outliers for categorical data. In this module, methods to construct tables and graphs for categorical data are described and the interpretation of the results demonstrated.

\warn{Do not describe shape, center, dispersion, and outliers for a categorical variable.}

These concepts are illustrated with three data sets. First, data recorded about MTH107 students in the Winter 2010 semester will be used. Specifically, whether or not a student was required to take the courses and the student's year-in-school will be summarized. Whether or not a student was required to take the course for a subset of individuals is shown in \tabref{tab:MTH107Subset}.

\begin{table}[htbp]
  \caption{Whether (Y) or not (N) MTH107 was required for eight individuals in MTH107 in Winter 2010.}
  \label{tab:MTH107Subset}
  \centering
  \begin{Verbatim}[xleftmargin=10mm]
Individual  1  2  3  4  5  6  7  8
Required    Y  N  N  Y  Y  Y  N  Y
  \end{Verbatim}
\end{table}
\vspace{-12pt}

Second, the General Sociological Survey (GSS) is a very large survey that has been administered 25 times since 1972. The purpose of the GSS is to gather data on contemporary American society in order to monitor and explain trends in attitudes, behaviors, and attributes. One question that was asked in a recent GSS was ``How often do you make a special effort to sort glass or cans or plastic or papers and so on for recycling?''  These data are found in the \var{recycle} variable in \href{https://raw.githubusercontent.com/droglenc/NCData/master/GSSEnviroQues.csv}{GSSEnviroQues.csv}.


\section{Summaries}
\subsection{Frequency and Percentage Tables}
A simple method to summarize categorical data is to count the number of individuals in each level of the categorical variable. These counts are called frequencies and the resulting table \tabrefp{tab:MTH107SubsetFreq} is called a frequency table. From this table, it is seen that there were five students that were required and three that were not required to take MTH107.

\begin{table}[htbp]
  \caption{Frequency table for whether MTH107 was required (Y) or not (N) for eight individuals in MTH107 in Winter 2010.}
  \label{tab:MTH107SubsetFreq}
  \centering
  \begin{Verbatim}[xleftmargin=15mm]
Required  Freq
    Y       5
    N       3
  \end{Verbatim}
\end{table}

The remainder of this module will use results from the entire class rather than the subset used above. For example, frequency tables of individuals by sex and year-in-school for the entire class are in \tabref{tab:Mth107Freq}.

\begin{table}[htbp]
  \caption{Frequency tables for whether (Y) or not (N) MTH107 was required (Left) and year-in-school (Right) for all individuals in MTH107 in Winter 2010.}
  \label{tab:Mth107Freq}
  \centering
  \begin{Verbatim}[xleftmargin=15mm]
Required  Freq          Year  Freq
    Y      38            Fr    19
    N      30            So    12
                         Jr    29
                         Sr     9
   \end{Verbatim}
\end{table}

Frequency tables are often modified to show the percentage of individuals in each level. \textbf{Percentage tables} are constructed from frequency tables by dividing the number of individuals in each level by the total number of individuals examined ($n$) and then multiplying by 100. For example, the percentage tables for both whether or not MTH107 was required and year-in-school \tabrefp{tab:Mth107Prop} for students in MTH107 is constructed from \tabref{tab:Mth107Freq} by dividing the value in each cell by 68, the total number of students in the class, and then multiplying by 100. From this it is seen that 55.9\% of students were required to take the course and 13.2\% were seniors \tabrefp{tab:Mth107Prop}.

\begin{table}[htbp]
  \caption{Percentage tables for whether (Y) or not (N) MTH107 was required (Left) and year-in-school (Right) for all individuals in MTH107 in Winter 2000.}
  \label{tab:Mth107Prop}
  \centering
  \begin{Verbatim}[xleftmargin=5mm]
Required   Perc          Year   Perc
   Y       55.9           Fr    27.9
   N       44.1           So    17.6
                          Jr    42.6
                          Sr    13.2
  \end{Verbatim}
\end{table}

\subsection{Bar Plots}
Bar plots, or bar charts, are used to display the frequency or percentage of individuals in each level of a categorical variable. Bar plots look similar to histograms in that they have the frequency of individuals on the y-axis. However, category labels rather than quantitative values are plotted on the x-axis. In addition, to highlight the categorical nature of the data, bars on a bar plot do not touch. A bar plot for whether or not individuals were required to take MTH107 is in \figref{fig:MTH107BarChart}-Left. This bar plot does not add much to the frequency table because there were only two categories. However, bar plots make it easier to compare the number of individuals in each of several categories as in \figref{fig:MTH107BarChart}-Right.

\begin{knitrout}
\definecolor{shadecolor}{rgb}{0.922, 0.922, 0.922}\color{fgcolor}\begin{figure}[hbtp]

{\centering \includegraphics[width=.4\linewidth]{Figs/MTH107BarChart-1} 
\includegraphics[width=.4\linewidth]{Figs/MTH107BarChart-2} 

}

\caption[Bar charts of the frequency of individuals in MTH107 during Winter 2010 by whether or not they were required to take MTH107 (\textbf{Left}) and year-in-school (\textbf{Right})]{Bar charts of the frequency of individuals in MTH107 during Winter 2010 by whether or not they were required to take MTH107 (\textbf{Left}) and year-in-school (\textbf{Right}).}\label{fig:MTH107BarChart}
\end{figure}


\end{knitrout}

\warn{Bar charts are used to display the frequency of individuals in the categories of a categorical variable. Histograms are used to display the frequency of individuals in classes created from quantitative variables.}


\subsection{Using in R}
The General Sociological Survey (GSS) data are loaded, the structure of the data.frame is examined, and the levels of the \var{recycle} variable are shown below. These results show the five levels in the \var{recycle} factor variable, ordered alphabetically as is the default in R. However, the levels should be ``Always'', ``Often'', ``Sometimes'', ``Never'', and ``Not Avail'' to follow the natural order of this ordinal variable.

\begin{knitrout}
\definecolor{shadecolor}{rgb}{0.922, 0.922, 0.922}\color{fgcolor}\begin{kframe}
\begin{verbatim}
> GSS <- read.csv("data/GSSEnviroQues.csv")
\end{verbatim}
\end{kframe}
\end{knitrout}
\begin{knitrout}
\definecolor{shadecolor}{rgb}{0.922, 0.922, 0.922}\color{fgcolor}\begin{kframe}
\begin{verbatim}
> str(GSS)
'data.frame':	3539 obs. of  2 variables:
 $ recycle: Factor w/ 5 levels "Always","Never",..: 1 1 1 1 1 1 1 1 1 1 ...
 $ tempgen: Factor w/ 5 levels "Extremely","Not",..: 1 1 1 1 1 1 1 1 1 1 ...
> levels(GSS$recycle)
[1] "Always"    "Never"     "Not Avail" "Often"     "Sometimes"
\end{verbatim}
\end{kframe}
\end{knitrout}

The order of a factor variable is controlled by including the ordered level names within a vector given to \R{levels=} in \R{factor()}. The names of the levels in this vector must be exactly as they appear in the original variable and they must be contained within quotes. The levels of \var{recycle} were reordered below. The advantage of correcting this order is that when the summary table is made, the order will follow the natural order of the variable rather than the alphabetical order.
\begin{knitrout}
\definecolor{shadecolor}{rgb}{0.922, 0.922, 0.922}\color{fgcolor}\begin{kframe}
\begin{verbatim}
> lvls <- c("Always","Often","Sometimes","Never","Not Avail")
> GSS$recycle <- factor(GSS$recycle,levels=lvls)
> levels(GSS$recycle)
[1] "Always"    "Often"     "Sometimes" "Never"     "Not Avail"
\end{verbatim}
\end{kframe}
\end{knitrout}

\warn{When changing the order of the levels with the \R{levels=} argument, the level names must be contained within quotes and they must be spelled exactly as they were spelled in the original variable.}

A frequency table of a single categorical variable is computed with \R{xtabs()}, where the first argument is a one-sided formula of the form \R{\TILDE var} and the corresponding data.frame is in \R{data=}. The result from \R{xtabs()} should be assigned to an object for further use. For example, the frequency table is produced, stored in \R{tabRecycle}, and displayed below. Thus, 1289 respondents answered ``Always'' to the recycling question.
\begin{knitrout}
\definecolor{shadecolor}{rgb}{0.922, 0.922, 0.922}\color{fgcolor}\begin{kframe}
\begin{verbatim}
> ( tabRecycle <- xtabs(~recycle,data=GSS) )
recycle
   Always     Often Sometimes     Never Not Avail 
     1289       850       823       448       129 
\end{verbatim}
\end{kframe}
\end{knitrout}

A percentage table is computed by including the saved frequency table as the first argument to \R{percTable()}.\footnote{Thus, \R{xtabs()} must be completed and saved to an object before \R{percTable()}.} The number of digits of output is controlled with \R{digits=}. Thus, 36.4\% of respondents answered ``Always'' to the recycling question.
\begin{knitrout}
\definecolor{shadecolor}{rgb}{0.922, 0.922, 0.922}\color{fgcolor}\begin{kframe}
\begin{verbatim}
> percTable(tabRecycle,digits=1)
recycle
   Always     Often Sometimes     Never Not Avail 
     36.4      24.0      23.3      12.7       3.6 
\end{verbatim}
\end{kframe}
\end{knitrout}

A bar plot is produced by giving the saved \R{xtabs()} object as the first argument to \R{barplot()}. The x- and y-axes may be explicitly labeled with \R{xlab=} and \R{ylab=}, respectively. For example, the bar plot for the recycling data \figrefp{fig:Barchart1} is produced below.
\begin{knitrout}
\definecolor{shadecolor}{rgb}{0.922, 0.922, 0.922}\color{fgcolor}\begin{kframe}
\begin{verbatim}
> barplot(tabRecycle,ylab="Frequency",xlab="Recycle Response")
\end{verbatim}
\end{kframe}\begin{figure}[hbtp]

{\centering \includegraphics[width=.55\linewidth]{Figs/Barchart1-1} 

}

\caption[Bar chart of the frequency of responses to the recycling question on the GSS]{Bar chart of the frequency of responses to the recycling question on the GSS.}\label{fig:Barchart1}
\end{figure}


\end{knitrout}


\section{Example Interpretations}
For categorical data, an appropriate EDA consists of identifying the major characteristics among the categories. Shape, center, dispersion, and outliers are NOT described for categorical data because the data is not numerical and, if nominal, no order exists. In general, the major characteristics of the table or graph are described from an intuitive basis. For example, there were more males than females in the Winter 2010 MTH107 class and mostly juniors and Freshmen. Other examples are below.

\subsection{Mixture Seed Count}
\begin{quote}
\textit{A bag of seeds was purchased for seeding a recently constructed wetland. The purchaser wanted to determine if the percentage of seeds in four broad categories -- ``grasses'', ``sedges'', ``wildflowers'', and ``legumes'' -- was similar to what the seed manufacturer advertised. The purchaser examined a 0.25-lb sample of seeds from the bag and recorded the results in \href{https://raw.githubusercontent.com/droglenc/NCData/master/WetlandSeeds.csv}{WetlandSeeds.csv}. Use these data to describe the distribution of seed counts into the four broad categories.}
\end{quote}



The majority of seeds were either sedge or grass with sedge being more than twice as abundant as grass (\tabref{tab:SeedTable}; \figref{fig:SeedBarplot}). Very few legumes or wildflowers were found in the sample.

\begin{knitrout}
\definecolor{shadecolor}{rgb}{0.922, 0.922, 0.922}\color{fgcolor}\begin{figure}[hbtp]

{\centering \includegraphics[width=.5\linewidth]{Figs/SeedBarplot-1} 

}

\caption[Barplot of the percentage of wetland seeds by type]{Barplot of the percentage of wetland seeds by type.}\label{fig:SeedBarplot}
\end{figure}


\end{knitrout}

% latex table generated in R 3.4.0 by xtable 1.8-2 package
% Wed Jun 14 09:31:38 2017
\begin{table}[ht]
\centering
\caption{Percentage distribution of wetland seeds by type.} 
\label{tab:SeedTable}
\begin{tabular}{rrrr}
  \hline
grass & legume & sedge & wildflower \\ 
  \hline
27.9 & 1.6 & 64.5 & 6.0 \\ 
   \hline
\end{tabular}
\end{table}


\begin{minipage}{\textwidth}
R Appendix:
\begin{knitrout}
\definecolor{shadecolor}{rgb}{0.922, 0.922, 0.922}\color{fgcolor}\begin{kframe}
\begin{verbatim}
ws <- read.csv("data/WetlandSeeds.csv")
str(ws)
wtbl <- xtabs(~type,data=ws)
percTable(wtbl,digits=1)
barplot(wptbl[-5],ylab="Percentage of Total Seeds",xlab="Seed Type")
\end{verbatim}
\end{kframe}
\end{knitrout}
\end{minipage}



\chapter{Normal Distribution}  \label{chap:NormDist}

\vspace{-24pt}
\minitoc
\vspace{24pt}

\lettrine{A}{ model for the distribution} of a single quantitative variable can be visualized by ``fitting'' a smooth curve to a histogram (\figref{fig:NormDensityEx}-Left), removing the histogram (\figref{fig:NormDensityEx}-Center), and using the remaining curve (\figref{fig:NormDensityEx}-Right) as a model for the distribution of the entire population of individuals. The smooth red curve drawn over the histogram serves as a model for the distribution of the \textbf{entire population}. If the smooth curve follows a known distribution, then certain calculations are greatly simplified.

\begin{knitrout}
\definecolor{shadecolor}{rgb}{0.922, 0.922, 0.922}\color{fgcolor}\begin{figure}[hbtp]

{\centering \includegraphics[width=.95\linewidth]{Figs/NormDensityEx-1} 

}

\caption[Depiction of fitting a smooth curve to a histogram to serve as a model for the distribution]{Depiction of fitting a smooth curve to a histogram to serve as a model for the distribution.}\label{fig:NormDensityEx}
\end{figure}


\end{knitrout}

The normal distribution is one of the most important distributions in statistics because it serves as a model for the distribution of individuals in many natural situations and the distribution of statistics from repeated samplings (i.e., sampling distributions).\footnote{See \modref{chap:SamplingDist}.}  The use of a normal distribution model to make certain calculations is demonstrated in this module.


\section{Characteristics of a Normal Distribution}
The normal distribution is the familiar bell-shaped curve (\figref{fig:NormDensityEx}-Right). Normal distributions have two parameters -- the population mean, $\mu$, and the population standard deviation, $\sigma$ -- that control the exact shape and position of the distribution. Specifically, the mean $\mu$ controls the center and the standard deviation $\sigma$ controls the dispersion of the distribution \figrefp{fig:NormMultDists}.

\begin{knitrout}
\definecolor{shadecolor}{rgb}{0.922, 0.922, 0.922}\color{fgcolor}\begin{figure}[hbtp]

{\centering \includegraphics[width=.8\linewidth]{Figs/NormMultDists-1} 

}

\caption[Nine normal distributions]{Nine normal distributions. Distributions with the same line type have the same value of $\mu$ (solid is $\mu$=0, dashed is $\mu$=2, dotted is $\mu$=5). Distributions with the same color have the same value of $\sigma$ (black is $\sigma$=0.5, red is $\sigma$=1, and green is $\sigma$=2).}\label{fig:NormMultDists}
\end{figure}


\end{knitrout}

There are an infinite number of normal distributions because there are an infinite number of combinations of $\mu$ and $\sigma$. However, each normal distribution will
\begin{Enumerate}
  \item be bell-shaped and symmetric,
  \item centered at $\mu$,
  \item have inflection points at $\mu \pm \sigma$, and
  \item have a total area under the curve equal to 1.
\end{Enumerate}

If a generic variable $X$ follows a normal distribution with a mean of $\mu$ and a standard deviation of $\sigma$, then it is said that $X\sim N(\mu,\sigma)$. For example, if the heights of students ($H$) follows a normal distribution with a $\mu$ of 66 and a $\sigma$ of 3, then it is said that $H\sim N(66,3)$. As another example, $Z\sim N(0,1)$ means that the variable $Z$ follows a normal distribution with a mean of $\mu$=0 and a standard deviation of $\sigma$=1.


\section{Simple Areas Under the Curve}
A common problem is to determine the proportion of individuals with a value of the variable between two numbers. For example, you might be faced with determining the proportion of all sites that have lead concentrations between 1.2 and 1.5 $\mu g \cdot m^{-3}$, the proportion of students that scored higher than 700 on the SAT, or the proportion of Least Weasels that are shorter than 150 mm. Before considering these more realistic situations, we explore calculations for the generic variable $X$ shown in \figref{fig:NormDistShade}.

Let's consider finding the proportion of individuals in a \textit{sample} with values between 0 and 2. A histogram can be used to answer this question because it is about the individuals in a sample (\figref{fig:NormDistShade}-Left). In this case, the proportion of individuals with values between 0 and 2 is computed by dividing the number of individuals in the red shaded bars by the total number of individuals in the histogram. The analogous computation on the superimposed smooth curve is to find the area under the curve between 0 and 2 (\figref{fig:NormDistShade}-Right). The area under the curve is a ``proportion of the total'' because, as stated above, the area under the entire curve is equal to 1. The actual calculations on the normal curve are shown in the following sections. However, at this point, note that the calculation of an area on a normal curve is analogous to summing the number of individuals in the appropriate classes of the histogram and dividing by $n$.

\begin{knitrout}
\definecolor{shadecolor}{rgb}{0.922, 0.922, 0.922}\color{fgcolor}\begin{figure}[hbtp]

{\centering \includegraphics[width=.8\linewidth]{Figs/NormDistShade-1} 

}

\caption[Depiction of finding the proportion of individuals between 0 and 2 on a histogram (\textbf{Left}) and on a standard normal distribution (\textbf{Right})]{Depiction of finding the proportion of individuals between 0 and 2 on a histogram (\textbf{Left}) and on a standard normal distribution (\textbf{Right}).}\label{fig:NormDistShade}
\end{figure}


\end{knitrout}

\warn{The proportion of individuals between two values of a variable that is normally distributed is the area under the normal distribution between those two values.}

The 68-95-99.7 (or Empirical) Rule states that 68\% of individuals that follow a normal distribution have values between $\mu-1\sigma$ and $\mu+1\sigma$, 95\% have values between $\mu-2\sigma$ and $\mu+2\sigma$, and 99.7\% have values between $\mu-3\sigma$ and $\mu+3\sigma$ \figrefp{fig:NormEmpiricalRule}.

\begin{knitrout}
\definecolor{shadecolor}{rgb}{0.922, 0.922, 0.922}\color{fgcolor}\begin{figure}[hbtp]

{\centering \includegraphics[width=.5\linewidth]{Figs/NormEmpiricalRule-1} 

}

\caption[Depiction of the 68-95-99.7 (or Empirical) Rule on a normal distribution]{Depiction of the 68-95-99.7 (or Empirical) Rule on a normal distribution.}\label{fig:NormEmpiricalRule}
\end{figure}


\end{knitrout}

\vspace{12pt} % needed because spaced is gobbled up by LaTeX
The 68-95-99.7 Rule is true no matter what $\mu$ and $\sigma$ are as long as the distribution is normal. For example, if $A\sim N(3,1)$, then 68\% of the individuals will fall between 2 (i.e., 3-1*1) and 4 (i.e., 3+1*1) and 99.7\% will fall between 0 (i.e., 3-3*1) and 6 (i.e., 3+3*1). Alternatively, if $B\sim N(9,3)$, then 68\% of the individuals will fall between 6 (i.e., 9-1*3) and 12 (i.e., 9+1*3) and 95\% will be between 3 (i.e., 9-2*3) and 15 (i.e., 9+2*3). Similar calculations can be made for any normal distribution.

The 68-95-99.7 Rule is used to find areas under the normal curve as long as the value of interest is an \textbf{integer} number of standard deviations away from the mean. For example, the proportion of individuals that have a value of A greater than 5 \figrefp{fig:NormEmpiricalRuleCalc} is found by first realizing that 95\% of the individuals on this distribution fall between 1 and 5 (i.e., $\pm2\sigma$ from $\mu$). By subtraction this means that 5\% of the individuals must be less than 1 \textbf{AND} greater than 5. Finally, because normal distributions are symmetric, the same percentage of individuals must be less than 1 as are greater than 5. Thus, half of 5\%, or 2.5\%, of the individuals have a value of A greater than 5.

\begin{knitrout}
\definecolor{shadecolor}{rgb}{0.922, 0.922, 0.922}\color{fgcolor}\begin{figure}[hbtp]

{\centering \includegraphics[width=.4\linewidth]{Figs/NormEmpiricalRuleCalc-1} 

}

\caption[The N(3,1) distribution depicting how the 68-95-99.7 Rule is used to compute the percentage of individuals with values greater than 5]{The N(3,1) distribution depicting how the 68-95-99.7 Rule is used to compute the percentage of individuals with values greater than 5.}\label{fig:NormEmpiricalRuleCalc}
\end{figure}


\end{knitrout}

\warn{The 68-95-99.7 Rule can only be used for questions involving \textbf{integer} standard deviations away from the mean.}


\section[Forward Calculations]{More Complex Areas (Forward Calculations)}
Areas under the curve relative to non-integer numbers of standard deviations away from the mean can only be found with the help of special tables or computer software. In this course, we will use R.

The area under a normal curve relative to a particular value is computed in R with \R{distrib()}. This function requires the \textit{particular value} as the first argument and the mean and standard deviation of the normal distribution in the \R{mean=} and \R{sd=} arguments, respectively. The \R{distrib()} function defaults to finding the area under the curve to the \textbf{left of} the particular value, but it can find the area under the curve to the right of the particular value by including \R{lower.tail=FALSE}.

For example, suppose that the heights of a population of students is known to be $H\sim N(66,3)$. The proportion of students in this population that have a height less than 71 inches is computed below. Thus, approximately 95.2\% of students in this population have a height less than 71 inches \figrefp{fig:NormZCalc1}.
\begin{knitrout}
\definecolor{shadecolor}{rgb}{0.922, 0.922, 0.922}\color{fgcolor}\begin{kframe}
\begin{verbatim}
> ( distrib(71,mean=66,sd=3) )
[1] 0.9522096
\end{verbatim}
\end{kframe}\begin{figure}[hbtp]

{\centering \includegraphics[width=.4\linewidth]{Figs/NormZCalc1-1} 

}

\caption[Calculation of the proportion of individuals on a $N(66,3)$ with a value less than 71]{Calculation of the proportion of individuals on a $N(66,3)$ with a value less than 71.}\label{fig:NormZCalc1}
\end{figure}


\end{knitrout}

The proportion of students in this population that have a height \textit{greater} than 68 inches is computed below (note use of \R{lower.tail=FALSE}). Thus, approximately 25.2\% of students in this population have a height greater than 68 inches \figrefp{fig:NormZCalc2}.
\begin{knitrout}
\definecolor{shadecolor}{rgb}{0.922, 0.922, 0.922}\color{fgcolor}\begin{kframe}
\begin{verbatim}
> ( distrib(68,mean=66,sd=3,lower.tail=FALSE) )
[1] 0.2524925
\end{verbatim}
\end{kframe}\begin{figure}[hbtp]

{\centering \includegraphics[width=.4\linewidth]{Figs/NormZCalc2-1} 

}

\caption[Calculation of the proportion of individuals on a $N(66,3)$ with a value greater than 68]{Calculation of the proportion of individuals on a $N(66,3)$ with a value greater than 68.}\label{fig:NormZCalc2}
\end{figure}


\end{knitrout}

Finding the area between two particular values is a bit more work. To answer ``between''-type questions, the area less than the smaller of the two values is subtracted from the area less than the larger of the two values. This is illustrated by noting that two values split the area under the normal curve into three parts -- A, B, and C in \figref{fig:NormDistBetween}. The area between the two values is B. The area to the left of the larger value corresponds to the area A+B. The area to the left of the smaller value corresponds to the area A. Thus, subtracting the latter from the former leaves the ``in-between'' area B (i.e., (A+B)-A = B).

\begin{knitrout}
\definecolor{shadecolor}{rgb}{0.922, 0.922, 0.922}\color{fgcolor}\begin{figure}[hbtp]

{\centering \includegraphics[width=.4\linewidth]{Figs/NormDistBetween-1} 

}

\caption[Schematic representation of how to find the area between two $Z$ values]{Schematic representation of how to find the area between two $Z$ values.}\label{fig:NormDistBetween}
\end{figure}


\end{knitrout}
\vspace{12pt} % because the spacing is gobbled by the R code.

For example, the area between 62 and 70 inches of height is found below. Thus, 81.8\% of students in this population have a height between 62 and 70 inches.

\begin{knitrout}
\definecolor{shadecolor}{rgb}{0.922, 0.922, 0.922}\color{fgcolor}\begin{kframe}
\begin{verbatim}
> ( AB <- distrib(70,mean=66,sd=3) )  # left-of 70
[1] 0.9087888
> ( A <- distrib(62,mean=66,sd=3) )   # left-of 62
[1] 0.09121122
> AB-A                                # between 62 and 70
[1] 0.8175776
\end{verbatim}
\end{kframe}
\end{knitrout}

\warn{The area between two values is found by subtracting the area less than the smaller value from the area less than the larger value.}



\section[Reverse Calculations]{Values from Areas (Reverse Calculations)}
Another important calculation with normal distributions is finding the value or values of $X$ with a given proportion of individuals less than, greater than, or between. For example, it may be necessary to find the test score such that 90\% (or 0.90 as a proportion) of the students scored lower. In contrast to the calculations in the previous section (where the value of $X$ was given and a proportion of individuals was asked for), the calculations in this section give a proportion and ask for a value of $X$. These types of questions are called \textbf{``reverse'' normal distribution questions} to contrast them with questions from the previous section.

Reverse questions are also answered with \R{distrib()}, though the first argument is now the given proportion (or area) of interest. The calculation is treated as a ``reverse'' question when \R{type="q"} is given to \R{distrib()}.\footnote{``q'' stands for quantile.}  For example, the height that has 20\% of all students shorter is  63.5 inches, as computed below \figrefp{fig:NormZCalc4}.
\begin{knitrout}
\definecolor{shadecolor}{rgb}{0.922, 0.922, 0.922}\color{fgcolor}\begin{kframe}
\begin{verbatim}
> ( distrib(0.20,mean=66,sd=3,type="q") )
[1] 63.47514
\end{verbatim}
\end{kframe}\begin{figure}[hbtp]

{\centering \includegraphics[width=.4\linewidth]{Figs/NormZCalc4-1} 

}

\caption[Calculation of the height with 20\% of all students shorter]{Calculation of the height with 20\% of all students shorter.}\label{fig:NormZCalc4}
\end{figure}


\end{knitrout}

``Greater than'' reverse questions are computed by including \R{lower.tail=FALSE}. For example, 10\% of the population of students is taller than 69.8 inches, as computed below \figrefp{fig:NormZCalc5}.
\begin{knitrout}
\definecolor{shadecolor}{rgb}{0.922, 0.922, 0.922}\color{fgcolor}\begin{kframe}
\begin{verbatim}
> ( distrib(0.10,mean=66,sd=3,type="q",lower.tail=FALSE) )
[1] 69.84465
\end{verbatim}
\end{kframe}\begin{figure}[hbtp]

{\centering \includegraphics[width=.4\linewidth]{Figs/NormZCalc5-1} 

}

\caption[Calculation of the height with 10\% of all students taller]{Calculation of the height with 10\% of all students taller.}\label{fig:NormZCalc5}
\end{figure}


\end{knitrout}

``Between'' questions can only be easily handled if the question is looking for endpoint values that are symmetric about $\mu$. In other words, the question must ask for the two values that contain the ``most common'' proportion of individuals. For example, suppose that you were asked to find the most common 80\% of heights. This type of question is handled by converting this ``symmetric between'' question into two ``less than'' questions. For example, in \figref{fig:NormRevBetween} the area D is the symmetric area of interest. If D is 0.80, then C+E must be 0.20.\footnote{Because all three areas must sum to 1.}  Because D is symmetric about $mu$, C and E must both equal 0.10. Thus, the lower bound on D is the value that has 10\% of all values smaller. Similarly, because the combined area of C and D is 0.90, the upper bound on D is the value that has 90\% of all values smaller. This question has now been converted from a ``symmetric between'' to two ``less than'' questions that can be answered exactly as shown above. For example, the two heights that have a symmetric 80\% of individuals between them are 62.2 and 69.8 as computed below.
\begin{knitrout}
\definecolor{shadecolor}{rgb}{0.922, 0.922, 0.922}\color{fgcolor}\begin{kframe}
\begin{verbatim}
> ( distrib(0.10,mean=66,sd=3,type="q") )
[1] 62.15535
> ( distrib(0.90,mean=66,sd=3,type="q") )
[1] 69.84465
\end{verbatim}
\end{kframe}
\end{knitrout}

\begin{knitrout}
\definecolor{shadecolor}{rgb}{0.922, 0.922, 0.922}\color{fgcolor}\begin{figure}[hbtp]

{\centering \includegraphics[width=.4\linewidth]{Figs/NormRevBetween-1} 

}

\caption[Depiction of areas in a reverse between type normal distribution question]{Depiction of areas in a reverse between type normal distribution question.}\label{fig:NormRevBetween}
\end{figure}


\end{knitrout}


\section{Distinguish Calculation Types}
It is critical to be able to distinguish between the two main types of calculations made from normal distributions. The first type of calculation is a ``forward'' calculation where the area or proportion of individuals relative to a value of the variable must be found. The second type of calculation is a ``reverse'' calculation where the value of the variable relative to a particular area is calculated.

Distinguishing between these two types of calculations is a matter of deciding if (i) the value of the variable is given and the proportion (or area) is to be found or (ii) if the proportion (or area) is given and the value of the variable is to be found. Therefore, distinguishing between the calculation types is as simple as identifying what is given (or known) and what must be found. If the value of the variable is given but not the proportion or area, then a forward calculation is used. If the area or proportion is given, then a reverse calculation to find the value of the variable is used.


\section{Standardization and Z-Scores}\label{sect:Standardizing}
An individual that is 59 inches tall is 7 inches shorter than average if heights are $N(66,3)$. Is this a large or a small difference?  Alternatively, this same individual is $\frac{-7}{3}$ = $-2.33$ standard deviations below the mean. Thus, a height of 59 inches is relatively rare in this population because few individuals are more than two standard deviations away from the mean.\footnote{From the 68-95-99.7\% Rule.} As seen here, the relative magnitude that an individual differs from the mean is better expressed as the number of standard deviations that the individual is away from the mean.

Values are ``standardized'' by changing the original scale (inches in this example) to one that counts the number of standard deviations (i.e., $\sigma$) that the value is away from the mean (i.e., $\mu$). For example, with the height variable above, 69 inches is one standard deviation above the mean, which corresponds to +1 on the standardized scale. Similarly, 60 inches is two standard deviations below the mean, which corresponds to -2 on the standardized scale. Finally, 67.5 inches on the original scale is one half standard deviation above the mean or +0.5 on the standardized scale.

The process of computing the number of standard deviations that an individual is away from the mean is called \textbf{standardizing}. Standardizing is accomplished with
\begin{equation}
  \label{eqn:Zgeneral}
    Z = \frac{\text{``value''}-\text{``center''}}{\text{``dispersion''}}
\end{equation}
or, more specifically,
\begin{equation}
  \label{eqn:Zspecific}
    Z = \frac{x-\mu}{\sigma}
\end{equation}
For example, the standardized value of an individual with a height of 59 inches is $z=\frac{59-66}{3}=-2.33$. Thus, this individual's height is 2.33 standard deviations below the average height in the population.

Standardized values ($Z$) follow a $N(0,1)$. Thus, $N(0,1)$ is called the ``standard normal distribution.''  The relationship between $X$ and $Z$ is one-to-one meaning that each value of $X$ converts to one and only one value of $Z$. This means that the area to the left of $X$ on a $N(\mu,\sigma)$ is the same as the area to the left of $Z$ on a $N(0,1)$. This one-to-one relationship is illustrated in \figref{fig:NormStandardizingEx} using the individual with a height of 59 inches and $Z=-2.33$.

\vspace{-6pt}
\warn{The standardized scale (i.e., z-scores) represents the number of standard deviations that a value is from the mean.}

\begin{knitrout}
\definecolor{shadecolor}{rgb}{0.922, 0.922, 0.922}\color{fgcolor}\begin{figure}[hbtp]

{\centering \includegraphics[width=.4\linewidth]{Figs/NormStandardizingEx-1} 
\includegraphics[width=.4\linewidth]{Figs/NormStandardizingEx-2} 

}

\caption{Plots depicting the area to the left of 59 on a $N(66,3)$ (\textbf{Left}) and the area to the right of the corresponding Z-score of $Z=-2.33$ on a $N(0,1)$ (\textbf{Right}). Not that the x-axis scales are different.}\label{fig:NormStandardizingEx}
\end{figure}


\end{knitrout}



\chapter{Bivariate EDA - Quantitative} \label{chap:BivEDAQuant}

\vspace{-30pt}
\minitoc
\vspace{18pt}

\lettrine{B}{ivariate data occurs when two} variables are measured on the same individuals. For example, you may measure (i) the height and weight of students in class, (ii) depth and area of a lake, (iii) gender and age of welfare recipients, or (iv) number of mice and biomass of legumes in fields. This module is focused on describing the bivariate relationship between two quantitative variables. Bivariate relationships between two categorical variables is described in \modref{chap:BivEDACat}.

Data on the \var{weight} (lbs) and highway miles per gallon (\var{HMPG}) for 93 cars from the 1993 model year are used as an example throughout this module. Ultimately, the relationship between highway MPG and the weight of a car is described. These data are read from \href{https://raw.githubusercontent.com/droglenc/NCData/master/93cars.csv}{93cars.csv} into R and several observations of \var{HMPG} and \var{weight} are shown below.\footnote{The vector in the second argument to \R{headtail()} is used to show only the two variables of interest.}

\begin{knitrout}
\definecolor{shadecolor}{rgb}{0.922, 0.922, 0.922}\color{fgcolor}\begin{kframe}
\begin{verbatim}
> cars93 <- read.csv("data/93cars.csv")
\end{verbatim}
\end{kframe}
\end{knitrout}
\begin{knitrout}
\definecolor{shadecolor}{rgb}{0.922, 0.922, 0.922}\color{fgcolor}\begin{kframe}
\begin{verbatim}
> headtail(cars93,which=c("HMPG","Weight"))
   HMPG Weight
1    31   2705
2    25   3560
3    26   3375
91   25   2810
92   28   2985
93   28   3245
\end{verbatim}
\end{kframe}
\end{knitrout}



\section[Response and Explanatory] {Response and Explanatory Variables} \label{sect:RespExplan1}
\vspace{-3pt}
The \textbf{response variable} is the variable that one is interested in explaining something (i.e., variability) or in making future predictions about. The \textbf{explanatory variable} is the variable that may help explain or allow one to predict the response variable. In general, the response variable is thought to depend on the explanatory variable. Thus, the response variable is often called the \textbf{dependent variable}, whereas the explanatory variable is often called the \textbf{independent variable}.

One may identify the response variable by determining which of the two variables depends on the other. For example, in the car data, highway MPG is the response variable because gas mileage is most likely affected by the weight of the car (e.g., hypothesize that heavier cars get worse gas mileage), rather than vice versa.

In some situations it is not obvious which variable is the response. For example, does the number of mice in the field depend on the number of legumes (lots of feed=lots of mice) or the other way around (lots of mice=not much food left)? Similarly, does area depend on depth or does depth depend on area of the lake? In these situations, the context of the research question is needed to identify the response variable. For example, if the researcher hypothesized that number of mice will be greater if there is more legumes, then number of mice is the response variable. In many cases, the more difficult variable to measure will likely be the response variable. For example, researchers likely wish to predict area of a lake (hard to measure) from depth of the lake (easy to measure).

\vspace{-9pt}
\warn{Which variable is the response may depend on the context of the research question.}


\vspace{-12pt}
\section{Summaries}
\vspace{-6pt}
\subsection{Scatterplots} \label{sect:ScatterplotsR}
\vspace{-3pt}
A scatterplot is a graph where each point simultaneously represents the values of both the quantitative response and quantitative explanatory variable. The value of the explanatory variable gives the x-coordinate and the value of the response variable gives the y-coordinate of the point plotted for an individual. For example, the first individual in the cars data is plotted at x (\var{Weight}) = 2705 and y (\var{HMPG}) = 31, whereas the second individual is at x = 3560 and y = 25 \figrefp{fig:carscat2}.


\begin{knitrout}
\definecolor{shadecolor}{rgb}{0.922, 0.922, 0.922}\color{fgcolor}\begin{figure}[hbtp]

{\centering \includegraphics[width=.4\linewidth]{Figs/carscat2-1} 

}

\caption[Scatterplot between the highway MPG and weight of cars manufactured in 1993]{Scatterplot between the highway MPG and weight of cars manufactured in 1993. For reference to the main text, the first individual is red and the second individual is green.}\label{fig:carscat2}
\end{figure}


\end{knitrout}

Scatterplots are constructed in R with \R{plot()} with a formula of the form \R{Y\TILDE X}, where \R{Y} and \R{X} are variables to be plotted on the y- and x-axes, as the first argument, and the corresponding data.frame in \R{data=}. The x- and y-axis labels may be modified with \R{xlab=} and \R{ylab=}. The character plotted at each point can be changed with \R{pch=},\footnote{This argument is short for ``plotting character''.} which defaults to 1 or an open-circle \figrefp{fig:Rpch}. The scatterplot, excluding the two highlighted points, of highway MPG versus car weight \figrefp{fig:carscat2} was created with the code below.
\begin{knitrout}
\definecolor{shadecolor}{rgb}{0.922, 0.922, 0.922}\color{fgcolor}\begin{kframe}
\begin{verbatim}
> plot(HMPG~Weight,data=cars93,xlab="Weight (lbs)",ylab="Highway MPG",pch=16)
\end{verbatim}
\end{kframe}
\end{knitrout}

\begin{knitrout}
\definecolor{shadecolor}{rgb}{0.922, 0.922, 0.922}\color{fgcolor}\begin{figure}[hbtp]

{\centering \includegraphics[width=.4\linewidth]{Figs/Rpch-1} 

}

\caption[Plotting characters available in R and their numerical codes]{Plotting characters available in R and their numerical codes. Note that for values of 21-25 that \R{bg='gray70'} is used to provide the background color.}\label{fig:Rpch}
\end{figure}


\end{knitrout}


\subsection{Correlation Coefficient}\label{sect:corr}
The sample correlation coefficient, abbreviated as $r$, is calculated with
\begin{equation} \label{eqn:Correlation}
  r = \frac{\Sum_{i=1}^{n}\left[\left(\frac{x_{i}-\bar{x}}{s_{x}}\right)\left(\frac{y_{i}-\bar{y}}{s_{y}}\right)\right]}{n-1}
\end{equation}
where $s_{x}$ and $s_{y}$ are the sample standard deviations for the explanatory and response variables, respectively.\footnote{See \sectref{sect:StdDev} for a review of standard deviations.} The formulae in the two sets of parentheses in the numerator are standardized values;\footnote{See \sectref{sect:Standardizing} for a review of standardized values.} thus, the value in each parenthesis is called the standardized x or standardized y, respectively. Using this terminology, Equation \eqref{eqn:Correlation} reduces to these steps:
\begin{Enumerate}
  \item For each individual, standardize x and standardize y.
  \item For each individual, find the product of the standardized x and standardized y.
  \item Sum all of the products from step 2.
  \item Divide the sum from step 3 by n-1.
\end{Enumerate}

The table below illustrates these calculations for the first five individuals in the cars data.\footnote{The five cars are treated as if they are the entire sample.} Note that the ``i'' column is an index for each individual, the $x_{i}$ and $y_{i}$ columns are the observed values of the two variables for individual $i$, $\bar{x}$ was computed by dividing the sum of the $x_{i}$ column by $n$, $s_{x}$ was computed by dividing the sum of the $(x_{i}-\bar{x})^{2}$ column by $n-1$ and taking the square root, and the ``std x'' column are the standardized x values found by dividing the values in the $x_{i}-\bar{x}$ column by $s_{x}$. Similar calculations were made for the y variable. The final correlation coefficient is the sum of the last column divided by $n-1$. Thus, the correlation between car weight and highway mpg for these five cars is -0.54.

\begin{center}
  \begin{tabular}{cccccccccc}
\hline\hline
 & HMPG & Weight & & & & & & & \\
i & $y_{i}$ & $x_{i}$ & $y_{i}-\bar{y}$ & $x_{i}-\bar{x}$ & $(y_{i}-\bar{y})^{2}$ & $(x_{i}-\bar{x})^{2}$ & std. y & std. x & (std. y)(std. x) \\
\hline
1 & 31 & 2705 &  3.4 & -632 & 11.56 & 399424 &  1.26 & -1.71 & -2.15 \\
2 & 25 & 3560 & -2.6 &  223 &  6.76 &  49729 & -0.96 &  0.6  & -0.58 \\
3 & 26 & 3375 & -1.6 &   38 &  2.56 &   1444 & -0.59 &  0.1  & -0.06 \\
4 & 26 & 3405 & -1.6 &   68 &  2.56 &   4624 & -0.59 &  0.18 & -0.11 \\
5 & 30 & 3640 &  2.4 &  303 &  5.76 &  91809 &  0.89 &  0.82 &  0.73 \\
\hline
sum & 138 & 16685 & 0 & 0 & 29.2 & 547030 & 0 & 0 &  -2.17 \\
\hline\hline
  \end{tabular}
\end{center}

The meaning and interpretation of $r$ is discussed in more detail in \sectref{sect:BivEDAItems}.

The correlation coefficient ($r$) between two quantitative variables is computed with \R{corr()} using a formula of the form \R{Y\TILDE X} or \R{\TILDE Y+X}, where \R{Y} and \R{X} are the names of quantitative variables, as the first argument and the corresponding data.frame in \R{data=}. For example, the correlation coefficient between highway MPG and weight for all cars in the car data is -0.81.
\begin{knitrout}
\definecolor{shadecolor}{rgb}{0.922, 0.922, 0.922}\color{fgcolor}\begin{kframe}
\begin{verbatim}
> corr(HMPG~Weight,data=cars93)
[1] -0.8106581
> corr(~HMPG+Weight,data=cars93)  # alternative form
[1] -0.8106581
\end{verbatim}
\end{kframe}
\end{knitrout}


\subsection{Pairs of Multiple Variables}
\vspace{-3pt}

Correlation coefficients can be computed or scatterplots can be constructed simultaneously for all pairs of many quantitative variables. A matrix of correlation coefficients is constructed with \R{corr()} as above using a formula of the form \R{\TILDE X1+X2+X3} (and so on), where the \R{X1}, \R{X2}, etc. are all quantitative variables to be used. In some instances, the data.frame may contain missing values (i.e., data that were not recorded). The individuals with missing data are efficiently removed from the correlation matrix with \R{use="pairwise.complete.obs"} in \R{corr()}.\footnote{Missing data are automatically removed from the scatterplots.} The number of digits reported in the correlation matrix is controlled with \R{digits=}. For example, the correlation between highway MPG and size of the fuel tank is -0.786, whereas the correlation between length and weight of the car is 0.806.
\begin{knitrout}
\definecolor{shadecolor}{rgb}{0.922, 0.922, 0.922}\color{fgcolor}\begin{kframe}
\begin{verbatim}
> corr(~HMPG+FuelTank+Length+Weight,data=cars93,use="pairwise.complete.obs",digits=3)
           HMPG FuelTank Length Weight
HMPG      1.000   -0.786 -0.543 -0.811
FuelTank -0.786    1.000  0.690  0.894
Length   -0.543    0.690  1.000  0.806
Weight   -0.811    0.894  0.806  1.000
\end{verbatim}
\end{kframe}
\end{knitrout}

A matrix of scatterplots is constructed with \R{pairs()} using the same formula notation as in \R{corr()}. The plotting character can be changed, as with \R{plot()}, with \R{pch=}. Each subplot in the resulting scatterplot matrix \figrefp{fig:Scatplot4} is a scatterplot with the variable listed in the same column on the x-axis and the variable listed in the same row on the y-axis. For example, the scatterplot in the upper-right corner of \figref{fig:Scatplot4} has highway MPG on the y-axis and car weight on the x-axis.

\begin{knitrout}
\definecolor{shadecolor}{rgb}{0.922, 0.922, 0.922}\color{fgcolor}\begin{kframe}
\begin{verbatim}
> pairs(~HMPG+FuelTank+Length+Weight,data=cars93,pch=21,bg="gray70")
\end{verbatim}
\end{kframe}\begin{figure}[hbtp]

{\centering \includegraphics[width=.7\linewidth]{Figs/Scatplot4-1} 

}

\caption[Scatterplot matrix of the highway MPG, fuel tank size, length, and weight of cars]{Scatterplot matrix of the highway MPG, fuel tank size, length, and weight of cars.}\label{fig:Scatplot4}
\end{figure}


\end{knitrout}


\section{Items to Describe} \label{sect:BivEDAItems}
Four characteristics should be described for a bivariate EDA with two quantitative variables:
\vspace{-8pt}
\begin{Enumerate}
  \item \textbf{form} of the relationship,
  \item presence (or absence) of \textbf{outliers}, and
  \item \textbf{association} or \textbf{direction} of the relationship,
  \item \textbf{strength} of the relationship.
\end{Enumerate}
\vspace{-8pt}
All four of these items can be described from a scatterplot. However, for certain relationships (discussed below), strength is best described from the correlation coefficient.

\subsection{Form and Outliers}
The form of a relationship is determined by whether the ``cloud'' of points on a scatterplot forms a line or some sort of curve \figrefp{fig:corrassn}. For the purposes of this introductory course, if the ``cloud'' appears linear then the form will said to be linear, whereas if the ``cloud'' is curved then the form will be nonlinear. Scatterplots should be considered \textbf{linear} unless there is an OBVIOUS curvature in the points.

\begin{knitrout}
\definecolor{shadecolor}{rgb}{0.922, 0.922, 0.922}\color{fgcolor}\begin{figure}[hbtp]

{\centering \includegraphics[width=.3\linewidth]{Figs/forms-1} 
\includegraphics[width=.3\linewidth]{Figs/forms-2} 
\includegraphics[width=.3\linewidth]{Figs/forms-3} 

}

\caption[Depictions of two linear (Left and Center) and one nonlinear (Right) relationship]{Depictions of two linear (Left and Center) and one nonlinear (Right) relationship.}\label{fig:forms}
\end{figure}


\end{knitrout}

\vspace{12pt} % because it got gobbled up
An outlier is a point that is far removed from the main cluster of points. Keep in mind (as always) that just because a point is an outlier doesn't mean it is wrong.

\subsection{Association or Direction}
A positive association is when the scatterplot resembles an increasing function (i.e., increases from lower-left to upper-right; \figref{fig:corrassn}-Left). For a positive association, most of the individuals are above average or below average for both of the variables. A negative association is when the scatterplot looks like a decreasing function (i.e., decreases from upper-left to lower-right; \figref{fig:corrassn}-Right). For a negative association, most of the individuals are above average for one variable and below average for the other variable. No association is when the scatterplot looks like a ``shotgun blast'' of points (\figref{fig:corrassn}-Center). For no association, there is no tendency for individuals to be above or below average for one variable and above or below average for the other.

\begin{knitrout}
\definecolor{shadecolor}{rgb}{0.922, 0.922, 0.922}\color{fgcolor}\begin{figure}[hbtp]

{\centering \includegraphics[width=.3\linewidth]{Figs/corrassn-1} 
\includegraphics[width=.3\linewidth]{Figs/corrassn-2} 
\includegraphics[width=.3\linewidth]{Figs/corrassn-3} 

}

\caption[Depiction of three types of association present in scatterplots]{Depiction of three types of association present in scatterplots. Dashed vertical lines are at the means of each variable.}\label{fig:corrassn}
\end{figure}


\end{knitrout}

\subsection{Strength (and Association, Again)} \label{sect:CorrStrength}
Strength is a summary of how closely the points cluster about the general form of the relationship. For example, if a linear form exists, then strength is how closely the points cluster around the line. Strength is difficult to define from a scatterplot because it is a relative term. However, the correlation coefficient ($r$; \sectref{sect:corr}) is a measure of strength (and association) between two variables, \textit{if the form is linear}.

The sign of $r$ indicates the association between the two variables. A positive $r$ means a positive association and a negative $r$ means a negative association. The absolute value of $r$ (i.e., the value ignoring the sign) is an indicator of strength of relationship. Absolute values nearer 1 are stronger relationships.

To better understand how $r$ is a measure of association and strength, reconsider the steps in calculating $r$ from \sectref{sect:corr}. The scatterplots in \figref{fig:corrdefn1} represent a positive (Left) and negative (Right) association. These scatterplots have dashed lines at the mean of both the x- and y-axis variables. Because the mean is subtracted from observed values when standardizing, points that fall above the mean will have positive standardized values and points that fall below the mean will have negative standardized values. The sign for the standardized values are depicted along the axes.

\begin{knitrout}
\definecolor{shadecolor}{rgb}{0.922, 0.922, 0.922}\color{fgcolor}\begin{figure}[hbtp]

{\centering \includegraphics[width=.65\linewidth]{Figs/corrdefn1-1} 

}

\caption[Scatterplot with mean lines superimposed and the signs of standardized values for both x and y shown for a positive (\textbf{Left}) and negative (\textbf{Right}) association]{Scatterplot with mean lines superimposed and the signs of standardized values for both x and y shown for a positive (\textbf{Left}) and negative (\textbf{Right}) association. Blue points have a positive product of standardized values, whereas red points have a negative product of standardized values.}\label{fig:corrdefn1}
\end{figure}


\end{knitrout}

\vspace{-6pt}
Now consider the product of standardized x's and y's in each quadrant of the scatterplots in \figref{fig:corrdefn1}. The product of standardized values is positive (blue points) in the quadrant where both standardized values are above average (i.e., both positive signs) and both are below average. The product of standardized values is negative (red points) in the other two quadrants.

Thus, for a positive association (\figref{fig:corrdefn1}-Left) the numerator of the correlation coefficient is positive because it is the sum of many positive (blue points) and few negative (red points) products of standardized values. The denominator (recall that it is $n-1$) is always positive. Therefore, $r$ for a positive association is positive. Conversely, for a negative association (\figref{fig:corrdefn1}-Right) the numerator of the correlation coefficient is negative because it is the sum of few positive (blue points) and many negative (red points) products of standardized values. Therefore, $r$ for a negative association is negative.

Correlations range from -1 to 1. Absolute values of $r$ equal to 1 indicate a perfect association (i.e., all points exactly on a line). A correlation of 0 indicates no association. Thus, absolute values of $r$ near 1 indicate strong relationships and those near 0 are weak. How strength and association of the relationship changes along the range of $r$ values is illustrated in \figref{fig:corrstrength2}. Categorizations in \tabref{tab:StrengthCriteria} can be used as a guideline for describing the strength of relationship between two variables.

\begin{knitrout}
\definecolor{shadecolor}{rgb}{0.922, 0.922, 0.922}\color{fgcolor}\begin{figure}[h]

{\centering \includegraphics[width=.95\linewidth]{Figs/corrstrength2-1} 

}

\caption[Scatterplots along the continuum of $r$ values]{Scatterplots along the continuum of $r$ values.}\label{fig:corrstrength2}
\end{figure}


\end{knitrout}

\begin{table}[htbp]
  \caption{Classifications of strength of relationship for absolute values of $r$ by type of study.}
  \label{tab:StrengthCriteria}
  \centering
  \begin{tabular}{c|ccc}
\hline\hline
\widen{0}{5}{Strength of} & Uncontrolled/ & Controlled/ \\
\widen{-2}{0}{Relationship} & Observational & Experimental \\
\hline
\widen{0}{4}{Strong} & $>0.8$ & $>0.95$ \\
\widen{0}{4}{Moderate} & $>0.6$ & $>0.9$ \\
\widen{-1}{5}{Weak} & $>0.4$ & $>0.8$ \\
\hline\hline
  \end{tabular}
\vspace{36pt} % to get some space before the next section
\end{table}


\newpage
\section{Example Interpretations}
When performing a bivariate EDA for two quantitative variables, the form, presence (or absence) of outliers, association, and strength should be specifically addressed. In addition, you should state how you assessed strength. Specifically, you should use $r$ to assess strength (see \sectref{sect:CorrStrength}) \textbf{IF} the relationship is linear without any outliers. However, if the relationship is nonlinear, has outliers, or both, then strength should be subjectively assessed from the scatterplot.

Two other points to consider when performing a bivariate EDA with quantitative variables. First, if outliers are present, do not let them completely influence your conclusions about form, association, and strength. In other words, assess these items ignoring the outlier(s). If you have raw data and the form excluding the outlier is linear, then compute $r$ with the outlier eliminated from the data. Second, the form of weak relationships is difficult to describe because, by definition, there is very little clustering to a form. As a rule-of-thumb, if the scatterplot is not obviously curved, then it is described as linear by default.

\warn{Outliers should not influence the descriptions of association, strength, and form.}

\vspace{-12pt}
\warn{The form is linear unless there is an OBVIOUS curvature.}

\vspace{12pt}
\subsection*{Highway MPG and Weight}
\textit{The following overall bivariate summary for the relationship between highway MPG and weight is made using the calculations from the previous sections.}

The relationship between highway MPG and weight of cars \figrefp{fig:carscat2} appears to be primarily linear (although I see a very slight concavity), negative, and moderately strong with a correlation of -0.81. The three points at (2400,46), (2500,27), and (1800,33) might be considered SLIGHT outliers (these are not far enough removed for me to consider them outliers, but some people may). The correlation coefficient was used to assess strength because I deemed the relationship to be linear without any outliers.

\newpage
\subsection*{State Energy Usage}
\begin{quote}
\textit{A 2001 report from the \href{http://www.eia.doe.gov/}{Energy Information Administration} of the Department of Energy details the total consumption of a variety of energy sources by state in 2001. Construct a proper EDA for the relationship between total petroleum and coal consumption (in trillions of BTU).}
\end{quote}

The relationship between total petroleum and coal consumption is generally linear, with two outliers at total petroleum levels greater than 3000 trillions of BTU, positive, and weak (\figref{fig:scatNRG1}-Left). I did not use the correlation coefficient because of the outliers. If the two outliers (Texas and California) are removed then the relationship is linear, with no additional outliers, positive, and weak ($r=$0.53) (\figref{fig:scatNRG1}-Right).

\begin{knitrout}
\definecolor{shadecolor}{rgb}{0.922, 0.922, 0.922}\color{fgcolor}\begin{figure}[hbtp]

{\centering \includegraphics[width=.4\linewidth]{Figs/scatNRG1-1} 
\includegraphics[width=.4\linewidth]{Figs/scatNRG1-2} 

}

\caption[Scatterplot of the total consumption of petroleum versus the consumption of coal (in trillions of BTU) by all 50 states and the District of Columbia]{Scatterplot of the total consumption of petroleum versus the consumption of coal (in trillions of BTU) by all 50 states and the District of Columbia. The points shown in the left with total petroleum values greater than 3000 trillion BTU are deleted in the right plot.}\label{fig:scatNRG1}
\end{figure}


\end{knitrout}

\subsubsection*{R Appendix}
\begin{knitrout}
\definecolor{shadecolor}{rgb}{0.922, 0.922, 0.922}\color{fgcolor}\begin{kframe}
\begin{verbatim}
NRG <- read.csv("data/NRG_Consump_2001.csv")
NRG1 <- NRG[-c(5,44),]
plot(TotalPet~Coal,data=NRG,pch=21,bg="gray70",xlab="Coal Consumption (trillion BTU)",
     ylab="Total Petroleum (trillion BTU)")
plot(TotalPet~Coal,data=NRG1,pch=21,bg="gray70",xlab="Coal Consumption (trillion BTU)",
     ylab="Total Petroleum (trillion BTU)")
corr(~Coal+TotalPet,data=NRG1)
\end{verbatim}
\end{kframe}
\end{knitrout}

\newpage
\subsection*{Hatch Weight and Incubation Time of Geckos}
\begin{quote}
\textit{A \href{http://www.moonvalleyreptiles.com/breeding/incubation-length-and-hatch-weight}{hobbyist} hypothesized that there would be a positive association between length of incubation (days) and hatchling weight (grams) for Crested Geckos (Rhacodactylus ciliatus). To test this hypothesis she collected the incubation time and weight for 21 hatchlings (shown below). Construct a proper EDA for the relationship between incubation time and hatchling weight.}
\end{quote}

\begin{verbatim}
Time  53  54  56  60  60  60  60  60  63  63  77  77  78  81  82  82  83  83  84  90  90
Wt   1.5 1.7 1.4 1.0 1.4 1.5 1.7 1.8 1.4 1.5 1.1 1.6 1.5 1.9 1.4 1.5 1.3 1.7 1.6 1.4 1.8
\end{verbatim}



The relationship between hatchling weight and incubation time for the Crested Geckos is linear, without obvious outliers (\textit{though some may consider the small hatchling at 60 days to be an outlier}), without a definitive association, and weak ($r$=0.11) \figrefp{fig:scatGecko}. I did compute $r$ because no outliers were present and the relationship was linear (or, at least, it was not nonlinear).

\begin{knitrout}
\definecolor{shadecolor}{rgb}{0.922, 0.922, 0.922}\color{fgcolor}\begin{figure}[hbtp]

{\centering \includegraphics[width=.4\linewidth]{Figs/scatGecko-1} 

}

\caption[Scatterplot of hatchling weight versus incubation time for Crested Geckos]{Scatterplot of hatchling weight versus incubation time for Crested Geckos.}\label{fig:scatGecko}
\end{figure}


\end{knitrout}

\vspace{-18pt}
\subsubsection*{R Appendix}
\vspace{-6pt}
\begin{knitrout}
\definecolor{shadecolor}{rgb}{0.922, 0.922, 0.922}\color{fgcolor}\begin{kframe}
\begin{verbatim}
df <- read.csv("data/Gecko.csv")
plot(hatchwt~inctime,data=df,pch=21,bg="gray70",xlab="Incubation Time (days)",
     ylab="Hatchling Weight (grams)")
corr(~inctime+hatchwt,data=df)
\end{verbatim}
\end{kframe}
\end{knitrout}


\section{Cautions About Correlation}
\vspace{-6pt}
Examining relationships between pairs of quantitative variables is common practice. Using $r$ can be an important part of this analysis, as described above. However, $r$ can be abused through misapplication and misinterpretation. Thus, it is important to remember the following characteristics of correlation coefficients:
\vspace{-12pt}
\begin{Itemize}
  \item Variables must be quantitative (i.e., if you cannot make a scatterplot, then you cannot calculate $r$).
  \item The correlation coefficient only measures strength of \textbf{LINEAR} relationships (i.e., if the form of the relationship is not linear, then $r$ is meaningless and should not be calculated).
  \item The units that the variables are measured in do not matter (i.e., $r$ is the same between heights and weights measured in inches and lbs, inches and kg, m and kg, cm and kg, and cm and inches). This is because the variables are standardized when calculating $r$.
  \item The distinction between response and explanatory variables is not needed to compute $r$. That is, the correlation of GPA and ACT scores is the same as the correlation of ACT scores and GPA.
  \item Correlation coefficients are between -1 and 1.
  \item Correlation coefficients are strongly affected by outliers (simply, because both the mean and standard deviation, used in the calculation of $r$, are strongly affected by outliers).
\end{Itemize}

Additionally, correlation is not causation! In other words, just because a strong correlation is observed it does not mean that the explanatory variable caused the response variable (an exception may be in carefully designed experiments). For example, it was found above that highway gas mileage decreased linearly as the weight of the car increased. One must be careful here to not state that increasing the weight of the car CAUSED a decrease in MPG because these data are part of an observational study and several other important variables were not considered in the analysis. For example, the scatterplot in \figref{fig:carscat3}, coded for different numbers of cylinders in the car's engine, indicates that the number of cylinders may be inversely related to highway MPG and positively related to weight of the car. So, does the weight of the car, the number of cylinders, or both, explain the decrease in highway MPG?

\begin{knitrout}
\definecolor{shadecolor}{rgb}{0.922, 0.922, 0.922}\color{fgcolor}\begin{figure}[hbtp]

{\centering \includegraphics[width=.4\linewidth]{Figs/carscat3-1} 

}

\caption[Scatterplot between the highway MPG and weight of cars manufactured in 1993 separated by number of cylinders]{Scatterplot between the highway MPG and weight of cars manufactured in 1993 separated by number of cylinders.}\label{fig:carscat3}
\end{figure}


\end{knitrout}

More interesting examples (e.g., high correlation between number of people who drowned by falling into a pool and the annual number of films that Nicolas Cage appeared in) that further demonstrate that ``correlation is not causation'' can be found on the \href{http://www.tylervigen.com/spurious-correlations}{Spurious Correlations website.}

Finally, the word ``correlation'' is often misused in everyday language. ``Correlation'' should only be used when discussing the actual correlation coefficient (i.e., $r$). When discussing the association between two variables, one should use ``association'' or ``relationship'' rather than ``correlation.'' For example, one might ask ``What is the relationship between age and rate of cancer?'', but should not ask (unless specifically interested in $r$) ``What is the correlation between age and rate of cancer?''.



\chapter{Bivariate EDA - Categorical} \label{chap:BivEDACat}

\minitoc
\vspace{60pt}

\lettrine{T}{wo-way frequency tables} summarize two categorical variables recorded on the same individual by displaying levels of the first variable as rows and levels of the second variable as columns. Each cell in this table contains the frequency of individuals that were in the corresponding levels of each variable. These frequency tables are often converted to percentage tables for ease of summarization and comparison among populations. This module explores the construction and interpretation of frequency and percentage tables.

The General Sociological Survey (GSS) is a very large survey that has been administered 25 times since 1972. The purpose of the GSS is to gather data on contemporary American society in order to monitor and explain trends in attitudes, behaviors, and attributes. Data from the following two questions on the GSS are used throughout this module.

\begin{Itemize}
  \item What is your highest degree earned? [choices -- ``less than high school diploma'', ``high school diploma'', ``junior college'', ``bachelors'', or ``graduate''; labeled as \var{degree}]
  \item How willing would you be to accept cuts in your standard of living in order to protect the environment? [choices -- ``very willing'', ``fairly willing'', ``neither willing nor unwilling'', ``not very willing'', or ``not at all willing''; labeled as \var{grnsol}]
\end{Itemize}

These data, stored in \href{https://raw.githubusercontent.com/droglenc/NCData/master/GSSWill2Pay.csv}{GSSWill2Pay.csv}, are loaded into R and examined below.

\begin{knitrout}
\definecolor{shadecolor}{rgb}{0.922, 0.922, 0.922}\color{fgcolor}\begin{kframe}
\begin{verbatim}
> gss <- read.csv("data/GSSWill2Pay.csv")
\end{verbatim}
\end{kframe}
\end{knitrout}
\begin{knitrout}
\definecolor{shadecolor}{rgb}{0.922, 0.922, 0.922}\color{fgcolor}\begin{kframe}
\begin{verbatim}
> str(gss)
'data.frame':	3955 obs. of  2 variables:
 $ degree: Factor w/ 5 levels "BS","grad","HS",..: 5 5 5 5 5 5 5 5 5 5 ...
 $ grnsol: Factor w/ 5 levels "neither","un",..: 4 4 4 4 4 4 4 4 4 4 ...
> headtail(gss)
     degree grnsol
1      ltHS  vwill
2      ltHS  vwill
3      ltHS  vwill
3953   grad    vun
3954   grad    vun
3955   grad    vun
\end{verbatim}
\end{kframe}
\end{knitrout}
The \var{degree} and \var{grnsol} variables are both \emph{ordinal} categorical variables. By default the levels of factor variables are ordered alphabetically in R (as seen below with \R{levels()}).
\begin{knitrout}
\definecolor{shadecolor}{rgb}{0.922, 0.922, 0.922}\color{fgcolor}\begin{kframe}
\begin{verbatim}
> levels(gss$degree)
[1] "BS"   "grad" "HS"   "JC"   "ltHS"
> levels(gss$grnsol)
[1] "neither" "un"      "vun"     "vwill"   "will"   
\end{verbatim}
\end{kframe}
\end{knitrout}
The order of levels can be specified using \R{factor()}. The variable to be reordered is the first argument to \R{factor()}, as well as the object to the left of the assignment operator. The desired order of the levels is listed in a vector that is given to \R{levels=}. It is important that the levels in this vector are ``spelled'' exactly as they appeared originally. Correct orders for \var{degree} and \var{grnsol} in the \R{gss} data.frame are created below.
\begin{knitrout}
\definecolor{shadecolor}{rgb}{0.922, 0.922, 0.922}\color{fgcolor}\begin{kframe}
\begin{verbatim}
> gss$degree <- factor(gss$degree,levels=c("ltHS","HS","JC","BS","grad"))
> gss$grnsol <- factor(gss$grnsol,levels=c("vwill","will","neither","un","vun"))
> levels(gss$degree)
[1] "ltHS" "HS"   "JC"   "BS"   "grad"
> levels(gss$grnsol)
[1] "vwill"   "will"    "neither" "un"      "vun"    
\end{verbatim}
\end{kframe}
\end{knitrout}
If the natural order of levels is alphabetical or the variable is nominal, then \R{factor()} is not needed.

\warn{Levels for a factor variable are ordered alphabetically by default in R. If the factor variable is ordinal, then \R{factor()} with \R{levels=} may be needed to specify the correct order of levels.}


\newpage
\section{Frequency Tables}
\vspace{-6pt}

A common method of summarizing bivariate categorical data is to count individuals that have each combination of levels of the two categorical variables. For example, how many respondents had less than a HS degree and were very willing, how many had a high school degree and were willing, and so on. The count of the number of individuals of each combination is called a frequency. A two-way frequency table offers an efficient way to display these frequencies \tabrefp{tab:EnvFreq}. For example, 40 of the respondents had less than a high school degree and were very willing to take a cut in their standard of living to protect the environment. Similarly, 542 respondents had a high school degree and were willing to cut their standard of living.

% latex table generated in R 3.4.0 by xtable 1.8-2 package
% Wed Jun 14 09:31:41 2017
\begin{table}[ht]
\centering
\caption{Frequency table of respondent's highest completed degree and willingness to cut their standard of living to protect the environment.} 
\label{tab:EnvFreq}
\begin{tabular}{rrrrrrr}
  \hline
 & vwill & will & neither & un & vun & Sum \\ 
  \hline
ltHS & 40 & 145 & 132 & 151 & 178 & 646 \\ 
  HS & 87 & 542 & 512 & 557 & 392 & 2090 \\ 
  JC & 15 & 61 & 64 & 54 & 44 & 238 \\ 
  BS & 42 & 199 & 179 & 187 & 75 & 682 \\ 
  grad & 24 & 104 & 83 & 64 & 24 & 299 \\ 
  Sum & 208 & 1051 & 970 & 1013 & 713 & 3955 \\ 
   \hline
\end{tabular}
\end{table}


The margins of a two-way frequency table may be augmented with row and column totals (as in \tabref{tab:EnvFreq}). Each marginal total represents the distribution of one of the, while ignoring the other, categorical variable. The total column represents the distribution of the row variable; in this case, the highest degree completed. The total row represents the distribution of the column variable; in this case, willingness to cut their standard of living to protect the environment. Thus, for example there were 238 respondents whose highest completed degree was junior college and there were 713 respondents who were very unwilling to cut their standard of living to protect the environment.

If one variables can be considered as the response, then this variable should form the columns of the frequency table. For example, ``willingness to cut'' could be considered the response variable and it was, appropriately, placed as the column variable in \tabref{tab:EnvFreq}.

\vspace{-6pt}
\subsection*{Frequency Tables in R}
\vspace{-6pt}
Two-way frequency tables are constructed in R with \R{xtabs()}, where the first argument is a formula of the form \R{\TILDE rowvar+colvar} and the corresponding data.frame is in \R{data=}. The result of \R{xtabs()} should be assigned to an object for further use.
\vspace{-4pt}
\begin{knitrout}
\definecolor{shadecolor}{rgb}{0.922, 0.922, 0.922}\color{fgcolor}\begin{kframe}
\begin{verbatim}
> ( tbl1 <- xtabs(~degree+grnsol,data=gss) )
      grnsol
degree vwill will neither  un vun
  ltHS    40  145     132 151 178
  HS      87  542     512 557 392
  JC      15   61      64  54  44
  BS      42  199     179 187  75
  grad    24  104      83  64  24
\end{verbatim}
\end{kframe}
\end{knitrout}
\vspace{-4pt}
Totals may be added to the margins of a saved table with \R{addMargins()}. For example, \R{addMargins()} was used to construct \tabref{tab:EnvFreq} from \R{tbl1}.
\vspace{-4pt}
\begin{knitrout}
\definecolor{shadecolor}{rgb}{0.922, 0.922, 0.922}\color{fgcolor}\begin{kframe}
\begin{verbatim}
> addMargins(tbl1)
\end{verbatim}
\end{kframe}
\end{knitrout}


\section{Percentage Tables}
Two-way frequency tables may be converted to percentage tables for ease of comparison between levels of the variables and also between populations. For example, it is difficult to determine from \tabref{tab:EnvFreq} if respondents with a high school degree are more likely to be very willing to cut their standard of living than respondents with a graduate degree, because there are approximately seven times as many respondents with a high school degree. However, if the frequencies are converted to percentages, then this comparison is easily made. Three types of percentage tables may be constructed from a frequency table.

\subsection{Row-Percentage Table}
A \textbf{row-percentage table} is computed by dividing each cell of the frequency table by the total in the same row of the frequency table and multiplying by 100 \tabrefp{tab:EnvRowP}. For example, the value in the ``vwill'' column and ``ltHS'' row of the row-percentage table is computed by dividing the value in the ``vwill'' column and ``ltHS'' row of the frequency table (i.e., 40; \tabref{tab:EnvFreq}) by the ``Sum'' of the ``ltHS'' row of the frequency table (i.e., 646) and multiplying by 100.

% latex table generated in R 3.4.0 by xtable 1.8-2 package
% Wed Jun 14 09:31:41 2017
\begin{table}[ht]
\centering
\caption{Row-percentage table of respondent's highest completed degree and willingness to cut their standard of living to protect the environment.} 
\label{tab:EnvRowP}
\begin{tabular}{rrrrrrr}
  \hline
 & vwill & will & neither & un & vun & Sum \\ 
  \hline
ltHS & 6.2 & 22.4 & 20.4 & 23.4 & 27.6 & 100.0 \\ 
  HS & 4.2 & 25.9 & 24.5 & 26.7 & 18.8 & 100.1 \\ 
  JC & 6.3 & 25.6 & 26.9 & 22.7 & 18.5 & 100.0 \\ 
  BS & 6.2 & 29.2 & 26.2 & 27.4 & 11.0 & 100.0 \\ 
  grad & 8.0 & 34.8 & 27.8 & 21.4 & 8.0 & 100.0 \\ 
   \hline
\end{tabular}
\end{table}


The value in each cell of a row-percentage table is the percentage OF ALL individuals in that row that have the characteristic of that column. For example, 6.2\% of the respondents with less than a high school degree are very willing to cut their standard of living to protect the environment. This statement must be read carefully. OF THE RESPONDENTS WITH LESS THAN A HIGH SCHOOL DEGREE, not of all respondents, 6.2\% were very willing to cut their standard of living.

If the response variable formed the columns, then the row-percentage table  allows one to compare percentages in levels of the response (i.e., columns) across groups (i.e., rows). For example, one can see that there is a general decrease in the percentage of respondents that were ``very unwilling'' to cut their standard of living to protect the environment as the level of education increased \tabrefp{tab:EnvRowP}.

\subsubsection*{Row-Percentage Table in R}
Percentage tables are constructed in R by submitting the saved \R{xtabs()} object to \R{percTable()}. The number of decimals to display is controlled with \R{digits=}. A row-percentage table is constructed by including \R{margin=1}. For example, the code below produced \tabref{tab:EnvRowP}.
\begin{knitrout}
\definecolor{shadecolor}{rgb}{0.922, 0.922, 0.922}\color{fgcolor}\begin{kframe}
\begin{verbatim}
> percTable(tbl1,margin=1,digits=1)
\end{verbatim}
\end{kframe}
\end{knitrout}

\subsection{Column-Percentage Table}
A \textbf{column-percentage table} is computed by dividing each cell of the frequency table by the total in the same column of the frequency table and multiplying by 100 \tabrefp{tab:EnvColP}. For example, the value in the ``vwill'' column and ``ltHS'' row on the column-percentage table is computed by dividing the value in the ``vwill'' column and ``ltHS'' row of the frequency table (i.e., 40; \tabref{tab:EnvFreq}) by the ``Sum'' of the ``vwill'' column of the frequency table (i.e., 208) and multiplying by 100.

% latex table generated in R 3.4.0 by xtable 1.8-2 package
% Wed Jun 14 09:31:41 2017
\begin{table}[ht]
\centering
\caption{Column-percentage table of respondent's highest completed degree and willingness to cut their standard of living to protect the environment.} 
\label{tab:EnvColP}
\begin{tabular}{rrrrrr}
  \hline
 & vwill & will & neither & un & vun \\ 
  \hline
ltHS & 19.2 & 13.8 & 13.6 & 14.9 & 25.0 \\ 
  HS & 41.8 & 51.6 & 52.8 & 55.0 & 55.0 \\ 
  JC & 7.2 & 5.8 & 6.6 & 5.3 & 6.2 \\ 
  BS & 20.2 & 18.9 & 18.5 & 18.5 & 10.5 \\ 
  grad & 11.5 & 9.9 & 8.6 & 6.3 & 3.4 \\ 
  Sum & 99.9 & 100.0 & 100.1 & 100.0 & 100.1 \\ 
   \hline
\end{tabular}
\end{table}


The value in each cell of a column-percentage table is the percentage OF ALL individuals in that column that have the characteristic of that row. For example, 19.2\% of respondents who were very willing to cut their standard of living had less than a high school degree. Again, this is a very literal statement. OF THE RESPONDENTS WHO WERE VERY WILLING TO CUT THEIR STANDARD OF LIVING, not of all respondents, 19.2\% had less than a high school degree.

\subsubsection*{Column-Percentage Table in R}
A column-percentage table is constructed by submitting the saved \R{xtabs()} object to \R{percTable()} with \R{margin=2}. For example, the code below produced \tabref{tab:EnvColP}.
\begin{knitrout}
\definecolor{shadecolor}{rgb}{0.922, 0.922, 0.922}\color{fgcolor}\begin{kframe}
\begin{verbatim}
> percTable(tbl1,margin=2,digits=1)
\end{verbatim}
\end{kframe}
\end{knitrout}

\subsection{Total-Percentage Table}
Each value in a \textbf{total-percentage table} is computed by dividing each cell of the frequency table by the total number of ALL individuals in the frequency table and multiplying by 100. For example, the value in the ``vwill'' column and ``ltHS'' row of the table-percentage table \tabrefp{tab:EnvTblP} is computed by dividing the value in the ``vwill'' column and ``ltHS'' row of the frequency table (i.e., 40; \tabref{tab:EnvFreq}) by the ``Sum'' of the entire frequency table (i.e., 3955) and multiplying by 100.

% latex table generated in R 3.4.0 by xtable 1.8-2 package
% Wed Jun 14 09:31:41 2017
\begin{table}[ht]
\centering
\caption{Table-percentage table of respondent's highest completed degree and willingness to cut their standard of living to protect the environment.} 
\label{tab:EnvTblP}
\begin{tabular}{rrrrrrr}
  \hline
 & vwill & will & neither & un & vun & Sum \\ 
  \hline
ltHS & 1.0 & 3.7 & 3.3 & 3.8 & 4.5 & 16.3 \\ 
  HS & 2.2 & 13.7 & 12.9 & 14.1 & 9.9 & 52.8 \\ 
  JC & 0.4 & 1.5 & 1.6 & 1.4 & 1.1 & 6.0 \\ 
  BS & 1.1 & 5.0 & 4.5 & 4.7 & 1.9 & 17.2 \\ 
  grad & 0.6 & 2.6 & 2.1 & 1.6 & 0.6 & 7.5 \\ 
  Sum & 5.3 & 26.5 & 24.4 & 25.6 & 18.0 & 99.8 \\ 
   \hline
\end{tabular}
\end{table}


The value in each cell of a table-percentage table is the percentage OF ALL individuals that have the characteristic of that column AND that row. For example, 1.0\% of ALL respondents had less than a high school degree AND were very willing to cut their standard of living to protect the environment. Compare this interpretation to the interpretations from the row and column-percentage tables above. This interpretation DOES refer to all respondents.

\subsubsection*{Total-Percentage Table in R}
A table-percentage table is constructed by submitting the saved \R{xtabs()} object to \R{percTable()} and omitting \R{margin=}. For example, the code below produced \tabref{tab:EnvTblP}.
\begin{knitrout}
\definecolor{shadecolor}{rgb}{0.922, 0.922, 0.922}\color{fgcolor}\begin{kframe}
\begin{verbatim}
> percTable(tbl1,digits=1)
\end{verbatim}
\end{kframe}
\end{knitrout}


\section{Which Table to Use?}
Determining which table to use comes from applying one simple rule and practicing with several tables. The rule comes from determining if the question restricts the frame of reference to a particular level or category of one of the variables. If the question does restrict to a particular level, then either the row or column-percentage table that similarly restricts the frame of reference must be used. If a restriction to a particular level is not made, then the total-percentage table is used.

For example, consider the question -- ``What percentage of respondents with a bachelor's degree were very unwilling to cut their standard of living to protect the environment?''  This question refers to only respondents with bachelor's degrees (i.e., ``... of respondents with a bachelor's degree ...''). Thus, the answer is restricted to the ``BS'' row of the frequency table. The ROW-percentage table restricts the original table to the row levels and would be used to answer this question. Therefore, 11.0\% of respondents with bachelor's degrees were very unwilling to cut their standard of living to protect the environment \tabrefp{tab:EnvRowP}.

Now consider the question -- ``What percentage of all respondents had a high school degree and were very willing to cut their standard of living?''  This question does not restrict the frame of reference because it refers to ``... of all respondents ...''. Therefore, from the total-percentage table \tabrefp{tab:EnvTblP}, 2.2\% of respondents had a high school degree and were very willing to cut their standard of living.

Also consider this question -- ``What percentage of respondents who were neither willing nor unwilling to cut their standard of living had graduate degrees?''  This question refers only to respondents who were neither willing nor unwilling to cut their standard of living and, thus, restricts the question to the ``neither'' column of the frequency table. Thus, the answer will come from the COLUMN-percentage table. Therefore, 8.6\% of respondents who were neither willing nor unwilling to cut their standard of living had graduate degrees \tabrefp{tab:EnvColP}.

Finally, consider this question -- ``What percentage of all respondents were very willing to cut their standard of living to help the environment?''  This question has no restrictions, so the total-percentage table would be used. In addition, this question is only concerned with one of the two variables; thus, the answer will come from a marginal distribution. Therefore, 208 out of all 3955 respondents, or 5.3\%, were very willing to cut their standard of living to help the environment.

\warn{To determine which percentage table to use determine what type of restriction, if any, has been placed on the frame of reference for the question.}

\vspace{-12pt}
\warn{If a question does not refer to one of the two variables, then the answer will generally come from the marginal distribution of the other variable.}



\chapter{Linear Regression}  \label{chap:Regress}

\minitoc
\vspace{48pt}

\lettrine{L}{inear regression analysis is used to model the relationship} between two quantitative variables for two related purposes -- (i) explaining variability in the response variable and (ii) predicting future values of the response variable. Examples include predicting future sales of a product from its price, family expenditures on recreation from family income, an animal's food consumption in relation to ambient temperature, and a person's score on a German assessment test based on how many years the person studied German.

Exact predictions cannot be made because of natural variability. For example, two people with the same intake of mercury (from consumption of fish) will not have the same level of mercury in their blood stream (e.g., observe the two individuals in \figref{fig:HGscat} that had intakes of 580 ug HG/day). Thus, the best that can be accomplished is to predict the average or expected value for a person with a particular intake value. This is accomplished by finding the line that best ``fits'' the points on a scatterplot of the data and using that line to make predictions. Finding and using the ``best-fit'' line is the topic of this module.

\begin{knitrout}
\definecolor{shadecolor}{rgb}{0.922, 0.922, 0.922}\color{fgcolor}\begin{figure}[hbtp]

{\centering \includegraphics[width=.4\linewidth]{Figs/HGscat-1} 

}

\caption[Scatterplot of intake of mercury in fish and the mercury in the blood stream]{Scatterplot of intake of mercury in fish and the mercury in the blood stream. The two individuals mentioned in the main text are shown as red points.}\label{fig:HGscat}
\end{figure}


\end{knitrout}


\section{Response and Explanatory Variables}
Recall from \sectref{sect:RespExplan1} that the response (or dependent) variable is the variable to be predicted or explained and the explanatory (or independent) variable is the variable that will help do the predicting or explaining. In the examples mentioned above, future sales, family expenditures on recreation, the animal's food consumption, and score on the assessment test are response variables and product price, family income, temperature, and years studying German are explanatory variables, respectively. The response variable is on the y-axis and the explanatory variable is on the x-axis of scatterplots.


\section{Slope and Intercept}
The equation of a line is commonly expressed as,
  \[ y = mx + b  \]
where both $x$ and $y$ are variables, $m$ represents the slope of the line, and $b$ represents the y-intercept.\footnote{Hereafter, simply called the ``intercept.''}  It is important that you can look at the equation of a line and identify the response variable, explanatory variable, slope, and intercept. The response variable will always appear on one side of the equation (usually the left) by itself.  The value or symbol that is multiplied by the explanatory variable (e.g., $x$) is the slope, and the value or symbol by itself is the intercept. For example, in
\[ blood = 3.501 + 0.579*intake \]
\var{blood} is the response variable, \var{intake} is the explanatory variable, $0.579$ is the slope (it is multiplied by the explanatory variable), and $3.501$ is the intercept (it is not multiplied by anything in the equation). The same conclusions would be made if the equation had been written as
  \[ blood = 0.579*intake+3.501 \]

\warn{In the equation of a line, the slope is always multiplied by the explanatory variable and the intercept is always by itself.}

In addition to being able to identify the slope and intercept of a line you also need to be able to interpret these values. Most students define the slope as ``rise over run'' and the intercept as ``where the line crosses the y-axis.''  These ``definitions'' are loose geometric representations. For our purposes, the slope and intercept must be more strictly defined.

To define the slope, first think of ``plugging'' two values of intake into the equation discussed above. For example, if $intake=100$, then $blood=3.501+0.579*100$=61.40 and if $intake$ is one unit larger at $101$), then $blood=3.501+0.579*101$=61.98.\footnote{For simplicity of exposition, the actual units are not used in this discussion. However, ``units'' would usually be replaced with the actual units used for the measurements.} The difference between these two values is 61.98-61.40=$0.579$. Thus, the slope is the change in value of the response variable for a single unit change in the value of the explanatory variable \figrefp{fig:SlopeInt}. That is, mercury in the blood changes 0.579 units for a single unit change in mercury intake. So, if an individual increases mercury intake by one unit, then mercury in the blood will increase by 0.579 units, ON AVERAGE. Alternatively, if one individual has one more unit of mercury intake than another individual, then the first individual will have, ON AVERAGE, 0.579 more units of mercury in the blood.

\begin{knitrout}
\definecolor{shadecolor}{rgb}{0.922, 0.922, 0.922}\color{fgcolor}\begin{figure}[hbtp]

{\centering \includegraphics[width=.4\linewidth]{Figs/SlopeInt-1} 

}

\caption[Schematic representation of the meaning of the intercept and slope in a linear equation]{Schematic representation of the meaning of the intercept and slope in a linear equation.}\label{fig:SlopeInt}
\end{figure}


\end{knitrout}

To define the intercept, first ``plug'' $intake=0$ into the equation discussed above; i.e., $blood=3.501+0.579*0$ = $3.501$. Thus, the intercept is the value of the response variable when the explanatory variable is equal to zero \figrefp{fig:SlopeInt}. In this example, the AVERAGE mercury in the blood for an individual with no mercury intake is 3.501. Many times, as is true with this example, the interpretation of the intercept will be nonsensical. This is because $x=0$ will likely be outside the range of the data collected and, perhaps, outside the range of possible data that could be collected.

The equation of the line is a model for the relationship depicted in a scatterplot. Thus, the interpretations for the slope and intercept represent the \textit{average} change or the \textit{average} response variable. Thus, whenever a slope or intercept is being interpreted it must be noted that the result is an \textit{average} or \textit{on average}.


\section{Predictions}
Once a best-fit line has been identified (criteria for doing so is discussed in \sectref{sect:BestFitLine}), the equation of the line can be used to predict the average value of the response variable for individuals with a particular value of the explanatory variable. For example, the best-fit line for the mercury data shown in \figref{fig:HGscat} is
  \[ blood = 3.501 + 0.579*intake \]
Thus, the predicated average level of mercury in the blood for an individual that consumed 240 ug HG/day is found with
  \[ blood = 3.501 + 0.579*240 = 142.461 \]
Similarly, the predicted average level of mercury in the blood for an individual that consumed 575 ug HG/day is found with
  \[ blood = 3.501 + 0.579*575 = 336.426 \]
A prediction may be visualized by finding the value of the explanatory variable on the x-axis, drawing a vertical line until the best-fit line is reached, and then drawing a horizontal line over to the y-axis where the value of the response variable is read \figrefp{fig:HGpredict}.

\begin{knitrout}
\definecolor{shadecolor}{rgb}{0.922, 0.922, 0.922}\color{fgcolor}\begin{figure}[hbtp]

{\centering \includegraphics[width=.4\linewidth]{Figs/HGpredict-1} 

}

\caption[Scatterplot between the intake of mercury in fish and the mercury in the blood stream of individuals with superimposed best-fit regression line illustrating predictions for two values of mercury intake]{Scatterplot between the intake of mercury in fish and the mercury in the blood stream of individuals with superimposed best-fit regression line illustrating predictions for two values of mercury intake.}\label{fig:HGpredict}
\end{figure}


\end{knitrout}

When predicting values of the response variable, it is important to not extrapolate beyond the range of the data. In other words, predictions with values outside the range of observed values of the explanatory variable should be made cautiously (if at all). An excellent example would be to consider height ``data'' collected during the early parts of a human's life (say the first ten years). During these early years there is likely a good fit between height (the response variable) and age. However, using this relationship to predict an individual's height at age 40 would likely result in a ridiculous answer (e.g., over ten feet). The problem here is that the linear relationship only holds for the observed data (i.e., the first ten years of life); it is not known if the same linear relationship exists outside that range of years. In fact, with human heights, it is generally known that growth first slows, eventually quits, and may, at very old ages, actually decline. Thus, the linear relationship found early in life does not hold for later years. Critical mistakes can be made when using a linear relationship to extrapolate outside the range of the data.

\section{Residuals}
The predicted value is a ``best-guess'' for an individual based on the best-fit line. The actual value for any individual is likely to be different from this predicted value. The \textbf{residual} is a measure of how ``far off'' the prediction is from what is actually observed. Specifically, the residual for an individual is found by subtracting the predicted value (given the individual's observed value of the explanatory variable) from the individual's observed value of the response variable, or
  \[ \text{residual}=\text{observed response}-\text{predicted response} \]

For example, consider an individual that has an observed intake of 650 and an observed level of mercury in the blood of 480. As shown in the previous section, the predicted level of mercury in the blood for this individual is
  \[ blood = 3.501 + 0.579*650 = 379.851 \]

The residual for this individual is then $480-379.851$ = $100.149$. This positive residual indicates that the observed value is approximately 100 units greater than the average for individuals with an intake of 650.\footnote{In other words, the observed value is ``above'' the line.}  As a second example, consider an individual with an observed intake of 250 and an observed level of mercury in the blood of 105. The predicted value for this individual is
  \[ blood = 3.501 + 0.579*250 = 148.251 \]

and the residual is $105-148.251$ = $-43.251$. This negative residual indicates that the observed value is approximately 43 units less than the average for individuals with an intake of 250.

Visually, a residual is the vertical distance between an individual's point and the best-fit line \figrefp{fig:HGresidual}.

\begin{knitrout}
\definecolor{shadecolor}{rgb}{0.922, 0.922, 0.922}\color{fgcolor}\begin{figure}[hbtp]

{\centering \includegraphics[width=.4\linewidth]{Figs/HGresidual-1} 

}

\caption[Scatterplot between the intake of mercury in fish and the mercury in the blood stream of individuals with superimposed best-fit regression line illustrating the residuals for the two individuals discussed in the main text]{Scatterplot between the intake of mercury in fish and the mercury in the blood stream of individuals with superimposed best-fit regression line illustrating the residuals for the two individuals discussed in the main text.}\label{fig:HGresidual}
\end{figure}


\end{knitrout}

\section{Best-fit Criteria}\label{sect:BestFitLine}
An infinite number of lines can be placed on a graph, but many of those lines do not adequately describe the data. In contrast, many of the lines will appear, to our eye, to adequately describe the data. So, how does one find THE best-fit line from all possible lines. The \textbf{least-squares} method described below provides a quantifiable and objective measure of which line best ``fits'' the data.

Residuals are a measure of how far an individual is from a candidate best-fit line. Residuals computed from all individuals in a data set measure how far all individuals are from the candidate best-fit line. Thus, the residuals for all individuals can be used to identify the best-fit line.

The residual sum-of-squares (RSS) is the sum of all squared residuals. The least-squares criterion says that the ``best-fit'' line is the one line out of all possible lines that has the minimum RSS \figrefp{fig:RSSanim}.

\begin{knitrout}
\definecolor{shadecolor}{rgb}{0.922, 0.922, 0.922}\color{fgcolor}









































































\begin{figure}[hbtp]

{\centering \animategraphics[width=.8\linewidth,controls,palindrome,autoplay]{1}{Figs/RSSanim-}{1}{75}

}

\caption[An animation illustrating how the residual sum-of-squares (RSS) for a series of candidate lines (red lines) is minimized at the best-fit line (green line)]{An animation illustrating how the residual sum-of-squares (RSS) for a series of candidate lines (red lines) is minimized at the best-fit line (green line).}\label{fig:RSSanim}
\end{figure}


\end{knitrout}

The discussion thusfar implies that all possible lines must be ``fit'' to the data and the one with the minimum RSS is chosen as the ``best-fit'' line. As there are an infinite number of possible lines, this would be impossible to do. Theoretical statisticians have shown that the application of the least-squares criterion always produces a best-fit line with a slope given by
  \[ slope = r\frac{s_{y}}{s_{x}}  \]

and an intercept given by
  \[ intercept = \bar{y}-slope*\bar{x}   \]

where $\bar{x}$ and $s_{x}$ are the sample mean and standard deviation of the explanatory variable, $\bar{y}$ and $s_{y}$ are the sample mean and standard deviation of the response variable, and $r$ is the sample correlation coefficient between the two variables. Thus, using these formulas finds the slope and intercept for the line, out of all possible lines, that minimizes the RSS.


\section{Assumptions}\label{sect:RegAssumptions}
The least-squares method for finding the best-fit line only works appropriately if each of the following five assumptions about the data has been met.

\begin{Enumerate}
  \item A line describes the data (i.e., a linear form).
  \item Homoscedasticity.
  \item Normally distributed residuals at a given x.
  \item Independent residuals at a given x.
  \item The explanatory variable is measured without error.
\end{Enumerate}

While all five assumptions of linear regression are important, only the first two are vital when the best-fit line is being used primarily as a descriptive model for data.\footnote{In contrast to using the model to make inferences about a population model.}  Description is the primary goal of linear regression used in this course and, thus, only the first two assumptions are considered further.

The linearity assumption appears obvious -- if a line does not represent the data, then don't try to fit a line to it!  Violations of this assumption are evident by a non-linear or curving form in the scatterplot.

The homoscedasticity assumption states that the variability about the line is the same for all values of the explanatory variable. In other words, the dispersion of the data around the line must be the same along the entire line. Violations of this assumption generally present as a ``funnel-shaped'' dispersion of points from left-to-right on a scatterplot.

Violations of these assumptions are often evident on a ``fitted-line plot'', which is a scatterplot with the best-fit line superimposed \figrefp{fig:ResidPlotEx}.\footnote{Residual plots, not discussed in this text, are another plot that often times is used to better assess assumption violations.} If the points look more-or-less like random scatter around the best-fit line, then neither the linearity nor the homoscedasticity assumption has been violated. A violation of one of these assumptions should be obvious on the scatterplot. In other words, there should be a clear curvature or funneling on the plot.

\begin{knitrout}
\definecolor{shadecolor}{rgb}{0.922, 0.922, 0.922}\color{fgcolor}\begin{figure}[hbtp]

{\centering \includegraphics[width=.8\linewidth]{Figs/ResidPlotEx-1} 

}

\caption[Fitted-line plots illustrating when the regression assumptions are met (upper-left) and three common assumption violations]{Fitted-line plots illustrating when the regression assumptions are met (upper-left) and three common assumption violations.}\label{fig:ResidPlotEx}
\end{figure}


\end{knitrout}
\vspace{12pt}  %fixes spaces gobbled up by R code

In this text, if an assumption has been violated, then one should not continue to interpret the linear regression. However, in many instances, an assumption violation can be ``corrected'' by transforming one or both variables to a different scale. Transformations are not discussed in this book.

\warn{If the regression assumptions are not met, then the regression results should not be interpreted.}


\newpage
\section{Coefficient of Determination}
The coefficient of determination ($r^{2}$) is the proportion of the total variability in the response variable that is explained away by knowing the value of the explanatory variable and the best-fit model. In simple linear regression, $r^{2}$ is literally the square of $r$, the correlation coefficient.\footnote{Simple linear regression is the fitting of a model with a single explanatory variable and is the only model considered in this module and this course. See \sectref{sect:corr} for a review of the correlation coefficient.} Values of $r^{2}$ are between 0 and 1.\footnote{It is common for $r^{2}$ to be presented as a percentage.}

The meaning of $r^{2}$ can be examined by making predictions of the response variable with and without knowing the value of the explanatory variable. First, consider predicting the value of the response variable without any information about the explanatory variable. In this case, the best prediction is the sample mean of the response variable (represented by the dashed blue horizontal line in \figref{fig:CoeffDeterm}). However, because of natural variability, not all individuals will have this value. Thus, the prediction might be ``bracketed'' by predicting that the individual will be between the observed minimum and maximum values (solid blue horizontal lines). Loosely speaking, this range is the ``total variability in the response variable'' (blue box).

\begin{knitrout}
\definecolor{shadecolor}{rgb}{0.922, 0.922, 0.922}\color{fgcolor}\begin{figure}[hbtp]

{\centering \includegraphics[width=.8\linewidth]{Figs/CoeffDeterm-1} 

}

\caption[Fitted line plot with visual representations of variabilities explained and unexplained]{Fitted line plot with visual representations of variabilities explained and unexplained. A full explanation is in the text.}\label{fig:CoeffDeterm}
\end{figure}


\end{knitrout}

Suppose now that the response variable is predicted for an individual with a known value of the explanatory variable (e.g., at the dashed vertical red line in \figref{fig:CoeffDeterm}). The predicted value for this individual is the value of the response variable at the corresponding point on the best-fit line (dashed horizontal red line). Again, because of natural variability, not all individuals with this value of the explanatory variable will have this exact value of the response variable. However, the prediction is now ``bracketed'' by the minimum and maximum value of the response variable \textbf{ONLY} for those individuals with the same value of the explanatory variable (solid red horizontal lines). Loosely speaking, this range is the ``variability in the response variable remaining after knowing the value of the explanatory variable'' (red box). This is the variability in the response variable that remains even after knowing the value of the explanatory variable or the variability in the response variable that cannot be explained away (by the explanatory variable).

The portion of the total variability in the response variable that was explained away consists of all the values of the response variable that would no longer be entertained as possible predictions once the value of the explanatory variable is known (green box in \figref{fig:CoeffDeterm}).

Now, by the definition of $r^{2}$, $r^{2}$ can be visualized as the area of the green box divided by the area of the blue box. This calculation does not depend on which value of the explanatory variable is chosen as long as the data are evenly distributed around the line (i.e., homoscedasticity -- see \sectref{sect:RegAssumptions}).

If the variability explained away (green box) approaches the total variability in the response variable (blue box), then $r^{2}$ approaches 1. This will happen only if the variability around the line approaches zero. In contrast, the variability explained (green box) will approach zero if the slope is zero (i.e., no relationship between the response and explanatory variables). Thus, values of $r^{2}$ also indicate the strength of the relationship; values near 1 are stronger than values near 0. Values near 1 also mean that predictions will be fairly accurate -- i.e., there is little variability remaining after knowing the explanatory variable.

\warn{A value of $r^{2}$ near 1 represents a strong relationship between the response and explanatory variables that will lead to accurate predictions.}


\section{Examples I}
There are twelve questions that are commonly asked about linear regression results. These twelve questions are listed below with some hints about things to remember when answering some of the questions. An example of these questions in context is then provided.

\begin{enumerate}
  \item What is the response variable?  \textit{Identify which variable is to be predicted or explained, which variable is dependent on another variable, which would be hardest to measure, or which is on the y-axis.}
  \item What is the explanatory variable?  \textit{The remaining variable after identifying the response variable.}
  \item Comment on linearity and homoscedasticity. \textit{Examine fitted-line plot for curvature (i.e., non-linearity) or a funnel-shape (i.e., heteroscedasticity).}
  \item What is the equation of the best-fit line?  \textit{In the generic equation of the line ($y=mx+b$) replace $y$ with the name of the response variable, $x$ with the name of the explanatory variable, $m$ with the value of the slope, and $b$ with the value of the intercept.}
  \item Interpret the value of the slope. \textit{Comment on how the response variable changes by slope amount for each one unit change of the explanatory variable, on average.}
  \item Interpret the value of the intercept. \textit{Comment on how the response variable equals the intercept, on average, if the explanatory variable is zero.}
  \item Make a prediction given a value of the explanatory variable. \textit{Plug the given value of the explanatory variable into the equation of the best-fit line. Make sure that this is not an extrapolation.}
  \item Compute a residual given values of both the explanatory and response variables. \textit{Make a prediction (see previous question) and then subtract this value from the observed value of the response. Make sure that the prediction is not an extrapolation.}
  \item Identify an extrapolation in the context of a prediction problem. \textit{Examine the x-axis scale on the fitted-line plot and do not make predictions outside of the plotted range.}
  \item What is the proportion of variability in the response variable explained by knowing the value of the explanatory variable?  \textit{This is $r^{2}$.}
  \item What is the correlation coefficient?  \textit{This is the square root of $r^{2}$. Make sure to put a negative sign on the result if the slope is negative.}
  \item How much does the response variable change if the explanatory variable changes by X units?  \textit{This is an alternative to asking for an interpretation of the slope. If the explanatory variable changes by X units, then the response variable will change by X*slope units, on average.}
\end{enumerate}

All answers should refer to the variables of the problem -- thus, ``y'', ``x'', ``response'', or ``explanatory'' should not be in any part of any answer. The questions about the slope, intercept, and predictions need to explicitly identify that the answer is an ``average'' or ``on average.''

\newpage
\subsection*{Chimp Hunting Parties}
\begin{quote}
\textit{Stanford (1996) gathered data to determine if the size of the hunting party (number of individuals hunting together) affected the hunting success of the party (number of hunts that resulted in a kill) for wild chimpanzees (Pan troglodytes) at Gombe. The results of their analysis for 17 hunting parties is shown in the figure below.\footnote{These data are in \href{https://raw.githubusercontent.com/droglenc/NCData/master/Chimp.csv}{Chimp.csv}.}  Use these results to answer the questions below.}
\end{quote}

\begin{knitrout}
\definecolor{shadecolor}{rgb}{0.922, 0.922, 0.922}\color{fgcolor}

{\centering \includegraphics[width=.4\linewidth]{Figs/ChimpFLP-1} 

}



\end{knitrout}

\begin{QAlist}
  \item What is the response variable?
  \begin{QAlist}
    \item The response variable is the percent of successful hunts because the authors are attempting to see if success depends on hunting party size. Additionally, the percent of successful hunts is shown on the y-axis.
  \end{QAlist}
  \item What is the explanatory variable?
  \begin{QAlist}
    \item The explanatory variable is the size of the hunting party.
  \end{QAlist}
  \item In terms of the variables of the problem, what is the equation of the best-fit line?
  \begin{QAlist}
    \item The equation of the best-fit line is \% Success of Hunt = 24.215 + 3.705*Number of Hunting Party Members.
  \end{QAlist}
  \item Interpret the value of the slope in terms of the variables of the problem.
  \begin{QAlist}
    \item The slope indicates that the percent of successful hunts increases by 3.705, on average, for every increase of one member to the hunting party.
  \end{QAlist}
  \item Interpret the value of the intercept in terms of the variables of the problem.
  \begin{QAlist}
    \item The intercept indicates that the percent of successful hunts is 24.215, on average, for hunting parties with no members.
  \end{QAlist}
  \item What is the predicted hunt success if the hunting party consists of 20 chimpanzees?
  \begin{QAlist}
    \item The predicted hunt success for parties with 20 individuals is an extrapolation, because 20 is outside the range of number of members observed on the x-axis of the fitted-line plot.
  \end{QAlist}
  \item What is the predicted hunt success if the hunting party consists of 12 chimpanzees?
  \begin{QAlist}
    \item The predicted hunt success for parties with 12 individuals is 24.215 + 3.705*12 = 68.7\%.
  \end{QAlist}
  \item What is the residual if the hunt success for 10 individuals is 50\%?
  \begin{QAlist}
    \item The residual in this case is $50$-(24.215 + 3.705*10) = $50$-61.3 = -11.3. Therefore, it appears that the success of this hunting party is 11.3\% lower than average for this size of hunting party.
  \end{QAlist}
  \item What proportion of the variability in hunting success is explained by knowing the size of the hunting party?
  \begin{QAlist}
    \item The proportion of the variability in hunting success that is explained by knowing the size of the hunting party is $r^{2}$=0.88.
  \end{QAlist}
  \item What is the correlation between hunting success and size of hunting party?
  \begin{QAlist}
    \item The correlation between hunting success and size of hunting party is $r=$0.94.
  \end{QAlist}
  \item How much does hunt success decrease, on average, if there are two fewer individuals in the party?
  \begin{QAlist}
    \item If the hunting party has two fewer members, then the hunting success would decrease by 7.4\% (i.e., $-2$*3.705), on average.
  \end{QAlist}
  \item Does any aspect of this regression concern you (i.e., consider the regression assumptions)?
  \begin{QAlist}
    \item The data appear to be very slightly curved but there is no evidence of a funnel-shape. Thus, the data may be slightly non-linear but they appear homoscedastic.
  \end{QAlist}
\end{QAlist}

\warn{All interpretations should be ``in terms of the variables of the problem'' rather than the generic terms of x, y, response variable, and explanatory variable.}


\section{Regression in R}
The mercury intake and amount in the blood data is loaded below to be used as an example for finding a regression line with R.
\begin{knitrout}
\definecolor{shadecolor}{rgb}{0.922, 0.922, 0.922}\color{fgcolor}\begin{kframe}
\begin{verbatim}
> setwd('c:/data/')
> merc <- read.csv("Mercury.csv")
\end{verbatim}
\end{kframe}
\end{knitrout}
\vspace{-14pt}
\begin{knitrout}
\definecolor{shadecolor}{rgb}{0.922, 0.922, 0.922}\color{fgcolor}\begin{kframe}
\begin{verbatim}
> str(merc)
'data.frame':	13 obs. of  2 variables:
 $ intake: num  180 200 230 410 600 550 275 580 580 105 ...
 $ blood : num  90 120 125 290 310 290 170 275 350 70 ...
\end{verbatim}
\end{kframe}
\end{knitrout}

The linear regression model is fit to two quantitative variables with \R{lm()}. The first argument is a formula of the form \R{response\TILDE explanatory}, where \R{response} contains the response variable and \R{explanatory} contains the explanatory variable, and the corresponding data.frame is in \R{data=}. The results of \R{lm()} should be assigned to an object so that specific results can be extracted.

\warn{The same formula used to make a scatterplot with \R{plot()} is used in \R{lm()} to find the best-fit line.}


The regression was fit to the mercury data below. From this it is seen that the intercept is 3.501 and the slope is 0.579.
\begin{knitrout}
\definecolor{shadecolor}{rgb}{0.922, 0.922, 0.922}\color{fgcolor}\begin{kframe}
\begin{verbatim}
> ( lm1 <- lm(blood~intake,data=merc) )
Coefficients:
(Intercept)       intake  
     3.5007       0.5791  
\end{verbatim}
\end{kframe}
\end{knitrout}

A fitted-line plot \figrefp{fig:HGFLP} is constructed by submitting the \R{lm()} object to \R{fitPlot()}. Aspects of this plot can be adjusted using the same arguments as described for \R{plot()} in \sectref{sect:ScatterplotsR}.
\begin{knitrout}
\definecolor{shadecolor}{rgb}{0.922, 0.922, 0.922}\color{fgcolor}\begin{kframe}
\begin{verbatim}
> fitPlot(lm1,pch=21,bg="gray70",xlab="Mercury Intake",ylab="Mercury in the Blood")
\end{verbatim}
\end{kframe}\begin{figure}[hbtp]

{\centering \includegraphics[width=.4\linewidth]{Figs/HGFLP-1} 

}

\caption[Fitted-line plots for the regression of mercury in the blood on mercury intake]{Fitted-line plots for the regression of mercury in the blood on mercury intake.}\label{fig:HGFLP}
\end{figure}


\end{knitrout}

Predicted values from the linear regression are obtained with \R{predict()}. The \R{predict()} function requires the saved \R{lm()} object as its first argument. The second argument is a data.frame constructed with \R{data.frame()} that contains the \textbf{EXACT} name of the explanatory variable as it appeared in \R{lm()} set equal to the value of the explanatory at which the prediction should be made. For example, the predicted amount of mercury in the blood for an intake of 240 $\mu$g per day is 142.5, as obtained below.
\begin{knitrout}
\definecolor{shadecolor}{rgb}{0.922, 0.922, 0.922}\color{fgcolor}\begin{kframe}
\begin{verbatim}
> predict(lm1,data.frame(intake=240))
       1 
142.4895 
\end{verbatim}
\end{kframe}
\end{knitrout}

The coefficient of determination is computed by submitting the saved \R{lm()} object to \R{rSquared()}. For example, 88.4\% of the variability in mercury in the blood is explained by knowing the amount of mercury at intake. [Note the use of \R{digits=} to control the number of decimals.]
\begin{knitrout}
\definecolor{shadecolor}{rgb}{0.922, 0.922, 0.922}\color{fgcolor}\begin{kframe}
\begin{verbatim}
> rSquared(lm1,digits=3)
[1] 0.884
\end{verbatim}
\end{kframe}
\end{knitrout}


\section{Examples II}
\subsection*{Car Weight and MPG}
In \modref{chap:BivEDAQuant}, an EDA for the relationship between \var{HMPG} (the highway miles per gallon) and \var{Weight} (lbs) of 93 cars from the 1993 model year was performed. This relationship will be explored further here as an example of a complete regression analysis. In this analysis, the regression output will be examined within the context of answering the twelve typical questions. These data are read into R below and the linear regression model is fit, coefficients extracted, fitted-line plot constructed, and coefficient of determination extracted.

\begin{knitrout}
\definecolor{shadecolor}{rgb}{0.922, 0.922, 0.922}\color{fgcolor}\begin{kframe}
\begin{verbatim}
> cars93 <- read.csv("data/93cars.csv")
\end{verbatim}
\end{kframe}
\end{knitrout}
\vspace{-12pt}
\begin{knitrout}
\definecolor{shadecolor}{rgb}{0.922, 0.922, 0.922}\color{fgcolor}\begin{kframe}
\begin{verbatim}
> ( lm2 <- lm(HMPG~Weight,data=cars93) )
Coefficients:
(Intercept)       Weight  
  51.601365    -0.007327  
> fitPlot(lm2,ylab="Highway MPG")
> rSquared(lm2,digits=3)
[1] 0.657
\end{verbatim}
\end{kframe}\begin{figure}[hbtp]

{\centering \includegraphics[width=.4\linewidth]{Figs/CarFit-1} 

}

\caption[Fitted line plot of the regression of highway MPG on weight of 93 cars from 1993]{Fitted line plot of the regression of highway MPG on weight of 93 cars from 1993.}\label{fig:CarFit}
\end{figure}


\end{knitrout}

The simple linear regression model appears to fit the data moderately well as the fitted-line plot \figrefp{fig:CarFit} shows only a very slight curvature and only very slight heteroscedasticity.\footnote{In advanced statistics books, objective measures for determining whether there is significant curvature or heteroscedasticity in the data are used. In this book, we will only be concerned with whether there is strong evidence of curvature or heteroscedasticity. There does not seem to be either here.}  The sample slope is -0.0073, the sample intercept is 51.6, and the coefficient of determination is 0.657.

\begin{QAlist}
  \item What is the response variable?
  \begin{QAlist}
    \item The response variable in this analysis is the highway MPG, because that is the variable that we are trying to learn about or explain the variability of.
  \end{QAlist}
  \item What is the explanatory variable?
  \begin{QAlist}
    \item The explanatory variable in this analysis is the weight of the car (by process of elimination).
  \end{QAlist}
  \item In terms of the variables of the problem, what is the equation of the best-fit line?
  \begin{QAlist}
    \item The equation of the best-fit line for this problem is HMPG = 51.6 - 0.0073Weight.
  \end{QAlist}
  \item Interpret the value of the slope in terms of the variables of the problem.
  \begin{QAlist}
    \item The slope indicates that for every increase of one pound of car weight the highway MPG decreases by -0.0073, on average.
  \end{QAlist}
  \item Interpret the value of the intercept in terms of the variables of the problem.
  \begin{QAlist}
    \item The intercept indicates that a car with 0 weight will have a highway MPG value of 51.6, on average.\footnote{This is the correct interpretation of the intercept. However, it is nonsensical because it is an extrapolation; i.e., no car will weigh 0 pounds.}
  \end{QAlist}
  \item What is the predicted highway MPG for a car that weighs 3100 lbs?
  \begin{QAlist}
    \item The predicted highway MPG for a car that weighs 3100 lbs is 51.60137 - 0.00733(3100) = 28.9 MPG. Alternatively, this value is computed with
\begin{knitrout}
\definecolor{shadecolor}{rgb}{0.922, 0.922, 0.922}\color{fgcolor}\begin{kframe}
\begin{verbatim}
> predict(lm2,data.frame(Weight=3100))
       1 
28.88748 
\end{verbatim}
\end{kframe}
\end{knitrout}
  \end{QAlist}
  \item What is the predicted highway MPG for a car that weighs 5100 lbs?
  \begin{QAlist}
    \item The predicted highway MPG for a car that weighs 5100 lbs should not be computed with the results of this regression, because 5100 lbs is outside the domain of the data \figrefp{fig:CarFit}.
  \end{QAlist}
  \item What is the residual for a car that weights 3500 lbs and has a highway MPG of 24?
  \begin{QAlist}
    \item The predicted highway MPG for a car that weighs 3500 lbs is 51.60137 - 0.00733(3500) = 26.0. Thus, the residual for this car is 24 - 26.0 = -2.0. Alternatively, this is computed in R with
\begin{knitrout}
\definecolor{shadecolor}{rgb}{0.922, 0.922, 0.922}\color{fgcolor}\begin{kframe}
\begin{verbatim}
> 24-predict(lm2,data.frame(Weight=3500))
        1 
-1.956658 
\end{verbatim}
\end{kframe}
\end{knitrout}
Therefore, it appears that this car gets 2.0 MPG less than an average car with the same weight.
  \end{QAlist}
  \item What proportion of the variability in highway MPG is explained by knowing the weight of the car?
  \begin{QAlist}
    \item The proportion of the variability in highway MPG that is explained by knowing the weight of the car is $r^{2}$=0.66.
  \end{QAlist}
  \item What is the correlation between highway MPG and car weight?
  \begin{QAlist}
    \item The correlation between highway MPG and car weight is $r=$-0.81.\footnote{Put a negative sign in front of your result from taking the square root of $r^2$, because the relationship between highway MPG and weight is negative.}
  \end{QAlist}
  \item How much is the highway MPG expected to change if a car is 1000 lbs heavier?
  \begin{QAlist}
    \item If the car was 1000 lbs heavier, you would expect the car's highway MPG to decrease by 7.33 (i.e., 1000 slopes).
  \end{QAlist}
\end{QAlist}



\chapter{Probability Introduction} \label{chap:ProbIntro}

\lettrine{P}{robability} is the ``language'' used to describe the proportion of times that a random event will occur. The language of probability is at the center of statistical inference (see Modules \ref{chap:HypothesisTests} and \ref{chap:ConfidenceRegions}). Only a minimal understanding of probability is required to understand most basic inferential methods, including all of those in this course. Thus, only a short, example-based, introduction to probability is provided here.\footnote{A deeper understanding of probability is required to understand more complex inferential methods beyond those in this course.}

The most basic forms of probability assume that items are selected randomly. In other words, simple probability calculations require that each item, whether that item is an individual or an entire sample, has the same chance of being selected. Thus, in simple intuitive examples it will be stated that the individuals were ``thoroughly mixed'' and more realistic examples will require randomization.\footnote{See \modref{chap:DataProd} for methods to randomly select or allocate individuals.}

If every individual has the same chance of being selected, then the probability of an event is equal to the proportion of items in the event out of the entire population. In other words, the probability is the number of items in the event divided by the total number of items in the population.

For example, the probability of selecting a red ball from a thoroughly mixed box containing 15 red and 10 blue balls is equal to $\frac{15}{25}=0.6$ (i.e., 15 individuals (``balls'') in the event (``red'') divided by the total number of individuals (``all balls in the box''); \figref{fig:ProbBox}-Left). Similarly, the probability of randomly selecting a woman from a room with 20 women and 30 men is 0.4 ($=\frac{20}{50}$; \figref{fig:ProbBox}-Right). In both examples, the calculation can be considered a probability because (i) individuals were randomly selected and (ii) a proportion of a total was computed.

\begin{knitrout}
\definecolor{shadecolor}{rgb}{0.922, 0.922, 0.922}\color{fgcolor}\begin{figure}[hbtp]

{\centering \includegraphics[width=.3\linewidth]{Figs/ProbBox-1} 
\includegraphics[width=.3\linewidth]{Figs/ProbBox-2} 

}

\caption[Depictions of a `box' with 15 red balls and 10 blue balls (Left) and a `room' with 30 men and 20 women (Right)]{Depictions of a `box' with 15 red balls and 10 blue balls (Left) and a `room' with 30 men and 20 women (Right).}\label{fig:ProbBox}
\end{figure}


\end{knitrout}

\newpage
The two previous examples are simple because the selection is from a small, discrete set of items. Probabilities may be computed for a continuous variable if the distribution of that variable is known for the entire population. For example, the probability that a random individual is greater than 71 inches tall can be calculated if the distribution of heights for all individuals in the population is known (or reasonably approximated). For example, as shown in \modref{chap:NormDist}, if it can be assumed that heights is $N(66,3)$, then the proportion of individuals in the population taller than 71 inches tall is 0.0478 \figrefp{fig:ProbNorm}.\footnote{As computed with \R{distrib(71,mean=66,sd=3,lower.tail=FALSE)}.} This result is a probability because (i) the individual was randomly selected and (ii) the proportion of all individuals of interest in the entire population was found.

\begin{knitrout}
\definecolor{shadecolor}{rgb}{0.922, 0.922, 0.922}\color{fgcolor}\begin{figure}[hbtp]

{\centering \includegraphics[width=.4\linewidth]{Figs/ProbNorm-1} 

}

\caption[Calculation of the probability that a randomly selected individual from a $N(66,3)$ population will have a height greater than 71 inches]{Calculation of the probability that a randomly selected individual from a $N(66,3)$ population will have a height greater than 71 inches.}\label{fig:ProbNorm}
\end{figure}


\end{knitrout}

A theory that explains the distribution of statistics computed from all possible samples from a population will be developed in \modref{chap:SamplingDist}. This distribution will be used to compute the probability of observing a particular range of statistics from random samples. This technique is the basis for making statistical inferences in Modules \ref{chap:HypothesisTests} and \ref{chap:ConfidenceRegions}.









%    \cleardoublepage
%    \phantomsection
%    \addcontentsline{toc}{part}{Appendix}
%    \chapter*{Appendices}
%    \appendix


%  \backmatter
%    %BIBLIOGRAPHY ---------------------------------------------------------------------------------------------------------
\cleardoublepage                             %not sure why but this is needed so TOC entry will point to right start page
\phantomsection                              %not sure why but this is needed so TOC entry will point to right start page
\addcontentsline{toc}{part}{Bibliography}    %Add a TOC entry
\bibliography{c:/aaaWork/zGnrlLatex/DHO_Bib} %make the bibliography
 
%INDEX ----------------------------------------------------------------------------------------------------------------
% cross references for the index
\index{Alternative Hypothesis|see{Hypothesis, Alternative}}
\index{Coefficient of Determination|see{$r^{2}$}}
\index{Forward calculation|see{Normal Distribution, Finding areas}}
\index{Frequency Table|see{Table, Frequency}}
\index{Goodness-of-Fit Test|see{Chi-square}}
\index{Line!Finding best-fit|see{Regression}}
\index{Mean!Inference|see{Z-test and t-Test}}
\index{Null Hypothesis|see{Hypothesis, Null}}
\index{Percentage Table|see{Table}}
\index{Proportions Table|see{Table, Proportion}}
\index{Row Proportions Table|see{Table, Proportion}}
\index{Column Proportions Table|see{Table, Proportion}}
\index{Table Proportions Table|see{Table, Proportion}}
\index{Proportions, Inference|see{Chi-square}}
\index{Rejection Criterion|see{$\alpha$}}
\index{Reverse calculation|see{Normal Distribution, Finding values}}
\index{Simple Linear Regression|see{Regression}}
\index{SLR|see{Regression}}
\index{Standard Normal Distribution|see{Normal Distribution}}
\index{Standardization|see{Normal Distribution, Converting to Z-scale}}
\index{One-sample Z-Test|see{Z-Test}}
\index{1-sample Z-Test|see{Z-Test}}
\index{One-sample t-Test|see{t-Test}}
\index{1-sample t-Test|see{t-Test}}
\index{Two-sample t-Test|see{t-Test}}
\index{2-sample t-Test|see{t-Test}}
\index{Matched-Pairs t-Test|see{t-Test}}
\index{Two-way Table|see{Table, Frequency}}
\index{Type I Error|see{Hypothesis Testing, Errors}}
\index{Type II Error|see{Hypothesis Testing, Errors}}
\index{Variability!Natural|see{Natural Variability}}
\index{Variability!Sampling|see{Sampling Variability}}
\index{Variable!Types|see{Quantitative, Continuous, Discrete, Categorical, Nominal, or Ordinal}}
\index{Variance!Testing Equality|see{Levene's Test}}
\index{Y-intercept|see{Intercept}}
\index{Z-Distribution|see{Normal Distribution, Standard Normal}}

% code to actually add the index
\addtocontents{toc}{\setlength{\cftbeforepartskip}{0.4em}}  %decrease dist before parts in back matter portion of TOC
\cleardoublepage                          %not sure why but this is needed so TOC entry will point to right start page
\phantomsection                           %not sure why but this is needed so TOC entry will point to right start page
\addcontentsline{toc}{part}{Index}        %Add a TOC entry
\printindex                               %Make the index


\end{document}
