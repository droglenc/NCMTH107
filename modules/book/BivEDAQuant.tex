

\chapter{Bivariate EDA - Quantitative} \label{chap:BivEDAQuant}

\vspace{-30pt}
\minitoc
\vspace{18pt}

\lettrine{B}{ivariate data occurs when two} variables are measured on the same individuals. For example, you may measure (i) the height and weight of students in class, (ii) depth and area of a lake, (iii) gender and age of welfare recipients, or (iv) number of mice and biomass of legumes in fields. This module is focused on describing the bivariate relationship between two quantitative variables. Bivariate relationships between two categorical variables is described in \modref{chap:BivEDACat}.

Data on the \var{weight} (lbs) and highway miles per gallon (\var{HMPG}) for 93 cars from the 1993 model year are used as an example throughout this module. Ultimately, the relationship between highway MPG and the weight of a car is described. A sample of these data are shown below
\begin{knitrout}
\definecolor{shadecolor}{rgb}{0.969, 0.969, 0.969}\color{fgcolor}\begin{kframe}


{\ttfamily\noindent\bfseries\color{errorcolor}{\#\# Error in headtail(cars93, which = c("{}MFG"{}, "{}Model"{}, "{}Type"{}, "{}CMPG"{}, "{}HMPG"{}, : could not find function "{}headtail"{}}}

{\ttfamily\noindent\bfseries\color{errorcolor}{\#\# Error in print(tmp, row.names = FALSE): object 'tmp' not found}}\end{kframe}
\end{knitrout}

\section[Response and Explanatory] {Response and Explanatory Variables} \label{sect:RespExplan1}
\vspace{-3pt}
The \textbf{response variable} is the variable that one is interested in explaining something (i.e., variability) or in making future predictions about. The \textbf{explanatory variable} is the variable that may help explain or allow one to predict the response variable. In general, the response variable is thought to depend on the explanatory variable. Thus, the response variable is often called the \textbf{dependent variable}, whereas the explanatory variable is often called the \textbf{independent variable}.

One may identify the response variable by determining which of the two variables depends on the other. For example, in the car data, highway MPG is the response variable because gas mileage is most likely affected by the weight of the car (e.g., hypothesize that heavier cars get worse gas mileage), rather than vice versa.

In some situations it is not obvious which variable is the response. For example, does the number of mice in the field depend on the number of legumes (lots of feed=lots of mice) or the other way around (lots of mice=not much food left)? Similarly, does area depend on depth or does depth depend on area of the lake? In these situations, the context of the research question is needed to identify the response variable. For example, if the researcher hypothesized that number of mice will be greater if there is more legumes, then number of mice is the response variable. In many cases, the more difficult variable to measure will likely be the response variable. For example, researchers likely wish to predict area of a lake (hard to measure) from depth of the lake (easy to measure).

\vspace{-9pt}
\warn{Which variable is the response may depend on the context of the research question.}


\vspace{-12pt}
\section{Summaries}
\vspace{-6pt}
\subsection{Scatterplots} \label{sect:ScatterplotsR}
\vspace{-3pt}
A scatterplot is a graph where each point simultaneously represents the values of both the quantitative response and quantitative explanatory variable. The value of the explanatory variable gives the x-coordinate and the value of the response variable gives the y-coordinate of the point plotted for an individual. For example, the first individual in the cars data is plotted at x (\var{Weight}) = 2705 and y (\var{HMPG}) = 31, whereas the second individual is at x = 3560 and y = 25 \figrefp{fig:carscat2}.


\begin{knitrout}
\definecolor{shadecolor}{rgb}{0.969, 0.969, 0.969}\color{fgcolor}\begin{kframe}


{\ttfamily\noindent\bfseries\color{errorcolor}{\#\# Error in ggplot(data = cars93, mapping = aes(x = Weight, y = HMPG)): could not find function "{}ggplot"{}}}\end{kframe}
\end{knitrout}























